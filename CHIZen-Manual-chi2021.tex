%% The first command in your LaTeX source must be the \documentclass command.
\documentclass[manuscript,screen]{acmart}

% Load CHIZen stuff
%
%  This file imports each of the separate .tex files in this directory -- feel free to enable/disable each separately as you'd like
%


%
%   A bunch of common includes for text formatting, features. 
%
%Robert's accented character fix
\usepackage[utf8]{inputenc}

% Enables use of Chinese characters
\usepackage{CJKutf8}


% ACCESSABILITY
% Experimental, no perfect solution yet... but need to try this
%.     -- Not enabled for now, not playing well with other packages -- markdown? Need to investigate: JW
%
%\usepackage[tagged,highstructure]{accessibility}

% MARKDOWN
%
% TODO: Explore how useful this is for general-purpose HCI papers
%       - Tables/Figures need to be a separate thing, but useful for everything else?
%
\usepackage[footnotes,definitionLists,hashEnumerators,smartEllipses,hybrid]{markdown}

\usepackage{paracol}
\usepackage{minted}

% This package is useful for revisions! 
\usepackage{savesym}
\savesymbol{comment}
\usepackage{changes}
%\usepackage[final]{changes}
\restoresymbol{SC}{comment}


% Mark likes to add these options to his papers .... worth considering
\frenchspacing
\raggedbottom

% Lennart's acronym package
\usepackage[nolist]{acronym}

\begin{acronym}
\acro{ANOVA}{Analysis of Variance}
\acro{HCI}{Human-Computer Interaction}
\end{acronym}


% Jim's quotation stuff
%\renewenvironment{quote}{%
%  \list{}{%
%    \leftmargin0.4cm   % this is the adjusting screw
%    \rightmargin\leftmargin
%  }
%  \item\relax
%}
%{\endlist}

% Jim added this -- useful highlighting function \hl{}
%\usepackage{soul,color}



% Study Variables
% ----------------------------

% scale up the sc font slightly
\usepackage{letltxmacro,lipsum,relsize}
\LetLtxMacro{\oldscshape}{\scshape}
\renewcommand{\scshape}[1]{\oldscshape\relscale{1.2}#1}

% an experiment factor/variable
\newcommand{\f}[1]{\mbox{{\textsc{#1}}}\xspace}
% usage: \f{block}, \f{width}, etc.
% NOTE: need \xspace package to fix spacing issues
%       \mbox prevents breaking

% a code when qualitative coding 
\newcommand{\cd}[1]{\textit{``{#1}''}\xspace}
% usage: \cd{errors}, \cd{work}


%
%   Figures -- formerly IDEA Plots -- Wraps most features from PGFPlots
%       * Handles common formatting, defines some common graph types
%       * See file for detailed documentation
%
%%%%%%%%%%%%%%%%%%%%%%%%%%%%%%%%%%%%%%%%%%%%%%%%%
%  HOW TO USE
%%%%%%%%%%%%%%%%%%%%%%%%%%%%%%%%%%%%%%%%%%%%%%%%%
%
% Include this line at the top of your TeX  File: 
%
% \input{IDEA_Plots.tex}
%
%
% Then, add graphs to your paper using the following template: 
%
%\begin{tikzpicture}
%\begin{axis}[<GRAPH TYPE>,
%	% X Axis options (xmin, xmax, xtick={})
%	% Y Axis options (ymin, ymax, ytick={})
%       ]
%
%    % DATA GOES IN HERE
%
%\end{axis}
%\end{tikzpicture}
%
%
% OR, for multiple plots use this template:
%
%\begin{tikzpicture}
%\begin{groupplot}[
%   group style={
%       group size= <COL> by <ROW>,
%       x descriptions at=edge bottom,
%       y descriptions at=edge left,
%       vertical sep=0pt,
%       horizontal sep=0pt},
%  <GRAPH TYPE>, 
%	% X Axis options (xmin, xmax, xtick={})
%	% Y Axis options (ymin, ymax, ytick={})
%      ]
%
%    % DATA GOES IN HERE
%
%\end{groupplot}
%\end{tikzpicture}
%
%
%
% The following graph types are defined in this library:
%
% IDEA bar - A clustered bar graph, similar to Excel
% IDEA tufte panel - a bare-bones bar graph, suitable for stacked panel graphs in Tufte's style
% .. more that aren't listed here
%




%%%%%%%%%%%%%%%%%%%%%%%%%%%%%%%%%%%%%%%%%%%%%%%%%
%   Common Includes
%%%%%%%%%%%%%%%%%%%%%%%%%%%%%%%%%%%%%%%%%%%%%%%%%
\usepackage{pgfplots}
\usepackage{pgfplotstable}
\usepgfplotslibrary{dateplot}
\usepgfplotslibrary{groupplots}
\usepackage{pgfgantt}

\usetikzlibrary{external}
%\tikzexternalize[prefix=tikz/] % Must be in main .tex file?

\usetikzlibrary{pgfplots.statistics}
\usetikzlibrary{matrix}

\usepackage{bbding}

\pgfplotsset{compat=newest}

%%%%%%%%%%%%%%%%%%%%%%%%%%%%%%%%%%%%%%%%%%%%%%%%%
%   Colour Definitions
%%%%%%%%%%%%%%%%%%%%%%%%%%%%%%%%%%%%%%%%%%%%%%%%%

\definecolor{RYB1}{RGB}{141, 211, 199}
\definecolor{RYB2}{RGB}{255, 255, 179}
\definecolor{RYB3}{RGB}{190, 186, 218}
\definecolor{RYB4}{RGB}{251, 128, 114}
\definecolor{RYB5}{RGB}{128, 177, 211}
\definecolor{RYB6}{RGB}{253, 180, 98}
\definecolor{RYB7}{RGB}{179, 222, 105}

\pgfplotscreateplotcyclelist{colorbrewer-RYB}{
{RYB1!50!black,fill=RYB1},
{RYB2!50!black,fill=RYB2},
{RYB3!50!black,fill=RYB3},
{RYB4!50!black,fill=RYB4},
{RYB5!50!black,fill=RYB5},
{RYB6!50!black,fill=RYB6},
{RYB7!50!black,fill=RYB7},
}

\pgfplotscreateplotcyclelist{colorbrewer-RYB-plain}{
{RYB1},
{RYB2},
{RYB3},
{RYB4},
{RYB5},
{RYB6},
{RYB7},
}

%%%%%%%%%%%%%%%%%%%%
% WATERLOO PALETTES - DIGITAL
%%%%%%%%%%%%%%%%%%%%

\definecolor{AHSLight}{RGB}{0, 154, 166}
\definecolor{AHSDark}{RGB}{0, 127, 138}

\definecolor{ArtsLight}{RGB}{233, 131, 0}
\definecolor{ArtsDark}{RGB}{172, 97, 0}

\definecolor{EngineeringLight}{RGB}{204, 170, 255}
\definecolor{EngineeringDark}{RGB}{87, 6, 140}

\definecolor{EnvironmentLight}{RGB}{182, 191, 0}
\definecolor{EnvironmentDark}{RGB}{116, 120, 0}

\definecolor{MathLight}{RGB}{255, 136, 221}
\definecolor{MathDark}{RGB}{224, 36, 154}
 
\definecolor{ScienceLight}{RGB}{119, 187, 225}
\definecolor{ScienceDark}{RGB}{0, 115, 207}


\definecolor{SchoolRedLight}{RGB}{247, 119, 119}
\definecolor{SchoolRedDark}{RGB}{150, 23, 46}



\pgfplotscreateplotcyclelist{waterloo-light}{
{AHSLight!50!black,fill=AHSLight},
{ArtsLight!50!black,fill=ArtsLight},
{EngineeringLight!50!black,fill=EngineeringLight},
{EnvironmentLight!50!black,fill=EnvironmentLight},
{MathLight!50!black,fill=MathLight},
{ScienceLight!50!black,fill=ScienceLight},
{SchoolRedLight!50!black,fill=SchoolRedLight},
}

\pgfplotscreateplotcyclelist{waterloo-dark}{
{AHSDark!50!black,fill=AHSDark},
{ArtsDark!50!black,fill=ArtsDark},
{EngineeringDark!50!black,fill=EngineeringDark},
{EnvironmentDark!50!black,fill=EnvironmentDark},
{MathDark!50!black,fill=MathDark},
{ScienceDark!50!black,fill=ScienceDark},
{SchoolRedDark!50!black,fill=SchoolRedDark},
}






%%%%%%%%%%%%%%%%%%%%
% 5-Point DIVERGENT PALETTES - From ColorBrewer2
%%%%%%%%%%%%%%%%%%%%

\definecolor{Likert5_SD}{RGB}{166,97,26}
\definecolor{Likert5_D}{RGB}{223,194,125}
\definecolor{Likert5_N}{RGB}{245,245,245}
\definecolor{Likert5_A}{RGB}{128,205,193}
\definecolor{Likert5_SA}{RGB}{1,133,113}

\definecolor{Likert5_1_SD}{RGB}{166,97,26}
\definecolor{Likert5_1_D}{RGB}{223,194,125}
\definecolor{Likert5_1_N}{RGB}{245,245,245}
\definecolor{Likert5_1_A}{RGB}{128,205,193}
\definecolor{Likert5_1_SA}{RGB}{1,133,113}

\definecolor{Likert5_2_SD}{RGB}{123,50,148}
\definecolor{Likert5_2_D}{RGB}{194,165,207}
\definecolor{Likert5_2_N}{RGB}{247,247,247}
\definecolor{Likert5_2_A}{RGB}{166,219,160}
\definecolor{Likert5_2_SA}{RGB}{0,136,55}



%%%%%%%%%%%%%%%%%%%%
% 7-Point DIVERGENT PALETTES - From ColorBrewer2
%%%%%%%%%%%%%%%%%%%%

% Purple -> Green
\definecolor{Likert7_0}{RGB}{118,42,131}
\definecolor{Likert7_1}{RGB}{175,141,195}
\definecolor{Likert7_2}{RGB}{231,212,232}
\definecolor{Likert7_3}{RGB}{229,229,229}
\definecolor{Likert7_4}{RGB}{217,240,211}
\definecolor{Likert7_5}{RGB}{127,191,123}
\definecolor{Likert7_6}{RGB}{27,120,55}

% Orange/Brown -> Blue
\definecolor{Likert7_2_0}{RGB}{140,81,10}
\definecolor{Likert7_2_1}{RGB}{216,179,101}
\definecolor{Likert7_2_2}{RGB}{246,232,195}
\definecolor{Likert7_2_3}{RGB}{229,229,229}
\definecolor{Likert7_2_4}{RGB}{199,234,229}
\definecolor{Likert7_2_5}{RGB}{90,180,172}
\definecolor{Likert7_2_6}{RGB}{1,102,94}

% Orange -> Purple
\definecolor{Likert7_3_0}{RGB}{179,88,6}
\definecolor{Likert7_3_1}{RGB}{241,163,64}
\definecolor{Likert7_3_2}{RGB}{254,224,182}
\definecolor{Likert7_3_3}{RGB}{229,229,229}
\definecolor{Likert7_3_4}{RGB}{216,218,235}
\definecolor{Likert7_3_5}{RGB}{153,142,195}
\definecolor{Likert7_3_6}{RGB}{84,39,136}

% Blue -> Red
\definecolor{Likert7_4_0}{RGB}{33,102,172}
\definecolor{Likert7_4_1}{RGB}{103,169,207}
\definecolor{Likert7_4_2}{RGB}{209,229,240}
\definecolor{Likert7_4_3}{RGB}{229,229,229}
\definecolor{Likert7_4_4}{RGB}{253,219,199}
\definecolor{Likert7_4_5}{RGB}{239,138,98}
\definecolor{Likert7_4_6}{RGB}{178,24,43}


%%%%%%%%%%%%%%%%%%%%
% 7-Point DIVERGENT PALETTES - Jim for Avatar Paper
%%%%%%%%%%%%%%%%%%%%

% Blue -> Green to match Cleric
\definecolor{Cleric_0}{RGB}{33,102,172}
\definecolor{Cleric_1}{RGB}{103,169,207}
\definecolor{Cleric_2}{RGB}{209,229,240}
\definecolor{Cleric_3}{RGB}{229,229,229}
\definecolor{Cleric_4}{RGB}{217,240,211}
\definecolor{Cleric_5}{RGB}{127,191,123}
\definecolor{Cleric_6}{RGB}{27,120,55}

% Orange -> Red
\definecolor{Monster_0}{RGB}{178,24,43}
\definecolor{Monster_1}{RGB}{239,138,98}
\definecolor{Monster_2}{RGB}{253,219,199}
\definecolor{Monster_3}{RGB}{229,229,229}
\definecolor{Monster_4}{RGB}{216,218,235}
\definecolor{Monster_5}{RGB}{153,142,195}
\definecolor{Monster_6}{RGB}{84,39,136}


\pgfplotscreateplotcyclelist{avatar}{Cleric_0, Monster_0}

%%%%%%%%%%%%%%%%%%%%%%%%%%%%%%%%%%%%%%%%%%%%%%%%%
%   Outer South Legend Hack
%%%%%%%%%%%%%%%%%%%%%%%%%%%%%%%%%%%%%%%%%%%%%%%%%
\makeatletter
\pgfplotsset{
    every axis x label/.append style={
        alias=current axis xlabel
    },
    legend pos/outer south/.style={
        /pgfplots/legend style={
            at={%
                (%
                \@ifundefined{pgf@sh@ns@current axis xlabel}%
                {xticklabel cs:0.5}%
                {current axis xlabel.south}%
                )%
            },
            anchor=north
        }
    }
}

%%%%%%%%%%%%%%%%%%%%%%%%%%%%%%%%%%%%%%%%%%%%%%%%%
%   Grid Figures Label Hack
%%%%%%%%%%%%%%%%%%%%%%%%%%%%%%%%%%%%%%%%%%%%%%%%%

\pgfplotsset{
    groupplot xlabel/.initial={},
    every groupplot x label/.style={
        at={($({\pgfplots@group@name\space c1r\pgfplots@group@rows.west}|-{\pgfplots@group@name\space c1r\pgfplots@group@rows.outer south})!0.5!({\pgfplots@group@name\space c\pgfplots@group@columns r\pgfplots@group@rows.east}|-{\pgfplots@group@name\space c\pgfplots@group@columns r\pgfplots@group@rows.outer south})$)},
        anchor=north,
    },
    groupplot ylabel/.initial={},
    every groupplot y label/.style={
            rotate=90,
        at={($({\pgfplots@group@name\space c1r1.north}-|{\pgfplots@group@name\space c1r1.outer
west})!0.5!({\pgfplots@group@name\space c1r\pgfplots@group@rows.south}-|{\pgfplots@group@name\space c1r\pgfplots@group@rows.outer west})$)},
        anchor=south
    },
    execute at end groupplot/.code={%
      \node [/pgfplots/every groupplot x label]
{\pgfkeysvalueof{/pgfplots/groupplot xlabel}};  
      \node [/pgfplots/every groupplot y label] 
{\pgfkeysvalueof{/pgfplots/groupplot ylabel}};  
    }
}

\def\endpgfplots@environment@groupplot{%
    \endpgfplots@environment@opt%
    \pgfkeys{/pgfplots/execute at end groupplot}%
    \endgroup%
}



%%%%%%%%%%%%%%%%%%%%%%%%%%%%%%%%%%%%%%%%%%%%%%%%%
%   IDEA Line
%%%%%%%%%%%%%%%%%%%%%%%%%%%%%%%%%%%%%%%%%%%%%%%%%
\pgfplotsset{
  IDEA line/.style={
 	cycle list name = avatar,
        width  = \columnwidth,
        height = 6cm,
        line width=10pt,
        	axis line style = very thin,
        font = \sffamily,
        % X Axis
        %major x tick style = transparent,
        xlabel style={name=xlabel},
	% Y Axis
        ymajorgrids = true,
        scaled y ticks = false,
        separate axis lines,
	axis lines* = left,
        ymajorgrids = true,
        major tick length = -5,
        %Legend
        legend style={at={(xlabel.south)},yshift=2ex, xshift=-2ex, anchor=north, legend columns = 3, draw=none, column sep = 0.25cm},
        	%legend image code/.code={%
            %        \draw[#1,fill] circle (0.075cm);
            %    },
	}
}

%%%%%%%%%%%%%%%%%%%%%%%%%%%%%%%%%%%%%%%%%%%%%%%%%
%   IDEA Bar / A Clustered Bar Graph
%%%%%%%%%%%%%%%%%%%%%%%%%%%%%%%%%%%%%%%%%%%%%%%%%
\pgfplotsset{
  IDEA bar/.style={
  	cycle list name = waterloo-light,
	ybar = 0pt,
        width  = \columnwidth,
        height = 6cm,
        	axis line style = very thin,
       	bar width=25pt,
        font = \sffamily,
        	% X Axis
        	major x tick style = transparent,
        	enlarge x limits = {abs=1.5cm},
	xlabel={a},
        	xlabel style={name=xlabel},
	% Y Axis
        	ymajorgrids = true,
        	ylabel = {Selection Time (s)},
        	scaled y ticks = false,
        	separate axis lines,
	axis lines* = left,
        	ymajorgrids = true,
        	major tick length = 0,
        %Legend
        	legend style={at={(xlabel.south)},yshift=2ex, xshift=-2ex, anchor=north, legend columns = 3, draw=none, column sep = 0.25cm},
        	legend image code/.code={\draw[#1] rectangle (0.25cm,0.25cm);}
	}
}

%%%%%%%%%%%%%%%%%%%%%%%%%%%%%%%%%%%%%%%%%%%%%%%%%
%   IDEA StackedBar / A Clustered Bar Graph
%%%%%%%%%%%%%%%%%%%%%%%%%%%%%%%%%%%%%%%%%%%%%%%%%
\pgfplotsset{
  IDEA stacked bar/.style={
  	ybar stacked,
  	cycle list name = waterloo-light,
        width  = \columnwidth,
        height = 6cm,
        	axis line style = very thin,
       	bar width=25pt,
       	font = \sffamily,
        	% X Axis
        	major x tick style = transparent,
        	enlarge x limits = {abs=1.5cm},
	xlabel={a},
        	xlabel style={name=xlabel},
	% Y Axis
        	ymajorgrids = true,
        	ylabel = {Selection Time (s)},
        	scaled y ticks = false,
        	separate axis lines,
	axis lines* = left,
        	ymajorgrids = true,
        	major tick length = 0,
        %Legend
        	legend style={at={(xlabel.south)},yshift=2ex, xshift=-2ex, anchor=north, legend columns = 3, draw=none, column sep = 0.25cm},
        	legend image code/.code={\draw[#1] rectangle (0.25cm,0.25cm);}
	}
}


%%%%%%%%%%%%%%%%%%%%%%%%%%%%%%%%%%%%%%%%%%%%%%%%%
%   IDEA Likhert / Divergent Bar
%%%%%%%%%%%%%%%%%%%%%%%%%%%%%%%%%%%%%%%%%%%%%%%%%
\pgfplotsset{
  IDEA Likert/.style={
 	xbar stacked,
        	axis line style = very thin,
       	bar width=5pt,
        	% X Axis
        	%major x tick style = transparent,
	xlabel={a},
        	xlabel style={name=xlabel},
	xtick = {},
	font = \sffamily,
	% Y Axis
	        %axis y line=none,
	        major y tick style = transparent,
        	%ymajorgrids = true,
        	scaled y ticks = false,
        	separate axis lines,
	axis lines* = left,
        	major tick length = -2,
        %Legend
        	legend style={at={(xlabel.south)},yshift=-2ex, xshift=-2ex, anchor=north, legend columns = 3, draw=none, column sep = 0.25cm},
        	legend image code/.code={\draw[#1] rectangle (0.25cm,0.25cm);}
	}
}

%%%%%%%%%%%%%%%%%%%%%%%%%%%%%%%%%%%%%%%%%%%%%%%%%
%   Tufte Stacked Panel 
%%%%%%%%%%%%%%%%%%%%%%%%%%%%%%%%%%%%%%%%%%%%%%%%%
\pgfplotsset{
  IDEA tufte panel/.style={
  cycle list name = waterloo-light,
 ybar, 
	% Generic Bar Graph Options
        width = 1.1\columnwidth,
        height = .325\columnwidth,
        font = \sffamily,
	% X Axis
	enlarge x limits = {abs=.35cm},
	axis line style = very thin,
	% Y Axis
	yticklabel=\empty,
        separate axis lines,
	y axis line style= { draw opacity=0 },
	axis lines* = left,
        axis on top,
        ymajorgrids = true,
        major tick length = 0,
        grid style = {white},
	% Legend Formatting
	title style = {yshift=-.75cm,font=\sffamily\tiny}
	}
}

%%%%%%%%%%%%%%%%%%%%%%%%%%%%%%%%%%%%%%%%%%%%%%%%%
%   Tufte Range Frame
%%%%%%%%%%%%%%%%%%%%%%%%%%%%%%%%%%%%%%%%%%%%%%%%%
\pgfplotsset{
range frame/.style={
    tick align=outside,
    axis line style={opacity=0},
    after end axis/.code={
      \draw ({rel axis cs:0,0}
          -|{axis cs:\pgfplotsdataxmin,0})
        -- ({rel axis cs:0,0}
          -|{axis cs:\pgfplotsdataxmax,0});
      \draw ({rel axis cs:0,0}
          |-{axis cs:0,\pgfplotsdataymin})
        -- ({rel axis cs:0,0}
          |-{axis cs:0,\pgfplotsdataymax});
} }
}

%%%%%%%%%%%%%%%%%%%%%%%%%%%%%%%%%%%%%%%%%%%%%%%%%
%   Tufte Dot Dash 
%%%%%%%%%%%%%%%%%%%%%%%%%%%%%%%%%%%%%%%%%%%%%%%%%

\pgfplotsset{
  dot dash plot/.style={
    tufte scatter
    axis line style={opacity=0},
    tick style={thin, black},
    major tick length=0.15cm,
    xtick=data,
    xticklabels={},
    ytick=data,
    yticklabels={},
    extra x ticks={
      \pgfplotsdataxmin,
      \pgfplotsdataxmax
    },
    extra y ticks={
      \pgfplotsdataymin,
      \pgfplotsdataymax
    },
    extra tick style={
      xticklabel={\pgfmathprintnumber[
        fixed,
        fixed zerofill,
        precision=1
      ]{\tick}},
      yticklabel={\pgfmathprintnumber[
        fixed,
        fixed zerofill,
        precision=1
]{\tick}} }
} }

%%%%%%%%%%%%%%%%%%%%%%%%%%%%%%%%%%%%%%%%%%%%%%%%%
%   Tufte Box Plot 
%%%%%%%%%%%%%%%%%%%%%%%%%%%%%%%%%%%%%%%%%%%%%%%%%
\pgfplotsset{
  IDEA tufte box/.style={
	% Colour Options
	cycle list name = waterloo-light,
	ybar = 0pt,
        width  = \columnwidth,
        height = 6cm,
        	axis line style = very thin,
       	bar width=25pt,
       	font = \sffamily,
        	% X Axis
        	major x tick style = transparent,
        	enlarge x limits = {abs=1.5cm},
        	xlabel style={name=xlabel},
	% Y Axis
        	ymajorgrids = true,
        	scaled y ticks = false,
        	separate axis lines,
	axis lines* = left,
        	ymajorgrids = true,
        	major tick length = 0,
        %Legend
        	legend style={at={(xlabel.south)},yshift=2ex, xshift=-2ex, anchor=north, legend columns = 3, draw=none, column sep = 0.25cm},
        	legend image code/.code={\draw[#1] rectangle (0.25cm,0.25cm);}
 	% Boxplot specific options
	boxplot,
	boxplot/draw direction=y, 
    	%clip = false,
    	every axis plot/.style={
      	mark=o,
      	boxplot/draw direction=y,
      	boxplot/whisker extend=0,
      	boxplot/draw/box/.code={ },
	boxplot/draw/median/.code={%
	 \draw[mark size=2pt,/pgfplots/boxplot/every median/.try]
          \pgfextra
          \pgftransformshift{
            \pgfplotsboxplotpointabbox
              {\pgfplotsboxplotvalue{median}}
              {0.5}
          }
          \pgfsetfillcolor{black}
          \pgfuseplotmark{*}
          \endpgfextra
        	;
      	},
      }
  },
}


%%%%%%%%%%%%%%%%%%%%%%%%%%%%%%%%%%%%%%%%%%%%%%%%%
%   IDEA Box Plot 
%%%%%%%%%%%%%%%%%%%%%%%%%%%%%%%%%%%%%%%%%%%%%%%%%

\pgfplotsset{
    IDEA box/.style={
    	% Colour Options
	cycle list name = waterloo-light,
	ybar = 0pt,
        width  = \columnwidth,
        height = 6cm,
        axis line style = very thin,
        font = \sffamily,
        	% X Axis
        	major x tick style = transparent,
        	enlarge x limits = {abs=1cm},
        	xlabel style={name=xlabel},
	% Y Axis
        	ymajorgrids = true,
        	scaled y ticks = false,
        	separate axis lines,
	axis lines* = left,
        	ymajorgrids = true,
        	major tick length = 0,
        %Legend
        	legend style={at={(xlabel.south)},yshift=2ex, xshift=-2ex, anchor=north, legend columns = 3, draw=none, column sep = 0.25cm},
        	legend image code/.code={\draw[#1] rectangle (0.25cm,0.25cm);}
        % draw whiskers as a single line:
        boxplot/draw/whisker/.code 2 args={%
            \draw[/pgfplots/boxplot/every whisker/.try]
                (boxplot cs:##1) -- (boxplot cs:##2)
            ;
        },%
        %
        % fill the boxes:
        boxplot/every box/.style={
            fill,
        },
        % 
        % the median should be visualized as a thick white line:
        boxplot/every median/.style={
        		ultra thick,
		fill=white, draw=white,
        },
        %boxplot/draw/median/.code={%
        %    \draw[fill=white]
        %        (boxplot cs:\pgfplotsboxplotvalue{median}) circle (3pt)
        %    ;
        %},
        % draw the average as a circle
        boxplot/average=auto,
        boxplot/draw/average/.code={%
	   \draw[fill=white,draw=gray]
            	(boxplot cs:\pgfplotsboxplotvalue{average}) circle (2pt)
            ;
        },
        %
        % do not clip to avoid problems with the median:
        clip=false,
        %
        boxplot/draw direction=y,
        boxplot/whisker extend=0,
        boxplot/every whisker/.style = very thick,
        %
        %
        % width of boxes:
        boxplot/box extend=0.15,
    },
    %
    %
    rshift/.style={
        xshift=\pgfkeysvalueof{/pgfplots/rshift scale},
        legend image post style={xshift=-\pgfkeysvalueof{/pgfplots/rshift scale}},
    },
    lshift/.style={
        xshift=-\pgfkeysvalueof{/pgfplots/lshift scale},
        legend image post style={xshift=\pgfkeysvalueof{/pgfplots/lshift scale}},
    },
    rshift scale/.initial=1em,
    lshift scale/.initial=1em,
}

%%%%%%%%%%%%%%%%%%%%%%%%%%%%%%%%%%%%%%%%%%%%%%%%%
% allows multiple lines for the same Y variable in stacked plots
%%%%%%%%%%%%%%%%%%%%%%%%%%%%%%%%%%%%%%%%%%%%%%%%%
\newcommand\resetstackedplots{
\makeatletter
\pgfplots@stacked@isfirstplottrue
\makeatother
}

%%%%%%%%%%%%%%%%%%%%%%%%%%%%%%%%%%%%%%%%%%%%%%%%%
% EOF
%%%%%%%%%%%%%%%%%%%%%%%%%%%%%%%%%%%%%%%%%%%%%%%%%
\makeatother

% This caches the PGF Plots stuff to speed up compile times
\usetikzlibrary{external}
\tikzexternalize[prefix=submission/tikz/]


%
%   Tables
%
%appendix tables
\usepackage{longtable}
\usepackage{float}
\usepackage{multicol}


% Katja's dashed lines in table stuff
\usepackage{arydshln}
\makeatletter
\def\adl@drawiv#1#2#3{%
	\hskip.5\tabcolsep
	\xleaders#3{#2.5\@tempdimb #1{1}#2.5\@tempdimb}%
	#2\z@ plus1fil minus1fil\relax
	\hskip.5\tabcolsep}
\newcommand{\cdashlinelr}[1]{%
	\noalign{\vskip\aboverulesep
		\global\let\@dashdrawstore\adl@draw
		\global\let\adl@draw\adl@drawiv}
	\cdashline{#1}
	\noalign{\global\let\adl@draw\@dashdrawstore
		\vskip\belowrulesep}}
\makeatother


%
%   Stats
%
\usepackage{etoolbox}
\usepackage{siunitx}

\sisetup{
    % range-phrase = --, % make ranges use en-dash instead of "to" (e.g., ages)
    round-mode = places,
    round-precision = 1, % can change precision globally
    round-half = even
}

\newcommand{\fprecision}{1}
\newcommand{\chisqprecision}{1}
\newcommand{\formatprob}[2]{%
    \ifdimless{#2 pt}{.001 pt}{$#1 < .001$}{%
        \ifdimless{#2 pt}{.01 pt}{$#1 < .01$}{%
            $#1 = \num[add-integer-zero=false,%
                      round-mode=places,%
                      round-precision=2,%
                      round-half=even]{#2}$%
        }%
    }%
}
\newcommand{\p}[1]{\formatprob{p}{#1}}
\newcommand{\etasqp}[1]{\formatprob{\eta_{p}^2}{#1}}
\newcommand{\F}[3]{{$F_{#1,#2}=\num[round-mode=places,%
                                    round-precision=\fprecision]{#3}$}}
\newcommand{\owF}[2]{{$F_{#1}=\num[round-mode=places,%
                                   round-precision=\fprecision]{#2}$}}
\newcommand{\chisq}[2]{$\chi^2(#1)=\num[round-mode=places,%
                                        round-precision=\chisqprecision]{#2}$}
\newcommand{\sd}[2][]{$SD=\SI{#2}{#1}$}
\newcommand{\mean}[2][]{$M=\SI{#2}{#1}$}
\newcommand{\mdn}[2][]{$Mdn=\SI{#2}{#1}$}
\newcommand{\ages}[2]{Ages \numrange{#1}{#2}}
\newcommand{\gender}[2]{\num{#1} identified as #2}
\newcommand{\genderlist}[2]{\numlist{#1} identified as #2, respectively}
\newcommand{\by}{\,$\times$\,}
\newcommand{\size}[3][]{\SIrange[range-phrase=\by,%
                                 range-units=single]{#2}{#3}{#1}}
\newcommand{\sizen}[2][]{\SIlist[list-separator=\by,%
                                 list-final-separator=\by,%
                                 list-pair-separator=\by,%
                                %  list-units=single% % repeat (default),single,brackets
                                 ]{#2}{#1}}
\newcommand{\msd}[3][]{\mean[#1]{#2}, \sd[#1]{#3}}
\newcommand{\anova}[4]{\F{#1}{#2}{#3}, \p{#4}}
\newcommand{\anovae}[5]{\F{#1}{#2}{#3}, \p{#4}, \etasqp{#5}}
\newcommand{\owanova}[4]{\owF{#1}{#2}, \p{#3}, \etasqp{#4}}
\newcommand{\chisquare}[3]{\chisq{#1}{#2}, \p{#3}}


%
%   Bibliography
%
% Jim added this -- sorts citations numerically! 
\usepackage[nocompress]{cite}




%%
%% \BibTeX command to typeset BibTeX logo in the docs
\AtBeginDocument{%
  \providecommand\BibTeX{{%
    \normalfont B\kern-0.5em{\scshape i\kern-0.25em b}\kern-0.8em\TeX}}}

%% Rights management information.  This information is sent to you
%% when you complete the rights form.  These commands have SAMPLE
%% values in them; it is your responsibility as an author to replace
%% the commands and values with those provided to you when you
%% complete the rights form.
% \setcopyright{acmcopyright}
% \copyrightyear{2020}
% \acmYear{2020}
% \acmDOI{XX.XXXX/XXXXXXX.XXXXXXX}


%%
%% These commands are for a JOURNAL article.
% \acmJournal{PACMHCI}
% \acmVolume{0}
% \acmNumber{0}
% \acmArticle{000}
% \acmMonth{0}

%%
%% Submission ID.
%% Use this when submitting an article to a sponsored event. You'll
%% receive a unique submission ID from the organizers
%% of the event, and this ID should be used as the parameter to this command.
%%\acmSubmissionID{123-A56-BU3}

%%
%% The majority of ACM publications use numbered citations and
%% references.  The command \citestyle{authoryear} switches to the
%% "author year" style.
%%
%% If you are preparing content for an event
%% sponsored by ACM SIGGRAPH, you must use the "author year" style of
%% citations and references.
%% Uncommenting
%% the next command will enable that style.
%%\citestyle{acmauthoryear}


%%
%% end of the preamble, start of the body of the document source.
\begin{document}

%%
%% The "title" command has an optional parameter,
%% allowing the author to define a "short title" to be used in page headers.
\title[Examples, Tips, and Hacks for Continuous Improvement of HCI Research Papers]{CHI Zen: Examples, Tips, and Hacks for Continuous Improvement of HCI Research Papers}

%%
%% The "author" command and its associated commands are used to define
%% the authors and their affiliations.
%% Of note is the shared affiliation of the first two authors, and the
%% "authornote" and "authornotemark" commands
%% used to denote shared contribution to the research.
\author{James R. Wallace}
\affiliation{%
  \institution{University of Waterloo}
}
\email{james.wallace@uwaterloo.ca}


%\renewcommand{\shortauthors}{F. Author et al.}

%%
%% By default, the full list of authors will be used in the page
%% headers. Often, this list is too long, and will overlap
%% other information printed in the page headers. This command allows
%% the author to define a more concise list
%% of authors' names for this purpose.
%\renewcommand{\shortauthors}{Author1, Author2, & Author3}



%%
%% The abstract is a short summary of the work to be presented in the
%% article.

\begin{abstract}
\begin{figure}[H]
    \centering
    \href{https://chi2022.acm.org}{\includegraphics[width=.32\textwidth]{figures/CHI-2022.png}}
    %\includegraphics[width=.32\textwidth]{figures/university-of-waterloo-vertical-logo.png} 
    \href{http://hci.uwaterloo.ca}{\includegraphics[width=.45\textwidth]{figures/uwhci-full-cs.png}}
    %\caption{Example teaser figure.}
    %\label{fig:teaser}
\end{figure}

\end{abstract}

%%
%% The code below is generated by the tool at http://dl.acm.org/ccs.cfm.
%% Please copy and paste the code instead of the example below.
%%
% \begin{CCSXML}
% <ccs2012>
% <concept>
% <concept_id>10003120.10003130.10011762</concept_id>
% <concept_desc>Human-centered computing~Empirical studies in collaborative and social computing</concept_desc>
% <concept_significance>500</concept_significance>
% </concept>
% </ccs2012>
% \end{CCSXML}

% \ccsdesc[500]{Human-centered computing~Empirical studies in collaborative and social computing}

%%
%% Keywords. The author(s) should pick words that accurately describe
%% the work being presented. Separate the keywords with commas.
%\keywords{mental health; peer-to-peer; serious game; cognitive behavioural therapy; proteus effect;}


%%
%% This command processes the author and affiliation and title
%% information and builds the first part of the formatted document.
\maketitle



%\vspace{2em}
%\begin{quote}
%    \emph{Great things are done by a series of small things brought together. --- Vincent Van Gough}
%\end{quote}

\begin{markdown}
%%%%%%%%%%%%%%%%%%%%%%%%%%%%%%%%%%%%%%%%%%%%%%%%%%%%%%%%%%%%%%%%%%%%%%%%%%%%%
%\section{Introduction}
%%%%%%%%%%%%%%%%%%%%%%%%%%%%%%%%%%%%%%%%%%%%%%%%%%%%%%%%%%%%%%%%%%%%%%%%%%%%%

# Introduction
Even the most experienced researchers are constantly looking for ways to improve their papers. And there are many different parts of a paper you can improve: from high-level concerns like experimental design, positioning the paper within the literature, and consideration of ethical issues, to sweating the details in a statistical analysis, or getting that graph *just right* to show off your earth-shattering results. But sometimes it's hard to put all the pieces together. That's where this project comes in. 
\end{markdown}

This project --- CHI Zen, a play on the words \begin{CJK}{UTF8}{min}改善\end{CJK} (`KaiZen') or `continuous improvement' --- describes common pitfalls and best practices based on my experience writing academic papers. That experience reflects some of my own preferences, but also a lot that I've learned from working with other HCI researchers. It also contains a lot of examples that you are free to copy, modify, or re-use as often as you'd like. The goal is to have something of an \emph{information zoo}~\citep{heer2010tour} that can make the process of creating that groundbreaking paper a little easier. 

\begin{markdown}
CHI Zen grew out of a a `CHI Paper Checklist' that I developed as a resource to help both myself and students as a paper deadline approached. It encapsulates some important advice I'd collected as a PhD student and Early Career Researcher, including bigger questions like how to frame your contributions and how early you should be seeking out peer feedback, as well as fine-grained issues like common mistakes people make when citing the literature. I also included some stylistic preferences, like words and phrases to avoid in academic writing. That checklist is now provided in the Appendix.

Th bulk of document provides a sample reference for a set of templates that I developed for \texttt{PGFPlots}. Please use \texttt{PGFPlots} if you find it useful. If you do, I'd love to hear about it. I'd also love to hear from you if you have any ways to improve them. You might also prefer to use a different graphing package, and that's cool too. The examples contained in this document hopefully can provide tips or shortcuts for use in your own papers. I suspect that you can generate more or less the same graphs with your tool of choice, be it R, a Python library, or Excel. Maybe those are an even better fit for your needs; if that's the case, you should use them instead! 

Beyond that, I've slowly accumulated information about best practices, such as transparency of research and reporting of statistics, as well as ways of simplifying the (often complex) workflow of academic publishing. For instance, I've most recently added a section on change tracking, to match the recent shift in many SIG CHI conferences to the *Proceedings of the ACM* journal, which often requires authors to revise and resubmit manuscript drafts with tracked changes. I've tried to summarize what I know, point to useful packages, and provide links to additional resources where I can. 

\end{markdown}





\begin{markdown}
# How to Use CHI Zen

This package brings together examples, tips, and hacks from a variety of different sources, and regarding just about every aspect of writing a research paper. You may find it all useful, or might only want to borrow bits and pieces. It has been designed to useful from a variety of perspectives, and these are only a few of the intended use cases:  

\subsection*{As a Template}

CHI Zen is written using the same \href{https://www.acm.org/publications/authors/submissions}{ACM Template that SIG CHI Conferences use}. It is divided into two parts: this manual, and a new submission. The new submission is contained in the `submission' folder, with placeholders for common paper sections like an Introduction, Related Work, and Discussion. These sections are pulled together in ``\texttt{00-my-new-paper.tex}'' in the main folder, which also pre-loads all of the \LaTeX\ packages and visualization templates described here. It is intended to be a quick way to start writing your own CHI paper.

\subsection*{As an Information Zoo}

Sections \ref{sec:code}, \ref{sec:tables}, and especially \ref{sec:figures} are an *information zoo* --- they contain annotated, working examples that you can use to help you think about how to best present data in your research. The intention *is not* to define right or wrong ways to present your work, but to provide a starting point, and point to the info vis literature to explain some of the considerations you should make when implementing those graphs in your own paper.


\subsection*{As a Quick Reference and a Checklist}

Sections \ref{sec:referencing} and \ref{sec:statistics} provide information about some common \LaTeX\ questions, like referencing the literature and reporting statistics. Section \ref{sec:transparency} and \autoref{sec:checklist} provide some guidance on how to manage the *process* of writing a CHI paper. Section \ref{sec:transparency} synthesizes some emerging best practices for transparency and replicability, and points to resources for implementing them in your own work. Section \ref{sec:checklist} is an actual checklist that you can print out and use as a summative evaluation tool for your paper, preferably a week or two before the paper deadline. 


# Comments, Suggestions, Feedback

Finally, my intention in making this project available via GitHub is to allow it to grow. If you find it useful, or can suggest a way improve it, I'd love to hear about it --- drop me a line at \href{mailto:james.wallace@uwaterloo.ca}{james.wallace@uwaterloo.ca}. It might also just be an important prompt to have a conversation about best practices, or an opportunity to explore different ways of thinking about or presenting research. In any case, I hope that it's an opportunity to think about how we can improve our research practices, one small step at a time. 

\end{markdown}


% \newpage 

% \begin{paracol}{2}
% \setlength{\parindent}{0em}
% \begin{CJK*}{UTF8}{gbsn}
% 其安易持, \\
% 其未兆易謀,\\
% 其脆易破,\\
% 其微易散。\\
% 為之于未有,\\
% 治之于未亂。\\
% 合抱之木,生于毫末;\\
% 九層之臺起于累土; \\
% 千里之行,始于足下。 \\
% 為者敗之,\\
% 執者失之。\\
% 聖人無為,故無敗;\\
% 無執,故無失。\\
% 民之徒事,常于幾成而敗之。\\
% 慎終如始,\\
% 則無敗事。\\
% 是以聖人欲不欲,\\
% 不貴難得之貨。\\
% 學不學,\\
% 復眾人之所過。\\
% 以輔萬物之自然,\\
% 而不敢為\\
% \flushright{-- 老子}
% \end{CJK*}

% \switchcolumn
% That settled is easily maintained, \\
% That without signs is easily conspired, \\
% That fragile is easily shattered, \\
% That insignificant is easily dispersed. \\
% Act on it before it materializes, \\
% Manage it before it becomes chaotic. \\
% A strapping tree is grown from a tiny sprout; \\
% A sky-scraping tower is built from a modest mound, \\
% A far-reaching journey begins with a small step. \\
% Those who act upon will fail, \\
% Those who hold on will lose. \\
% The master acts not, therefore fails not; \\
% Holds not on, therefore loses not. \\
% Amateurs often fail at the verge of success. \\
% Be focused in the end as in the beginning, \\
% Then there will be no failure. \\
% Hence the master desires not to be desirous, \\
% Treasures not precious possessions. \\
% Learn to be unlearned, \\
% Liberate the people of their past. \\
% Assist the myriad things in returning to their essence, \\
% And not dare act. \\
% \flushright{-- Laozi}

% \end{paracol}

% \newpage



%
% This section is written in Markdown
%

%\tableofcontents

\newpage
\begin{markdown}

# Why \LaTeX?

\LaTeX\ is far from a perfect writing tool. For one thing, it's a *typesetting environment* and not a word processor, so little things like the difference between a \` and a ' can end up being a distraction. That is, when you're writing a paper, you generally want to focus on your words, and not getting the syntax right. So why use \LaTeX?

1. Formatting is done `for free' by the SIG CHI template, and it does a great job of this for you, so you can spend your time thinking about *what* you say, not how it should look.
2. The University of Waterloo provides a Pro Overleaf account if you login with your UW userid. Overleaf is lightweight, easy-to-use, and supports collaborative writing really well. 
3. \LaTeX\ integrates well with other (free) tools like R and Python, and any referencing software you can find through the Bib\TeX package.

However, as hinted at above, \LaTeX\ also isn't perfect, and so this project embodies some examples, tips, and hacks that make our typical publication workflow a little easier. For instance, I've included some supports for all of the following tasks:

1. Writing: It supports Markdown: enabling you to focus on *content*, and the ACM templates to handle all formatting

2. Collaborative Editing: Sections are separated out into files, so that authors can edit each one without interrupting one another on Overleaf, without the need to email drafts back and forth between authors, and with features like version control, and automatic backups. 

3. Illustrating: Several commonly-used \LaTeX\ packages are automatically included, for tasks like creating figures and tables, documenting code, and referencing

4. Revising: The \texttt{changes} package helps authors to show revisions to their work when working within the ACM's Revise \& Resubmit model

5. Many (optional) in-text shortcuts are also included, for tasks like reporting statistics, defining colour palettes, and including alternative languages. They are there if you need them, and easy to ignore if you don't. 



## Markdown
A common complaint about \LaTeX\ is that it's hard to read and write, and that the learning curve is *really* steep. For this reason, we're following the general philosophy of letting \LaTeX\ handle the stuff it's good at --- formatting, math, figures, references --- and writing the rest in Markdown. Markdown is extremely easy to read and write. In fact, you're probably familiar with much of its syntax already: 

>    "The overriding design goal for Markdown’s formatting syntax is to make it as readable as possible. The idea is that a Markdown-formatted document should be publishable as-is, as plain text, without looking like it’s been marked up with tags or formatting instructions. While Markdown’s syntax has been influenced by several existing text-to-HTML filters, the single biggest source of inspiration for Markdown’s syntax is the format of plain text email"
> --- John Gruber \citep{gruber2006}.

However, you should be aware of some its limitations --- especially for some of the more technical aspects of a scientific paper. It doesn't always play well with tables, and while you can include images, Markdown doesn't necessarily offer some of the nice features that \LaTeX\ does. Keep reading for more information on that. 

**This template has been configured to allow *both* Markdown and \LaTeX\ syntax in the same file**, through the \texttt{markdown} package's \texttt{hybrid} option. You can even combine them in the same *sentence*, and for example write paragraphs in Markdown that include BibTeX citations.



## PGFPlots and My Templates 
As as I was completing my PhD, I decided to spend some time trying to learn to make better figures since this was something I expected to spend a lot of time doing over the rest of my career. Some of the things I'd hoped to improve on were using scalable graphics, having better control over the width/height of figures, and being able to use the data itself to create the images (better fit with a typical academic workflow). I also prefer a minimalist feel, and wanted my papers to reflect that (And even if you prefer something different, this is a good starting point, you can always make things more complex where needed). 

So after doing some research, I found that \texttt{PGFPlots} was a good fit for my needs. But it wasn't perfect, and in particular can be complicated to get right. So in the interest of saving the time of re-inventing the wheel, and (hopefully) enabling others to use my work for their own papers, I decided to create a set of templates that would encapsulate the common types of graphs I use in HCI research. I called these templates \texttt{IDEA Plots} after my lab at the time. As I developed the templates, they grew to include style choices, colour palettes, and some common shortcuts for layout. I expect they'll continue to evolve for a while to come.  

All of the examples in this document use those templates, which are now wrapped into CHI Zen, with some in-place customization based on the needs of the graph. 



## Tracking Changes

Since SIG CHI has begun transitioning its conferences to the *Proceedings of the ACM* model, most involve some form of Revise \& Resubmit. That is, you should expect to take reviewer feedback into consideration and submit an updated version of your paper for a second round of review. When doing so, conferences typically ask for a marked-up version of the paper that shows where changes have been made.  To track changes in \LaTeX, we use the \texttt{changes} package. It provides a few simple functions to track anything that you've \texttt{added()}, \texttt{deleted()}, or \texttt{replaced} in the paper. 



## Accessibility
A perennial to-do list item, and a terrible gap in academic writing. I'm currently exploring the \texttt{accessibility} package to provide some support, but it is not (yet) working with this template. We clearly need to spend more time on this over the next writing cycles.

Unfortunately, the \href{https://chi2021.acm.org/for-authors/presenting/papers/guide-to-an-accessible-submission}{CHI 2021 Guide to an Accessible Submission} does not yet provide any additional resources for writing in \LaTeX. 


\end{markdown}

%%%%%%%%%%%%%%%%%%%%%%%%%%%%%%%%%%%%%%%%%%%%%%%%%%%%%%%%%%%%%%%%%%%%%%%%%%%%%
\section{\LaTeX\ Packages and Tools}
%%%%%%%%%%%%%%%%%%%%%%%%%%%%%%%%%%%%%%%%%%%%%%%%%%%%%%%%%%%%%%%%%%%%%%%%%%%%%

- Lots of different options here
- Many come down to preference

- have included quite a few here, as needs evolved over years
- some basic documentation



%
% This Section is written in LaTeX because Markdown wasn't playing well with the tables
%
\newpage
%%%%%%%%%%%%%%%%%%%%%%%%%%%%%%%%%%%%%%%%%%%%%%%%%%%%%%
\section{Tables}
\label{sec:tables}
%%%%%%%%%%%%%%%%%%%%%%%%%%%%%%%%%%%%%%%%%%%%%%%%%%%%%%

As a rule of thumb, if you can show your data with a table, consider doing that first. Tables are quite effective at `showing your data', and are usually easy to interpret. The Vis literature even suggests that tables might be more convincing for a skeptical audience \citep{pandey2014persuasive}!

In terms of table formatting, many of the same principles apply to tables as figures. My advice is to: 

\begin{enumerate}
    \item Maintain a high Data-Ink Ratio \citep{tufte1983visual}: 
        \begin{itemize}
            \item Don't use vertical lines
            \item Use horizontal lines sparingly
            \item Use whitespace to help group things together that belong together
        \end{itemize}
    \item Use \texttt{\textbackslash toprule}, \texttt{\textbackslash midrule}, and \texttt{\textbackslash bottomrule} to frame your table
    \item Use the \texttt{[tb!]} flags when declaring the table in \LaTeX to push it to the top or bottom of a page
    \item Use the \texttt{label\{\}} command to give your table a label that you can use later with \texttt{autoref\{\}}
\end{enumerate}
%
In general, writing tables directly in \LaTeX is a huge pain --- the output is generally \emph{beautiful}, but the input is complicated, and generally not very human-readable. I suggest using an online table generator, like \href{https://www.tablesgenerator.com}{https://www.tablesgenerator.com}, and then fine-tuning the formatting in your document. \autoref{tab:exampletable} is an example from \citet{wallace2013}. 

\begin{center}
\begin{table*}[b!]
\small
\begin{center}
\begin{tabular}{l l l l l}
\toprule
Condition & \multicolumn{4}{l}{Teamwork Measures}\\
\cmidrule(r){2-5}
& \multicolumn{2}{l}{Measure 1} & Measure 2 & Measure 3 \\
\cmidrule(r){2-3} 
&  Sub 1 & Sub 2 &  \\ 


\midrule


Condition 1 & .478 & .468 & 1.68 & 12.3  \\
                  & (.0259) & (.031) & (.703) & (5.93) \\
\\

Condition 2 & .444 & .451 & 1.32 & 18.4 \\
                              & (.0478) & (.0585) & (.319) & (2.76) \\
\\

Condition 3 &  & .410 &  2.857 &  \\
                     &        & (.0384)  & (.350) & \\
\\


ANOVA Results & $F_{(1,12)} = 2.27,$ & $F_{(2,18)} = 2.76,$  & $F_{(2,18)} = 16.16,$ & $F_{(1,12)} = 6.16,$ \\
                          & $p = .158$ & $p = .090  $ &  $p < .0001$* &  $p = 0.029$ * \\
\bottomrule
\end{tabular}

\caption[a]{Mean values and standard deviations (in parentheses) for teamwork measures, and ANOVA results for comparisons between experimental conditions. Significant results denoted by *.}
\label{tab:exampletable}
\end{center}
\end{table*}
\end{center}

\subsection*{Additional Resources}
\begin{enumerate}
    \item Overleaf's documentation for tables: \href{https://www.overleaf.com/learn/latex/Tables}{https://www.overleaf.com/learn/latex/Tables}
\end{enumerate}

\newpage

%
% This section written in LaTeX because a lot of the graphing is untested in Markdown
%


%%%%%%%%%%%%%%%%%%%%%%%%%%%%%%%%%%%%%%%%%%%%%%%%%%%%%%%%%%%%%%%%%%%%%%%%%%%%%
\section{Figures}
%%%%%%%%%%%%%%%%%%%%%%%%%%%%%%%%%%%%%%%%%%%%%%%%%%%%%%%%%%%%%%%%%%%%%%%%%%%%%

\begin{markdown}
Figures and Tables provide your readers with a `visual abstract' of your work. They can make the (often dense) results section more accessible to readers who are unfamiliar with your research area, and are a fallback when something in the text doesn't make sense right away. That is, they're often the first impression a reader has of your paper, make your work more accessible to the general audience, and provide a second, more visual, way of explaining results to your readers.

A general principal that I try to embody in all of these graphs is to **show your data**. Too often, we provide only summary data like an average and a standard error value, when we could have shown the entire data set. Summary statistics are useful, but often hide important trends in the data. (We should take this to heart well before writing the paper up, this all holds true for *data analysis*). I think that in the past, tools weren't always up to the task of showing our data. Tools like Excel would help you easily produce a bar chart, but don't have a preset for box plots. Fortunately, today our tools have become much better. 

I've also tried to take best practice into consideration where possible, but there's always room for improvement. \citet{munznervisualization} provides an excellent summary of best practice from the literature, and I encourage you to read Chapter 6 on Rules of Thumb in depth to start thinking about how to approach thinking about visualizations as you're working on them. In particular, tips like **get it right in black and white** and **eyes beat memory** will help to improve the readability *and* accessibility of your paper. 

The following sections are framed in terms of problem/solution for common types of graphs that we see in HCI research, with examples for each. The style for each graph is informed and inspired by \citet{tufte1983visual}, who emphasizes minimalist presentation of quantitative data. I find that Tufte's principles of information visualization, and general approach to working on graphs, are largely consistent with Munzner's advice above. That said, the style cetainly isn't for everyone --- your mileage may vary, please use what you find useful.  


 \end{markdown}
 
%  For example, some good examples:
% Bartneck and Hu \cite{Bartneck:2009:SAC:1518701.1518810}


% Senellart \cite{Senellart:2013:DYR:2513166.2514938} discusses some useful cases of `bad' style, which superficially may appear to be picky but offer important guidance towards submitting polished papers. 
 
 
% Munzer \cite{Munzner:2008:PPW:1422919.1422927} 
% - some good advice in general, but more focused on info vis research

% - text is large and legible
% - all axes are labelled, including units of measure
% - axes start at 0, unless explicitly justified (for a good reason!)






\newpage
%%%%%%%%%%%%%%%%%%%%%%%%%%%%%%%%%%%%%%%%%%%%%%%%%%%%%%%%%%
\subsection{Comparing Performance: Bar Charts}
%%%%%%%%%%%%%%%%%%%%%%%%%%%%%%%%%%%%%%%%%%%%%%%%%%%%%%%%%%

- common to compare performance, efficiency, error rate, or other dependent measures between two study groups. For example, our new technique for 3D pointing, compared to a computer mouse. 
- most simple way of doing so is to show the means and standard error in a bar graph, and this is commonly accepted throughout the ACM. 
- For example, Figure \ref{bargraph} is an example drawn from \citet{pietroszek2015}.





\begin{figure}[b!]
\begin{tikzpicture}
\begin{axis}[IDEA bar,
        symbolic x coords={non-occluded,occluded},
        xtick = {non-occluded,occluded},
        enlarge x limits=0.5,
	    ymin = 0, ymax = 10,
        ylabel = {Selection Time (s)},
        width=\columnwidth,
        height = 5cm,
]

%Depth Ray
\addplot[style={draw=none,fill=AHSLight},error bars/.cd, y dir = both, y explicit, error bar style={black}]
coordinates {(occluded, 6.8)  +- (1.7,1.7)};
\addlegendentry{Depth Cursor}

%Tiltcasting
\addplot+[style={draw=none, fill=SchoolRedLight,line width = 0pt},error bars/.cd, y dir = both, y explicit, error bar style={black}]
coordinates {(non-occluded,3.24) +- (.712,.712)
		    (occluded,4)  +- (.922,.922)};
\addlegendentry{Tiltcasting}

%Smartcasting
\addplot+[style={draw=none, fill=EnvironmentLight},error bars/.cd, y dir = both, y explicit, error bar style={black}]
coordinates {(non-occluded, 2.10)  +- (.42,.42)};
\addlegendentry{Smartcasting}

\end{axis}
\end{tikzpicture}
\caption{An example bar graph with error bars. Data is specified in the latex file.}
\label{bargraph}
\end{figure}






\newpage
%%%%%%%%%%%%%%%%%%%%%%%%%%%%%%%%%%%%%%%%%%%%%%%%%%%%%%%%%%
\subsection{Comparing Performance: Box Plots, Violin Plots, Bee Swarm Plots}
%%%%%%%%%%%%%%%%%%%%%%%%%%%%%%%%%%%%%%%%%%%%%%%%%%%%%%%%%%


- to be more accurate and descriptive, box plots are often used
- similar to a bar graph with error bars, but also indicate quartiles and outliers 
- typically, better practice to include a box plot when possible -- but can be substantially more difficult to create in Microsoft Word. 

variations: 
- Admittedly, "Bee Swarm" plots are my personal preference for this kind of data since they "show your data" in a very clear way... but I haven't yet found a  way to generate them in \LaTeX. If you happen to work in R, there are some excellent supports for generating these graphs, and you can import those directly into this template with a little work. 
- Violin plots 




\newpage
%%%%%%%%%%%%%%%%%%%%%%%%%%%%%%%%%%%%%%%%%%%%%%%%%%%%%%%%%%
\subsection{Likert Scale Data: Stacked Divergent Bar Graphs}
%%%%%%%%%%%%%%%%%%%%%%%%%%%%%%%%%%%%%%%%%%%%%%%%%%%%%%%%%%

A \emph{lot} of HCI research involves questionnaires, and interpreting diferences in questionnaire respones. For instance, Likert scale questions are very common. My preference for visualizing Likert scale responses is a stacked divergent bar graph (e.g., \autoref{fig:avatar_identification}), where responses are stacked into a bar, with neutral responses lined up at the graph's origin. This type of graph has a few properties that make it useful: in particular, readers can get a quick sense of `how positive' or `how negative' responses were based on the bar's position, while still being able to look at how many responses fell into each category. 

\begin{figure}[b!]
\begin{tikzpicture}
\pgfplotstableread[col sep = comma]{data/Cleric_AvatarIdentificationStacked.csv}\ClericIdentificationData
\pgfplotstableread[col sep = comma]{data/Monster_AvatarIdentificationStacked.csv}\MonsterIdentificationData

\begin{axis}[
    IDEA Likert,
    height = 6.5cm,
    width = .75\columnwidth,
    % Y AXIS
    symbolic y coords = {Ava_Conn,	Ava_WE,	AvavsPhys,	AvaVsPers,	AvaVsPhysIdeal,	AvaVsPersIdeal},
    ytick={Ava_Conn, Ava_WE, AvavsPhys, AvaVsPers, AvaVsPhysIdeal, AvaVsPersIdeal},
    yticklabels={Connectedness, Refer to \\ Avatar as ``We'', Physical \\ Similarity, Personality\\ Similarity, Physical Ideal, Ideal Personality},
    yticklabel style={align=right,font={\sffamily\small}},
    y axis line style={draw=none},
    y dir = reverse,
    ymajorgrids = false,
    % X AXIS
    xmin=-100, xmax=100,
    xlabel={\% of Responses},
    xlabel style = {font=\small\sffamily},
    xticklabel style = {font=\small\sffamily},
    xtick = {-100,-50,0,50,100},
    xticklabels = {100, 50, 0, 50, 100},
    extra x ticks = {-75, 75},
    extra x tick labels = {$\leftarrow$ Disagree, Agree $\rightarrow$},
    every extra x tick/.style={major tick length=0,yshift={-8pt}},
    ]
    \addplot[draw=none,fill=none, forget plot] coordinates {(0,Ava_Conn)(0,Ava_WE)(0,AvavsPhys)(0,AvaVsPers)(0,AvaVsPhysIdeal)(0,AvaVsPersIdeal)};
    \addplot[draw=none,fill=Cleric_3, forget plot, bar shift=.1cm] table [x expr={\thisrow{4}*-0.5},y=Measure] {\ClericIdentificationData};
    \addplot[draw=none,fill=Cleric_2, forget plot, bar shift=.1cm] table [x expr={\thisrow{3}*-1},y=Measure] {\ClericIdentificationData};
    \addplot[draw=none,fill=Cleric_1, forget plot, bar shift=.1cm] table [x expr={\thisrow{2}*-1}, y=Measure ] {\ClericIdentificationData};
    \addplot[draw=none,fill=Cleric_0, forget plot, bar shift=.1cm] table [x expr={\thisrow{1}*-1}, y=Measure ] {\ClericIdentificationData};
    \resetstackedplots
    \addplot[draw=none,fill=none, forget plot] coordinates {(0,Ava_Conn)(0,Ava_WE)(0,AvavsPhys)(0,AvaVsPers)(0,AvaVsPhysIdeal)(0,AvaVsPersIdeal)};
    \addplot[draw=none,fill=Monster_3, forget plot, bar shift=-.1cm] table [x expr={\thisrow{4}*-0.5},y=Measure] {\MonsterIdentificationData};
    \addplot[draw=none,fill=Monster_2, forget plot, bar shift=-.1cm] table [x expr={\thisrow{3}*-1},y=Measure] {\MonsterIdentificationData};
    \addplot[draw=none,fill=Monster_1, forget plot, bar shift=-.1cm] table [x expr={\thisrow{2}*-1}, y=Measure ] {\MonsterIdentificationData};
    \addplot[draw=none,fill=Monster_0, forget plot, bar shift=-.1cm] table [x expr={\thisrow{1}*-1}, y=Measure ] {\MonsterIdentificationData};
\end{axis}

\begin{axis}[
    IDEA Likert,
    height = 6.5cm,
    width = .75\columnwidth,
    % Y AXIS
    symbolic y coords = {Ava_Conn,	Ava_WE,	AvavsPhys,	AvaVsPers,	AvaVsPhysIdeal,	AvaVsPersIdeal},
    ytick={Ava_Conn, Ava_WE, AvavsPhys, AvaVsPers, AvaVsPhysIdeal, AvaVsPersIdeal},
    yticklabels={Connectedness, Refer to \\ Avatar as ``We'', Physical \\ Similarity, Personality\\ Similarity, Physical Ideal, Ideal Personality},
    yticklabel style={align=right,font={\sffamily\small}},
    y axis line style={draw=none},
    y dir = reverse,
    ymajorgrids = false,
    % X AXIS
    xmin=-100, xmax=100,
    xlabel={\% of Responses},
    xlabel style = {font=\small\sffamily},
    xticklabel style = {font=\small\sffamily},
    xtick = {-100,-50,0,50,100},
    xticklabels = {100, 50, 0, 50, 100},
    extra x ticks = {-75, 75},
    extra x tick labels = {$\leftarrow$ Disagree, Agree $\rightarrow$ },
    every extra x tick/.style={major tick length=0,yshift={-8pt}},
    ]
    \addplot[draw=none,fill=none, forget plot] coordinates {(0,Ava_Conn)(0,Ava_WE)(0,AvavsPhys)(0,AvaVsPers)(0,AvaVsPhysIdeal)(0,AvaVsPersIdeal)};
    
    \addplot[draw=none,fill=Cleric_3, forget plot, bar shift=.1cm] table [x expr={\thisrow{4}*0.5}, y=Measure ] {\ClericIdentificationData};
    \addplot[draw=none,fill=Cleric_4, forget plot, bar shift=.1cm] table [x expr={\thisrow{5}}, y=Measure ] {\ClericIdentificationData};
    \addplot[draw=none,fill=Cleric_5, forget plot, bar shift=.1cm] table [x expr={\thisrow{6}}, y=Measure ] {\ClericIdentificationData};
    \addplot[draw=none,fill=Cleric_6, forget plot, bar shift=.1cm] table [x expr={\thisrow{7}}, y=Measure ] {\ClericIdentificationData}; \label{plot:cleric6}
    \resetstackedplots
    \addplot[draw=none,fill=none, forget plot] coordinates {(0,Ava_Conn)(0,Ava_WE)(0,AvavsPhys)(0,AvaVsPers)(0,AvaVsPhysIdeal)(0,AvaVsPersIdeal)};
    \addplot[draw=none,fill=Monster_3, forget plot, bar shift=-.1cm] table [x expr={\thisrow{4}*0.5}, y=Measure ] {\MonsterIdentificationData};
    \addplot[draw=none,fill=Monster_4, forget plot, bar shift=-.1cm] table [x expr={\thisrow{5}}, y=Measure ] {\MonsterIdentificationData};
    \addplot[draw=none,fill=Monster_5, forget plot, bar shift=-.1cm] table [x expr={\thisrow{6}}, y=Measure ] {\MonsterIdentificationData};
    \addplot[draw=none,fill=Monster_6, forget plot, bar shift=-.1cm] table [x expr={\thisrow{7}}, y=Measure ] {\MonsterIdentificationData};
    %\addlegendimage{only marks, mark=o}
    %\addlegendimage{only marks, mark=o}
    %\legend{Neutral, Agree, Strongly Agree, Disagree,Strongly Disagree}
    after end axis/.code={
        \node at (axis cs:80,Ava_Conn) [anchor=east, ,font=\tiny] {\AsteriskBold};
        \node at (axis cs:80,Ava_WE) [anchor=east, font=\tiny] {\AsteriskBold};
        %\draw [white] (axis cs:0,AvaVsPersIdeal) -- (axis cs:0,Ava_Conn);
        %\draw [white] (axis cs:-50,AvaVsPersIdeal) -- (axis cs:-50,Ava_Conn);
        %\draw [white] (axis cs:50,AvaVsPersIdeal) -- (axis cs:50,Ava_Conn);
    }
    \coordinate (legend) at (axis description cs:1.15,.85);
\end{axis}

% this is a dummy `axis' environment only to create the legend
\matrix [
    matrix of nodes, 
    every node/.style={anchor=center}, 
    ] at (legend) {
        |[fill=Cleric_0]| & |[fill=Cleric_1]| & |[fill=Cleric_2]| & |[fill=Cleric_3]| & |[fill=Cleric_4]| & |[fill=Cleric_5]| & |[fill=Cleric_6]| & |[font=\small\sffamily]|Cleric \\
        |[fill=Monster_0]| & |[fill=Monster_1]| & |[fill=Monster_2]| & |[fill=Monster_3]| & |[fill=Monster_4]| & |[fill=Monster_5]| & |[fill=Monster_6]| & |[font=\small\sffamily]|Monster \\
    };
\end{tikzpicture}
    
\caption{Summary of avatar identification responses, based on 7-point Likert scales. Mann Whitney U tests revealed significant differences between \textsc{Avatar} groups for `Connectness' and `Refer to Avatar as We'. }
\label{fig:avatar_identification}
\end{figure}





\newpage
%%%%%%%%%%%%%%%%%%%%%%%%%%%%%%%%%%%%%%%%%%%%%%%%%%%%%%%%%%
\subsection{Trends over Time: Line Graph}
%%%%%%%%%%%%%%%%%%%%%%%%%%%%%%%%%%%%%%%%%%%%%%%%%%%%%%%%%%



\begin{figure}[b!]
\begin{tikzpicture}[baseline]
\begin{axis}[IDEA line,
	% Graph Options
	width=.475*\columnwidth,
	height=4cm,
	title=\texttt{r/stop\-drinking},
	% X Axis Options
	%date coordinates in=x,
	%date ZERO=2014-01-01,
    %xticklabel=\year,
    %xmin=2014-01-01,
    %xmax=2018-01-01,
    xlabel=Year,
	xtick={1,13,25,37,49},                              %% Jim: apologies... this is a hack
    xticklabels={2014,2015,2016,2017,2018},             %%      will figure out what's going on here if we have time later
	%xtick ={1,2,3,4,5,6,7,8,9},
	% Y Axis Options
	ylabel = Distinct Users / Active Threads,
	ylabel style = {font=\tiny},
	ymin = 0, ymax = 6000,
	ytick = {0,2000,4000,6000},
	ytick style = {font=\tiny},
    %legend image code/.code={\draw[#1] circle (0.1cm);}
]

\pgfplotstableread[col sep = comma]{data/stopdrinking-usagebymonth.csv}\mydata;
\addplot+[draw=ScienceDark,fill=none,thick] table[x=index,y=users] {\mydata};
\addplot+[draw=ArtsLight,fill=none,thick] table[x=index,y=threads] {\mydata};

\end{axis}
\end{tikzpicture}%
%
\begin{tikzpicture}[baseline]
\begin{axis}[IDEA line,
	% Graph Options
	width=.475*\columnwidth,	
	height=4cm,
	title=\texttt{r/Opiates\-Recovery},
	% X Axis Options
	xtick={1,13,25,37,49},                              %% Jim: apologies... this is a hack
    xticklabels={2014,2015,2016,2017,2018},             %%      will figure out what's going on here if we have time later
	%date coordinates in=x,
	%date ZERO=2014-01-01,
    %xticklabel=\year,
    %xmin=2014-01-01,
    %xmax=2018-01-01,
	xlabel = Year,
	%xtick ={1,2,3,4,5,6,7,8,9},
	% Y Axis Options
	ylabel = Distinct Users / Active Threads,
	ylabel style = {font=\tiny},
	ymin = 0, ymax = 800,
	ytick = {0,200,400,600,800},
	ytick style = {font=\tiny},
	legend style={at={(0.5,-1)},anchor=south}
    %legend image code/.code={\draw[#1] circle (0.1cm);}
]

\pgfplotstableread[col sep = comma]{data/opiaterecovery-usagebymonth.csv}\mydata;
\addplot+[draw=ScienceDark,fill=none,thick] table[x=index,y=users] {\mydata};
\addlegendentry{Distinct Users}
\addplot+[draw=ArtsLight,fill=none,thick] table[x=index,y=threads] {\mydata};
\addlegendentry{Active Threads}

\end{axis}
\end{tikzpicture}
%\setlength{\belowcaptionskip}{-20pt}
\caption{Distinct users and active threads for \texttt{r/stop\-drinking} (top) and \texttt{r/Opiates\-Recovery} (bottom) showing that both counts are experiencing an upwards trend over time for both subreddits.} 

\label{fig:activity}
\end{figure}



\begin{figure*}[tb!]
\begin{center}
\begin{tikzpicture}
\begin{groupplot}[
   group style={
       group size=1 by 5,
       x descriptions at=edge bottom,
       y descriptions at=edge left,
       vertical sep=1pt,
       horizontal sep=0pt},
  IDEA tufte panel, 
	% X Axis
	width=\columnwidth,
	xmin = 1990, xmax = 2015,
	xtick = {1990,1994,1998,2002,2006,2010,2015},
	x tick label style={font=\small,rotate=90, /pgf/number format/1000 sep=},
	axis line style={thick},
	% Y Axis
	ytick={10,20,30,40,50,60,70,80,90},
        ymin = 0,
        ymax = 50,
      ]


% Bibliographic
\nextgroupplot[bar width=7pt, height = 2.25cm, ymax = 25, title={Bibliographic}]
\addplot+[draw=none, fill=black!50] 
 coordinates {(1990,20) (1992,4) (1994,0) (1996,0) (1998,2) (2000,0) (2002,0) (2004,2) (2006,2) (2008,1) (2010,0) (2011,0) (2012,0) (2013,1) (2014,0) (2015,1) };       
  
% Not Empirical      
\nextgroupplot[bar width=7pt, height = 3.5cm, ymax=70, title={Not Empirical}]
\addplot+[draw=none, fill=black!50] 
coordinates {(1990,37) (1992,55) (1994,33) (1996,38) (1998,54) (2000,35) (2002,20) (2004,18) (2006,24) (2008,13) (2010,5) (2011,31) (2012,9) (2013,4) (2014,6) (2015,7) };

% Explanatory
\nextgroupplot[bar width=7pt, height = 3cm, title={Explanatory}]
\addplot+[draw=none, fill=black!50]  
coordinates {(1990,3) (1992,4) (1994,5) (1996,0) (1998,0) (2000,5) (2002,15) (2004,16) (2006,29) (2008,25) (2010,21) (2011,16) (2012,8) (2013,14) (2014,15) (2015,12) };

% D & E
\nextgroupplot[bar width=7pt, ymax=60, height = 3.5cm, title={Design and Evaluation}]
\addplot+[draw=none, fill=blue!50] 
coordinates {(1990,17) (1992,13) (1994,26) (1996,22) (1998,15) (2000,27) (2002,39) (2004,35) (2006,23) (2008,19) (2010,21) (2011,10) (2012,28) (2013,26) (2014,24) (2015,22) };

% Descriptive
\nextgroupplot[bar width=7pt, ymax = 70, height = 3.5cm, title={Descriptive}]
\addplot+[draw=none,fill=black!50]
coordinates {(1990,23) (1992,23) (1994,36) (1996, 40) (1998,29) (2000,32) (2002,27) (2004,29) (2006,23) (2008,42) (2010,53) (2011,42) (2012,55) (2013,55) (2014,55) (2015,58) };


\end{groupplot}
\end{tikzpicture}
\caption{An example of Tufte's stacked panel graph, which presents the same information as a stacked bar graph but better enables comparisons across each vertical dimension. Best practice is to use black and white when possible, with sparing colour for emphasis where appropriate. }
\label{stackedpanel}
\end{center}
\end{figure*}




% \begin{figure}[tb!]
% \begin{tikzpicture}
% \begin{groupplot}[
%   group style={
%       group size=1 by 5,
%       x descriptions at=edge bottom,
%       y descriptions at=edge left,
%       vertical sep=10pt,
%       horizontal sep=0pt},
%   IDEA line,
% 	% Graph Options
% 	height=3cm,
% 	 separate axis lines,
% 	% X Axis Options
% 	xmin=0, xmax= 35,
% 	xtick={1,5,10,15,20,25,30, 35},
% 	%date coordinates in=x,
% 	%date ZERO=2014-01-01,
%     %xticklabel=\year,
%     %xmin=2014-01-01,
%     %xmax=2018-01-01,
% 	xlabel = Number of NPC Requests,
% 	xlabel style = {font=\small\sffamily},
%     xticklabel style = {font=\small\sffamily},
% 	%xtick ={1,2,3,4,5,6,7,8,9},
% 	% Y Axis Options
% 	%ylabel = Proportion of Responses,
% 	ymajorgrids = false,
% 	ylabel shift = -.3cm,
% 	ymin = 0, ymax = 1,
% 	ytick = {0, 1.00},
% 	ylabel style = {font=\small\sffamily},
%     yticklabel style = {font=\small\sffamily},
% 	%ytick style = {font=\tiny},
% 	title style = {yshift=-.75cm,font=\sffamily\small}
% ]
% \pgfplotstableread[col sep = comma]{data/CBT_PropOverTime.csv}\cbtovertimedata;

% \nextgroupplot[xtick style = {white}, title style = {yshift=+.35cm,font=\sffamily\small}, title={Not Damaging}]
% \addplot+[mark=*, mark size=1.5pt, draw=gray, only marks,fill=gray,thin] table[x=Query,y=NotDamaging] {\cbtovertimedata};
% \addplot [solid, draw=black, fill=none,ultra thick] table[y={create col/linear regression={y=NotDamaging}}] {\cbtovertimedata};
% %\addlegendentry{Not Damaging}

% \nextgroupplot[xtick style = {white}, title={Empathize}]
% \addplot+[mark=*, mark size=1.5pt, draw=gray, only marks,fill=gray,thin] table[x=Query,y=Empathize] {\cbtovertimedata};
% \addplot [solid, draw=black, fill=none,ultra thick] table[y={create col/linear regression={y=Empathize}}] {\cbtovertimedata};
% %\addlegendentry{Empathize}

% \nextgroupplot[xtick style = {white}, title style = {yshift=+.25cm,font=\sffamily\small}, title={Reframe}]
% \addplot+[mark=*, mark size=1.5pt, draw=gray, only marks,fill=gray,thin] table[x=Query,y=Reframe] {\cbtovertimedata};
% \addplot [solid, draw=black, fill=none,ultra thick] table[y={create col/linear regression={y=Reframe}}] {\cbtovertimedata};
% %\addlegendentry{Reframe}

% \nextgroupplot[xtick style = {white}, title={Encourage}]
% \addplot+[mark=*, mark size=1.5pt, draw=gray, only marks,fill=gray,thin] table[x=Query,y=Encourage] {\cbtovertimedata};
% \addplot [solid, draw=black, fill=none,ultra thick] table[y={create col/linear regression={y=Encourage}}] {\cbtovertimedata};
% %\addlegendentry{Encourage}

% \nextgroupplot[title={Solution}]
% \addplot+[mark=*, mark size=1.5pt, draw=gray, only marks,fill=gray,thin] table[x=Query,y=Solution] {\cbtovertimedata};
% \addplot [solid, draw=black, fill=none,ultra thick] table[y={create col/linear regression={y=Solution}}] {\cbtovertimedata};
% %\addlegendentry{Solution}


% %\addplot+[mark=x, only marks,draw=Monster_0,fill=none,semithick] table[x=Query,y=Proportion, y error plus=CI-upper, y error minus=CI-lower] {\barbariansolutiondata};
% %\addlegendentry{Monster}
% %\addplot [draw=Monster_0, fill=none,ultra thick] table[y={create col/linear regression={y=Proportion}}] {\barbariansolutiondata};

% \end{groupplot}
% \end{tikzpicture}

% %\setlength{\belowcaptionskip}{-20pt}
% \caption{The proportion of responses that were classified as meeting five criteria: Empathize, Reframe, Encourage, Solution, and Not Damaging. The three categories related to CBT, Empathize, Reframe, and Encourage, were found to decrease over time ($p=0.05$). However, we did not find these changes for Solution and Not Damaging. 
% %decreased over time for all participants, however responses by participants in the \textsc{Cleric} group decreased more quickly than those in the \textsc{Monster} group.
% } 

% \label{fig:SolutionsOverQueryNumber}
% \end{figure}



\newpage
%%%%%%%%%%%%%%%%%%%%%%%%%%%%%%%%%%%%%%%%%%%%%%%%%%%%%%%%%%
\subsection{Illustrating Scenarios: Rotograph}
%%%%%%%%%%%%%%%%%%%%%%%%%%%%%%%%%%%%%%%%%%%%%%%%%%%%%%%%%%

- You don't need to be an artist
- take a picture, trace over it
- often by far the most effective technique for simply demonstrating key concepts in your paper 


\newpage
%%%%%%%%%%%%%%%%%%%%%%%%%%%%%%%%%%%%%%%%%%%%%%%%%%%%%%%%%%
\subsection{Scheduling: Gantt Charts}
%%%%%%%%%%%%%%%%%%%%%%%%%%%%%%%%%%%%%%%%%%%%%%%%%%%%%%%%%%

\begin{figure}[b]
\begin{ganttchart}[
    x unit=.208cm,
    y unit title=0.4cm,
    y unit chart=0.3cm,
    vgrid,
    time slot format=isodate-yearmonth,
    time slot unit=month,
    vgrid={*{3}{draw=none},dotted,*{3}{draw=none},dotted,*{3}{draw=none},*{1}{draw=gray}},
    %
    % TITLE OPTIONS
    %
    title/.style={draw=none, fill=none},
    title label node/.append style={below=-1.6ex},
    title top shift=-.5,
    title label font=\tiny,
    %
    % BAR OPTIONS
    %
    bar height=.5,
    bar/.append style={draw = none, fill=blue!50},
    bar label font=\tiny,
    %   
    % GROUP OPTIONS
    %
    group label font={\tiny},
    group right shift=0,
    group top shift=.6,
    group height=.3,
    group peaks height=.2,
   ]{2018-05}{2023-04}

    
    \gantttitle{\textbf{Year 1: 2019 -- 2020}}{12}
    \gantttitle{\textbf{Year 2: 2020 -- 2021}}{12}
    \gantttitle{\textbf{Year 3: 2021 -- 2022}}{12}
    \gantttitle{\textbf{Year 4: 2022 -- 2023}}{12}
    \gantttitle{\textbf{Year 5: 2023 -- 2024}}{12}
                
\ganttset{progress label text={}, link/.style={black, -to}}

\ganttgroup{\textbf{Objective 1: Design Framework}}{2018-05}{2021-04} \\
\ganttbar{Literature Review (1.1)}{2018-05}{2019-03} \\
\ganttbar{Artifact Review (1.2)}{2018-09}{2019-07} \\
\ganttbar{Framework Development (1.3)}{2019-05}{2020-04} \\
\ganttbar{Mentoring \& Publication (1.4)}{2020-05}{2021-04} \\

\ganttgroup{\textbf{Objective 2: Interaction Mechanics}}{2020-05}{2022-04} \\
\ganttbar{Design (2.1)}{2020-05}{2020-07} \\
\ganttbar{Development (2.2)}{2020-08}{2020-12} \\
\ganttbar{Lab Study (2.3)}{2021-01}{2021-07} \\
\ganttbar{Analysis \& Publication (2.4)}{2021-07}{2022-04} \\

\ganttgroup{\textbf{Objective 3: In-Situ Evaluation}}{2021-05}{2023-04} \\
\ganttbar{Requirements Gathering (3.1)}{2021-05}{2021-10} \\
\ganttbar{Development (3.2)}{2021-11}{2022-06} \\
\ganttbar{Ecological Validation (3.3)}{2022-06}{2022-12} \\
\ganttbar{Analysis \& Publication (3.4)}{2023-01}{2023-04} 


\end{ganttchart}
\caption{Research activities are designed to scaffold development of a design framework and evaluation methodology, culminating in an in-situ evaluation with Homewood Research Institute.}
\label{gantt}
\end{figure}


\input{manual/06-discussion.tex}

\input{manual/07-limitations.tex}

\input{manual/08-conclusion.tex}


%%%%%%%%%%%%%%%%%%%%%%%%%%%%%%%%%%%%%%%%%%%%%%%%%%%%%%%%%%%%%%%%%%%%%%%%%%%%%
%\section{Acknowledgements}
%%%%%%%%%%%%%%%%%%%%%%%%%%%%%%%%%%%%%%%%%%%%%%%%%%%%%%%%%%%%%%%%%%%%%%%%%%%%%


\begin{markdown}

# Acknowledgements

This template is a compilation of common mistakes, reviewer feedback, and paper writing experience. It is intended to help identify the `little things' that often get in the way of a successful CHI submission, and may be helpful in reviewing your paper \textit{well before} the deadline. For example, to guide peer reviews, such as reviewing circles, or to help communicate necessary revisions. 

While I compiled the different examples, tips, and hacks in one place, they incorporate the experience, advice, and feedback from many others I've worked with, including (in alphabetical order): Robert Gauthier, Mark Hancock, Lennart Nacke, Adrian Reetz, Stacey Scott, and Dan Vogel. Suggestions are always welcome, please email to \href{mailto:james.wallace@uwaterloo.ca}{james.wallace@uwaterloo.ca}. 

\end{markdown}


\bibliographystyle{ACM-Reference-Format}
\bibliography{manual/bibliography.bib}

%\input{submission/11-appendix.tex}

%\newpage
%
%\pagestyle{fancy}
\rhead{\href{http://hci.uwaterloo.ca}{\includegraphics[width=3cm]{figures/uwhci-full-cs.png}}}
\lhead{\textcolor{gray}{\tiny{v3.0 -- \today}}}
%\renewcommand{\headrulewidth}{0pt}

\newenvironment{checklist}{%
  \begin{list}{}{}% whatever you want the list to be
  \let\olditem\item
  \renewcommand\item{\olditem[$\Box$] }
}{%
  \end{list}
}


%%%%%%%%%%%%%%%%%%%%%%%%%%%%%%%%%%%%%%%%%%%%%%%%%%%%%%
\section*{\textsc{CHI Paper Checklist}}
%%%%%%%%%%%%%%%%%%%%%%%%%%%%%%%%%%%%%%%%%%%%%%%%%%%%%%
The following checklist is a compilation of common mistakes, reviewer feedback, and paper writing experience. It is intended to help identify the `little things' that often get in the way of a successful CHI submission, and may be helpful in reviewing your paper \textit{well before} the deadline. For example, to guide peer reviews, such as reviewing circles, or to help communicate necessary revisions. 

The list was compiled by Jim Wallace, but incorporates the experience, advice, and feedback from many others at UWaterloo, including: Stacey Scott, Adrian Reetz, Mark Hancock, and Lennart Nacke. Suggestions are welcome, please email to \href{mailto:james.wallace@uwaterloo.ca}{james.wallace@uwaterloo.ca}. 


\noindent\makebox[\linewidth]{\rule{\linewidth}{0.4pt}}
%%%%%%%%%%%%%%%%%%%%%%%%%%%%%%%%%%%%%%%%%%%%%%%%%%%%%%
\subsection*{\textsc{The Big Picture}}
%%%%%%%%%%%%%%%%%%%%%%%%%%%%%%%%%%%%%%%%%%%%%%%%%%%%%%
\vspace{0.5cm}
\begin{checklist}
	\item What is the specific contribution of the paper? (e.g., Precisely how does your design build on existing work? or What new findings does this study provide over similar existing studies?)
	\item Is the research described clearly? Could a fellow student replicate the work if all they had was this paper?  
	\item Have you justified the research methods? (e.g., Construct / Internal / External validity)
	
	\item Have you summarized your findings? Are take-home messages clear and meaningful? 
\end{checklist}


% \noindent\makebox[\linewidth]{\rule{\linewidth}{0.4pt}}
% %%%%%%%%%%%%%%%%%%%%%%%%%%%%%%%%%%%%%%%%%%%%%%%%%%%%%%
% \subsection*{\textsc{Structure}}
% %%%%%%%%%%%%%%%%%%%%%%%%%%%%%%%%%%%%%%%%%%%%%%%%%%%%%%
% \begin{multicols}{2}
% \begin{itemize}
%   \item Introduction
%   	\begin{checklist}
% 		\item General Introduction
% 		\item Problem Statement
% 		\item Summary of findings
% 	\end{checklist}
	
%   \item Related Work
%   	\begin{checklist}
% 		\item General Introduction
% 		\item Problem Statement
% 		\item What has been done?
% 		\item How did it shape this work?
% 	\end{checklist}
   
%   \item Experimental Design
%   	\begin{checklist}
% 		\item Participants
% 		\item Experimental Setup
% 		\item Experimental Design
% 		\item Experimental Task
% 		\item Hypotheses
% 		\item Procedure
% 		\item Data Collection and Analysis
% 		\item Results
% 	\end{checklist}
	
%   \item Discussion
%   	\begin{checklist}
% 		\item All results discussed?
% 		\item Implications for Design/Theory
% 		\item Take-home message?
% 	\end{checklist}


%   \item Conclusion
%   	\begin{checklist}
% 		\item Impact Statement
% 		\item Summary of findings
% 		\item General Implications
% 	\end{checklist}
	
% \end{itemize}
% \end{multicols}



\newpage
%%%%%%%%%%%%%%%%%%%%%%%%%%%%%%%%%%%%%%%%%%%%%%%%%%%%%%
\section*{\textsc{Grammar}}
%%%%%%%%%%%%%%%%%%%%%%%%%%%%%%%%%%%%%%%%%%%%%%%%%%%%%%

\begin{checklist}
	\item Has a friend proof-read the full draft of your paper? 
	\item When writing about numbers less than or equal to ten, use full words: ``one'', ``two'' ...etc, 
	\item When a sentence starts with a number, it should also use words (e.g., ``Twenty-four participants were recruited...'') 
	\item Be consistent (e.g., ``There were 23 participants who wrote CHI papers carefully, and only 2 that wrote them at the last minute'')
\end{checklist}



\noindent\makebox[\linewidth]{\rule{\linewidth}{0.4pt}}
%%%%%%%%%%%%%%%%%%%%%%%%%%%%%%%%%%%%%%%%%%%%%%%%%%%%%%
\section*{\textsc{Words and Phrases to Avoid}}
%%%%%%%%%%%%%%%%%%%%%%%%%%%%%%%%%%%%%%%%%%%%%%%%%%%%%%
\begin{multicols}{5}
\small
\begin{itemize}
\setlength{\itemsep}{0.5pt}
\item[] actually
\item[] almost
\item[] always
\item[] anyway
\item[] basically
\item[] best
\item[] believe
\item[] can't
\item[] clearly
\item[] don't
\item[] essentially
\item[] feel
\item[] folks
\item[] good
\item[] guy
\item[] hassle
\item[] I
\item[] impossible
\item[] kids
\item[] kind of
\item[] like
\item[] lots
\item[] many
\item[] never
\item[] nice
\item[] obviously
\item[] OK
\item[] of course
\item[] pretty good
\item[] prove(n)
\item[] quite
\item[] really
\item[] so
\item[] stuff
\item[] thing
\item[] totally
\item[] user
\item[] various
\item[] very
\item[] well
\item[] won't
\item[] wouldn't
\item[] you
\end{itemize}

\end{multicols}


\noindent\makebox[\linewidth]{\rule{\linewidth}{0.4pt}}
%%%%%%%%%%%%%%%%%%%%%%%%%%%%%%%%%%%%%%%%%%%%%%%%%%%%%%
\section*{\textsc{Data Collection, Analysis, and Results}}
%%%%%%%%%%%%%%%%%%%%%%%%%%%%%%%%%%%%%%%%%%%%%%%%%%%%%%

Be sure to report and describe: 
\begin{checklist}
	\item What data was collected? (i.e., independent variables) And how? (e.g., computer logs, video recordings, field notes, questionnaires, etc.) 

	\item What qualitative data analysis method(s) were conducted (e.g., high-level video review, affinity diagramming on your interview data, in-depth video coding, etc.)
	
	\item What quantitative data analysis method(s) were used? (e.g., One-Way Analysis of Variance (ANOVA), Repeated-measures Analysis of Variance (RM-ANOVA), t-test, regression, etc.) 	
	
	\begin{enumerate}
 		\item State your alpha value, and any transformations or adjustments that were applied (e.g., for family-wise errors)
		\item Report summary statistics of your actual data (e.g. mean, standard deviations, or standard error) as well as the statistical test results ($t$-values, $F$-values, $p$-values, effect size, etc.)
	\end{enumerate}
	\item When including any figures or tables of your data, always indicate directly in the figure or table any statistically significant differences between data values
\end{checklist}


\newpage
%%%%%%%%%%%%%%%%%%%%%%%%%%%%%%%%%%%%%%%%%%%%%%%%%%%%%%
\section*{\textsc{Figures and Tables}}
%%%%%%%%%%%%%%%%%%%%%%%%%%%%%%%%%%%%%%%%%%%%%%%%%%%%%%
\vspace{0.5cm}
Do all of the Figures and Tables:
\begin{checklist}
	\item Clearly highlight the relevant information? Can they be understood without reading the paper? 
	\item Have a caption and title? Are they annotated? 
	\item Have labelled axes, and appropriate units / scale?
	\item Are they legible when printed? In black and white?
	\item Are they referenced in the body text? Are they numbered sequentially?
\end{checklist}


\noindent\makebox[\linewidth]{\rule{\linewidth}{0.4pt}}
%%%%%%%%%%%%%%%%%%%%%%%%%%%%%%%%%%%%%%%%%%%%%%%%%%%%%%
\section*{\textsc{References}}
%%%%%%%%%%%%%%%%%%%%%%%%%%%%%%%%%%%%%%%%%%%%%%%%%%%%%%
\vspace{0.5cm}

\begin{checklist}
	\item Have you \begin{em}\textbf{thoroughly}\end{em} checked your reference list at the end of your paper?
	\item Have you checked references to author's names in the main text for spelling?
\end{checklist}

\begin{center}

\small % The size of the table text can be changed depending on content. Remove if desired.
\def\arraystretch{2}
\begin{tabular}{p{6cm} p{5cm} p{5cm}}
& \textbf{YES} & \textbf{NO} \\
 \cmidrule{2-2}  \cmidrule{3-3} 
Include in-line citations in sentence & your claim [1]. & your claim. [1]. \\
Citations follow author's name & Bauer et al. [1] claimed X, Y, and Z. & Bauer et al. claimed X, Y, and Z [1]. \\
Citations are not nouns & ... as Bauer et al. [1] found ... & ... as [1] found ... \\
Citations immediately follow claims & Previous research has found X [1], Y [2,3], and Z [4, 5]. &  Previous research has found X, Y, and Z [1, 2, 3, 4, 5]. \\
All claims backed by a citation & The long-term use of smartphones causes increased anxiety [1, 2, 4]. & The long-term use of smartphones causes anxiety.  \\
Summarize relevant citations, including findings from previous studies & Bauer et al. [1] previously showed that FitBit's have long-term benefits for mental health. They found initial positive health benefits ... & Bauer et al. [1] previously studied the impact of FitBit on long-term mental health. \\
\end{tabular}
\end{center}


\noindent\makebox[\linewidth]{\rule{\linewidth}{0.4pt}}
%%%%%%%%%%%%%%%%%%%%%%%%%%%%%%%%%%%%%%%%%%%%%%%%%%%%%%
\section*{\textsc{Formatting \& Misc.}}
%%%%%%%%%%%%%%%%%%%%%%%%%%%%%%%%%%%%%%%%%%%%%%%%%%%%%%
\vspace{0.5cm}

\begin{checklist}
 	\item Did you use the most recent \href{https://chi2021.acm.org/for-authors/chi-publication-formats}{ACM SIGCHI Paper Format}?
%	\item Does the paper fall within the conference page length guidelines? 
	\item Are all acronyms defined? Are footnotes used sparingly? 
\end{checklist}











\end{document}
\endinput
%%
%% End of file.
