
%\vspace{2em}
%\begin{quote}
%    \emph{Great things are done by a series of small things brought together. --- Vincent Van Gough}
%\end{quote}

\begin{markdown}
%%%%%%%%%%%%%%%%%%%%%%%%%%%%%%%%%%%%%%%%%%%%%%%%%%%%%%%%%%%%%%%%%%%%%%%%%%%%%
%\section{Introduction}
%%%%%%%%%%%%%%%%%%%%%%%%%%%%%%%%%%%%%%%%%%%%%%%%%%%%%%%%%%%%%%%%%%%%%%%%%%%%%

# Introduction
Even the most experienced researchers are constantly looking for ways to improve their papers. And there are many different aspects of a paper to improve on: from high-level concerns like experimental design, positioning the paper within the literature, and consideration of ethical issues, to sweating the details in a statistical analyis, or getting that graph \emph{just right} to show off your earth-shattering results. But sometimes it's hard to put all the pieces together. That's where this project comes in. 
\end{markdown}

This project --- CHI Zen, a play on the words \begin{CJK}{UTF8}{min}改善\end{CJK} (`KaiZen') or `continuous improvement' --- describes common pitfalls and best practices based on my experience writing academic papers. That experience reflects some of my own preferences, but also a lot that I've learned from working with other HCI researchers. It also contains a lot of examples that you are free to copy, modify, or re-use as often as you'd like. The goal is to have something of an \emph{information zoo} \citep{heer2010tour} that can help you think about how to best visualize results, and hopefully make the process of creating that visualization easy, too. 

\begin{markdown}
This document also provides a sample reference for a set of templates that I developed for \texttt{PGFPlots}. Please use \texttt{PGFPlots} if you find it useful. If you do, I'd love to hear about it, and if you have any ways to improve them. You might also prefer to use a different graphing package, and that's cool too. The examples contained in this document hopefully can provide tips or shortcuts for use in your own papers. I suspect that you can generate more or less the same graphs with your tool of choice. 

Finally, my intention in making this project available via GitHub is to allow it to grow. If you find it useful, please drop me a line. If you find a way to improve it, please push the changes to GitHub. It might also just be an important prompt to have a conversation about best practices, or an opportunity to explore different ways of thinking about or presenting research. In any case, I hope that it's an opportunity to think about how we can improve our research practices, one small step at a time. 

\end{markdown}