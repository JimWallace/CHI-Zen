\newpage

%
% This section written in LaTeX because a lot of the graphing is untested in Markdown
%


%%%%%%%%%%%%%%%%%%%%%%%%%%%%%%%%%%%%%%%%%%%%%%%%%%%%%%%%%%%%%%%%%%%%%%%%%%%%%
\section{Figures}
%%%%%%%%%%%%%%%%%%%%%%%%%%%%%%%%%%%%%%%%%%%%%%%%%%%%%%%%%%%%%%%%%%%%%%%%%%%%%

\begin{markdown}
Figures and Tables provide your readers with a `visual abstract' of your work. They can make the (often dense) results section more accessible to readers who are unfamiliar with your research area, and are a fallback when something in the text doesn't make sense right away. That is, they're often the first impression a reader has of your paper, make your work more accessible to the general audience, and provide a second, more visual, way of explaining results to your readers.

A general principal that I try to embody in all of these graphs is to **show your data**. Too often, we provide only summary data like an average and a standard error value, when we could have shown the entire data set. Summary statistics are useful, but often hide important trends in the data. (We should take this to heart well before writing the paper up, this all holds true for *data analysis*). I think that in the past, tools weren't always up to the task of showing our data. Tools like Excel would help you easily produce a bar chart, but don't have a preset for box plots. Fortunately, today our tools have become much better. 

I've also tried to take best practice into consideration where possible, but there's always room for improvement. \citet{munznervisualization} provides an excellent summary of best practice from the literature, and I encourage you to read Chapter 6 on Rules of Thumb in depth to start thinking about how to approach thinking about visualizations as you're working on them. In particular, tips like **get it right in black and white** and **eyes beat memory** will help to improve the readability *and* accessibility of your paper. 

The following sections are framed in terms of problem/solution for common types of graphs that we see in HCI research, with examples for each. The style for each graph is informed and inspired by \citet{tufte1983visual}, who emphasizes minimalist presentation of quantitative data. I find that Tufte's principles of information visualization, and general approach to working on graphs, are largely consistent with Munzner's advice above. That said, the style cetainly isn't for everyone --- your mileage may vary, please use what you find useful.  


 \end{markdown}
 
%  For example, some good examples:
% Bartneck and Hu \cite{Bartneck:2009:SAC:1518701.1518810}


% Senellart \cite{Senellart:2013:DYR:2513166.2514938} discusses some useful cases of `bad' style, which superficially may appear to be picky but offer important guidance towards submitting polished papers. 
 
 
% Munzer \cite{Munzner:2008:PPW:1422919.1422927} 
% - some good advice in general, but more focused on info vis research

% - text is large and legible
% - all axes are labelled, including units of measure
% - axes start at 0, unless explicitly justified (for a good reason!)






\newpage
%%%%%%%%%%%%%%%%%%%%%%%%%%%%%%%%%%%%%%%%%%%%%%%%%%%%%%%%%%
\subsection{Comparing Performance: Bar Charts}
%%%%%%%%%%%%%%%%%%%%%%%%%%%%%%%%%%%%%%%%%%%%%%%%%%%%%%%%%%

- common to compare performance, efficiency, error rate, or other dependent measures between two study groups. For example, our new technique for 3D pointing, compared to a computer mouse. 
- most simple way of doing so is to show the means and standard error in a bar graph, and this is commonly accepted throughout the ACM. 
- For example, Figure \ref{bargraph} is an example drawn from \citet{pietroszek2015}.





\begin{figure}[b!]
\begin{tikzpicture}
\begin{axis}[IDEA bar,
        symbolic x coords={non-occluded,occluded},
        xtick = {non-occluded,occluded},
        enlarge x limits=0.5,
	    ymin = 0, ymax = 10,
        ylabel = {Selection Time (s)},
        width=\columnwidth,
        height = 5cm,
]

%Depth Ray
\addplot[style={draw=none,fill=AHSLight},error bars/.cd, y dir = both, y explicit, error bar style={black}]
coordinates {(occluded, 6.8)  +- (1.7,1.7)};
\addlegendentry{Depth Cursor}

%Tiltcasting
\addplot+[style={draw=none, fill=SchoolRedLight,line width = 0pt},error bars/.cd, y dir = both, y explicit, error bar style={black}]
coordinates {(non-occluded,3.24) +- (.712,.712)
		    (occluded,4)  +- (.922,.922)};
\addlegendentry{Tiltcasting}

%Smartcasting
\addplot+[style={draw=none, fill=EnvironmentLight},error bars/.cd, y dir = both, y explicit, error bar style={black}]
coordinates {(non-occluded, 2.10)  +- (.42,.42)};
\addlegendentry{Smartcasting}

\end{axis}
\end{tikzpicture}
\caption{An example bar graph with error bars. Data is specified in the latex file.}
\label{bargraph}
\end{figure}






\newpage
%%%%%%%%%%%%%%%%%%%%%%%%%%%%%%%%%%%%%%%%%%%%%%%%%%%%%%%%%%
\subsection{Comparing Performance: Box Plots, Violin Plots, Bee Swarm Plots}
%%%%%%%%%%%%%%%%%%%%%%%%%%%%%%%%%%%%%%%%%%%%%%%%%%%%%%%%%%


- to be more accurate and descriptive, box plots are often used
- similar to a bar graph with error bars, but also indicate quartiles and outliers 
- typically, better practice to include a box plot when possible -- but can be substantially more difficult to create in Microsoft Word. 

variations: 
- Admittedly, "Bee Swarm" plots are my personal preference for this kind of data since they "show your data" in a very clear way... but I haven't yet found a  way to generate them in \LaTeX. If you happen to work in R, there are some excellent supports for generating these graphs, and you can import those directly into this template with a little work. 
- Violin plots 




\newpage
%%%%%%%%%%%%%%%%%%%%%%%%%%%%%%%%%%%%%%%%%%%%%%%%%%%%%%%%%%
\subsection{Likert Scale Data: Stacked Divergent Bar Graphs}
%%%%%%%%%%%%%%%%%%%%%%%%%%%%%%%%%%%%%%%%%%%%%%%%%%%%%%%%%%

A \emph{lot} of HCI research involves questionnaires, and interpreting diferences in questionnaire respones. For instance, Likert scale questions are very common. My preference for visualizing Likert scale responses is a stacked divergent bar graph (e.g., \autoref{fig:avatar_identification}), where responses are stacked into a bar, with neutral responses lined up at the graph's origin. This type of graph has a few properties that make it useful: in particular, readers can get a quick sense of `how positive' or `how negative' responses were based on the bar's position, while still being able to look at how many responses fell into each category. 

\begin{figure}[b!]
\begin{tikzpicture}
\pgfplotstableread[col sep = comma]{data/Cleric_AvatarIdentificationStacked.csv}\ClericIdentificationData
\pgfplotstableread[col sep = comma]{data/Monster_AvatarIdentificationStacked.csv}\MonsterIdentificationData

\begin{axis}[
    IDEA Likert,
    height = 6.5cm,
    width = .75\columnwidth,
    % Y AXIS
    symbolic y coords = {Ava_Conn,	Ava_WE,	AvavsPhys,	AvaVsPers,	AvaVsPhysIdeal,	AvaVsPersIdeal},
    ytick={Ava_Conn, Ava_WE, AvavsPhys, AvaVsPers, AvaVsPhysIdeal, AvaVsPersIdeal},
    yticklabels={Connectedness, Refer to \\ Avatar as ``We'', Physical \\ Similarity, Personality\\ Similarity, Physical Ideal, Ideal Personality},
    yticklabel style={align=right,font={\sffamily\small}},
    y axis line style={draw=none},
    y dir = reverse,
    ymajorgrids = false,
    % X AXIS
    xmin=-100, xmax=100,
    xlabel={\% of Responses},
    xlabel style = {font=\small\sffamily},
    xticklabel style = {font=\small\sffamily},
    xtick = {-100,-50,0,50,100},
    xticklabels = {100, 50, 0, 50, 100},
    extra x ticks = {-75, 75},
    extra x tick labels = {$\leftarrow$ Disagree, Agree $\rightarrow$},
    every extra x tick/.style={major tick length=0,yshift={-8pt}},
    ]
    \addplot[draw=none,fill=none, forget plot] coordinates {(0,Ava_Conn)(0,Ava_WE)(0,AvavsPhys)(0,AvaVsPers)(0,AvaVsPhysIdeal)(0,AvaVsPersIdeal)};
    \addplot[draw=none,fill=Cleric_3, forget plot, bar shift=.1cm] table [x expr={\thisrow{4}*-0.5},y=Measure] {\ClericIdentificationData};
    \addplot[draw=none,fill=Cleric_2, forget plot, bar shift=.1cm] table [x expr={\thisrow{3}*-1},y=Measure] {\ClericIdentificationData};
    \addplot[draw=none,fill=Cleric_1, forget plot, bar shift=.1cm] table [x expr={\thisrow{2}*-1}, y=Measure ] {\ClericIdentificationData};
    \addplot[draw=none,fill=Cleric_0, forget plot, bar shift=.1cm] table [x expr={\thisrow{1}*-1}, y=Measure ] {\ClericIdentificationData};
    \resetstackedplots
    \addplot[draw=none,fill=none, forget plot] coordinates {(0,Ava_Conn)(0,Ava_WE)(0,AvavsPhys)(0,AvaVsPers)(0,AvaVsPhysIdeal)(0,AvaVsPersIdeal)};
    \addplot[draw=none,fill=Monster_3, forget plot, bar shift=-.1cm] table [x expr={\thisrow{4}*-0.5},y=Measure] {\MonsterIdentificationData};
    \addplot[draw=none,fill=Monster_2, forget plot, bar shift=-.1cm] table [x expr={\thisrow{3}*-1},y=Measure] {\MonsterIdentificationData};
    \addplot[draw=none,fill=Monster_1, forget plot, bar shift=-.1cm] table [x expr={\thisrow{2}*-1}, y=Measure ] {\MonsterIdentificationData};
    \addplot[draw=none,fill=Monster_0, forget plot, bar shift=-.1cm] table [x expr={\thisrow{1}*-1}, y=Measure ] {\MonsterIdentificationData};
\end{axis}

\begin{axis}[
    IDEA Likert,
    height = 6.5cm,
    width = .75\columnwidth,
    % Y AXIS
    symbolic y coords = {Ava_Conn,	Ava_WE,	AvavsPhys,	AvaVsPers,	AvaVsPhysIdeal,	AvaVsPersIdeal},
    ytick={Ava_Conn, Ava_WE, AvavsPhys, AvaVsPers, AvaVsPhysIdeal, AvaVsPersIdeal},
    yticklabels={Connectedness, Refer to \\ Avatar as ``We'', Physical \\ Similarity, Personality\\ Similarity, Physical Ideal, Ideal Personality},
    yticklabel style={align=right,font={\sffamily\small}},
    y axis line style={draw=none},
    y dir = reverse,
    ymajorgrids = false,
    % X AXIS
    xmin=-100, xmax=100,
    xlabel={\% of Responses},
    xlabel style = {font=\small\sffamily},
    xticklabel style = {font=\small\sffamily},
    xtick = {-100,-50,0,50,100},
    xticklabels = {100, 50, 0, 50, 100},
    extra x ticks = {-75, 75},
    extra x tick labels = {$\leftarrow$ Disagree, Agree $\rightarrow$ },
    every extra x tick/.style={major tick length=0,yshift={-8pt}},
    ]
    \addplot[draw=none,fill=none, forget plot] coordinates {(0,Ava_Conn)(0,Ava_WE)(0,AvavsPhys)(0,AvaVsPers)(0,AvaVsPhysIdeal)(0,AvaVsPersIdeal)};
    
    \addplot[draw=none,fill=Cleric_3, forget plot, bar shift=.1cm] table [x expr={\thisrow{4}*0.5}, y=Measure ] {\ClericIdentificationData};
    \addplot[draw=none,fill=Cleric_4, forget plot, bar shift=.1cm] table [x expr={\thisrow{5}}, y=Measure ] {\ClericIdentificationData};
    \addplot[draw=none,fill=Cleric_5, forget plot, bar shift=.1cm] table [x expr={\thisrow{6}}, y=Measure ] {\ClericIdentificationData};
    \addplot[draw=none,fill=Cleric_6, forget plot, bar shift=.1cm] table [x expr={\thisrow{7}}, y=Measure ] {\ClericIdentificationData}; \label{plot:cleric6}
    \resetstackedplots
    \addplot[draw=none,fill=none, forget plot] coordinates {(0,Ava_Conn)(0,Ava_WE)(0,AvavsPhys)(0,AvaVsPers)(0,AvaVsPhysIdeal)(0,AvaVsPersIdeal)};
    \addplot[draw=none,fill=Monster_3, forget plot, bar shift=-.1cm] table [x expr={\thisrow{4}*0.5}, y=Measure ] {\MonsterIdentificationData};
    \addplot[draw=none,fill=Monster_4, forget plot, bar shift=-.1cm] table [x expr={\thisrow{5}}, y=Measure ] {\MonsterIdentificationData};
    \addplot[draw=none,fill=Monster_5, forget plot, bar shift=-.1cm] table [x expr={\thisrow{6}}, y=Measure ] {\MonsterIdentificationData};
    \addplot[draw=none,fill=Monster_6, forget plot, bar shift=-.1cm] table [x expr={\thisrow{7}}, y=Measure ] {\MonsterIdentificationData};
    %\addlegendimage{only marks, mark=o}
    %\addlegendimage{only marks, mark=o}
    %\legend{Neutral, Agree, Strongly Agree, Disagree,Strongly Disagree}
    after end axis/.code={
        \node at (axis cs:80,Ava_Conn) [anchor=east, ,font=\tiny] {\AsteriskBold};
        \node at (axis cs:80,Ava_WE) [anchor=east, font=\tiny] {\AsteriskBold};
        %\draw [white] (axis cs:0,AvaVsPersIdeal) -- (axis cs:0,Ava_Conn);
        %\draw [white] (axis cs:-50,AvaVsPersIdeal) -- (axis cs:-50,Ava_Conn);
        %\draw [white] (axis cs:50,AvaVsPersIdeal) -- (axis cs:50,Ava_Conn);
    }
    \coordinate (legend) at (axis description cs:1.15,.85);
\end{axis}

% this is a dummy `axis' environment only to create the legend
\matrix [
    matrix of nodes, 
    every node/.style={anchor=center}, 
    ] at (legend) {
        |[fill=Cleric_0]| & |[fill=Cleric_1]| & |[fill=Cleric_2]| & |[fill=Cleric_3]| & |[fill=Cleric_4]| & |[fill=Cleric_5]| & |[fill=Cleric_6]| & |[font=\small\sffamily]|Cleric \\
        |[fill=Monster_0]| & |[fill=Monster_1]| & |[fill=Monster_2]| & |[fill=Monster_3]| & |[fill=Monster_4]| & |[fill=Monster_5]| & |[fill=Monster_6]| & |[font=\small\sffamily]|Monster \\
    };
\end{tikzpicture}
    
\caption{Summary of avatar identification responses, based on 7-point Likert scales. Mann Whitney U tests revealed significant differences between \textsc{Avatar} groups for `Connectness' and `Refer to Avatar as We'. }
\label{fig:avatar_identification}
\end{figure}





\newpage
%%%%%%%%%%%%%%%%%%%%%%%%%%%%%%%%%%%%%%%%%%%%%%%%%%%%%%%%%%
\subsection{Trends over Time: Line Graph}
%%%%%%%%%%%%%%%%%%%%%%%%%%%%%%%%%%%%%%%%%%%%%%%%%%%%%%%%%%



\begin{figure}[b!]
\begin{tikzpicture}[baseline]
\begin{axis}[IDEA line,
	% Graph Options
	width=.475*\columnwidth,
	height=4cm,
	title=\texttt{r/stop\-drinking},
	% X Axis Options
	%date coordinates in=x,
	%date ZERO=2014-01-01,
    %xticklabel=\year,
    %xmin=2014-01-01,
    %xmax=2018-01-01,
    xlabel=Year,
	xtick={1,13,25,37,49},                              %% Jim: apologies... this is a hack
    xticklabels={2014,2015,2016,2017,2018},             %%      will figure out what's going on here if we have time later
	%xtick ={1,2,3,4,5,6,7,8,9},
	% Y Axis Options
	ylabel = Distinct Users / Active Threads,
	ylabel style = {font=\tiny},
	ymin = 0, ymax = 6000,
	ytick = {0,2000,4000,6000},
	ytick style = {font=\tiny},
    %legend image code/.code={\draw[#1] circle (0.1cm);}
]

\pgfplotstableread[col sep = comma]{data/stopdrinking-usagebymonth.csv}\mydata;
\addplot+[draw=ScienceDark,fill=none,thick] table[x=index,y=users] {\mydata};
\addplot+[draw=ArtsLight,fill=none,thick] table[x=index,y=threads] {\mydata};

\end{axis}
\end{tikzpicture}%
%
\begin{tikzpicture}[baseline]
\begin{axis}[IDEA line,
	% Graph Options
	width=.475*\columnwidth,	
	height=4cm,
	title=\texttt{r/Opiates\-Recovery},
	% X Axis Options
	xtick={1,13,25,37,49},                              %% Jim: apologies... this is a hack
    xticklabels={2014,2015,2016,2017,2018},             %%      will figure out what's going on here if we have time later
	%date coordinates in=x,
	%date ZERO=2014-01-01,
    %xticklabel=\year,
    %xmin=2014-01-01,
    %xmax=2018-01-01,
	xlabel = Year,
	%xtick ={1,2,3,4,5,6,7,8,9},
	% Y Axis Options
	ylabel = Distinct Users / Active Threads,
	ylabel style = {font=\tiny},
	ymin = 0, ymax = 800,
	ytick = {0,200,400,600,800},
	ytick style = {font=\tiny},
	legend style={at={(0.5,-1)},anchor=south}
    %legend image code/.code={\draw[#1] circle (0.1cm);}
]

\pgfplotstableread[col sep = comma]{data/opiaterecovery-usagebymonth.csv}\mydata;
\addplot+[draw=ScienceDark,fill=none,thick] table[x=index,y=users] {\mydata};
\addlegendentry{Distinct Users}
\addplot+[draw=ArtsLight,fill=none,thick] table[x=index,y=threads] {\mydata};
\addlegendentry{Active Threads}

\end{axis}
\end{tikzpicture}
%\setlength{\belowcaptionskip}{-20pt}
\caption{Distinct users and active threads for \texttt{r/stop\-drinking} (top) and \texttt{r/Opiates\-Recovery} (bottom) showing that both counts are experiencing an upwards trend over time for both subreddits.} 

\label{fig:activity}
\end{figure}



\newpage
%%%%%%%%%%%%%%%%%%%%%%%%%%%%%%%%%%%%%%%%%%%%%%%%%%%%%%%%%%
\subsection{Trends over Time: Stacked Bar Graph}
%%%%%%%%%%%%%%%%%%%%%%%%%%%%%%%%%%%%%%%%%%%%%%%%%%%%%%%%%%


\begin{figure*}[b!]
\begin{center}
\begin{tikzpicture}
\begin{groupplot}[
   group style={
       group size=1 by 5,
       x descriptions at=edge bottom,
       y descriptions at=edge left,
       vertical sep=1pt,
       horizontal sep=0pt},
  IDEA tufte panel, 
	% X Axis
	width=\columnwidth,
	xmin = 1990, xmax = 2015,
	xtick = {1990,1994,1998,2002,2006,2010,2015},
	x tick label style={font=\small,rotate=90, /pgf/number format/1000 sep=},
	axis line style={thick},
	% Y Axis
	ytick={10,20,30,40,50,60,70,80,90},
        ymin = 0,
        ymax = 50,
      ]


% Bibliographic
\nextgroupplot[bar width=7pt, height = 2.25cm, ymax = 25, title={Bibliographic}]
\addplot+[draw=none, fill=black!50] 
 coordinates {(1990,20) (1992,4) (1994,0) (1996,0) (1998,2) (2000,0) (2002,0) (2004,2) (2006,2) (2008,1) (2010,0) (2011,0) (2012,0) (2013,1) (2014,0) (2015,1) };       
  
% Not Empirical      
\nextgroupplot[bar width=7pt, height = 3.5cm, ymax=70, title={Not Empirical}]
\addplot+[draw=none, fill=black!50] 
coordinates {(1990,37) (1992,55) (1994,33) (1996,38) (1998,54) (2000,35) (2002,20) (2004,18) (2006,24) (2008,13) (2010,5) (2011,31) (2012,9) (2013,4) (2014,6) (2015,7) };

% Explanatory
\nextgroupplot[bar width=7pt, height = 3cm, title={Explanatory}]
\addplot+[draw=none, fill=black!50]  
coordinates {(1990,3) (1992,4) (1994,5) (1996,0) (1998,0) (2000,5) (2002,15) (2004,16) (2006,29) (2008,25) (2010,21) (2011,16) (2012,8) (2013,14) (2014,15) (2015,12) };

% D & E
\nextgroupplot[bar width=7pt, ymax=60, height = 3.5cm, title={Design and Evaluation}]
\addplot+[draw=none, fill=blue!50] 
coordinates {(1990,17) (1992,13) (1994,26) (1996,22) (1998,15) (2000,27) (2002,39) (2004,35) (2006,23) (2008,19) (2010,21) (2011,10) (2012,28) (2013,26) (2014,24) (2015,22) };

% Descriptive
\nextgroupplot[bar width=7pt, ymax = 70, height = 3.5cm, title={Descriptive}]
\addplot+[draw=none,fill=black!50]
coordinates {(1990,23) (1992,23) (1994,36) (1996, 40) (1998,29) (2000,32) (2002,27) (2004,29) (2006,23) (2008,42) (2010,53) (2011,42) (2012,55) (2013,55) (2014,55) (2015,58) };


\end{groupplot}
\end{tikzpicture}
\caption{An example of Tufte's stacked panel graph, which presents the same information as a stacked bar graph but better enables comparisons across each vertical dimension. Best practice is to use black and white when possible, with sparing colour for emphasis where appropriate. }
\label{stackedpanel}
\end{center}
\end{figure*}




% \begin{figure}[tb!]
% \begin{tikzpicture}
% \begin{groupplot}[
%   group style={
%       group size=1 by 5,
%       x descriptions at=edge bottom,
%       y descriptions at=edge left,
%       vertical sep=10pt,
%       horizontal sep=0pt},
%   IDEA line,
% 	% Graph Options
% 	height=3cm,
% 	 separate axis lines,
% 	% X Axis Options
% 	xmin=0, xmax= 35,
% 	xtick={1,5,10,15,20,25,30, 35},
% 	%date coordinates in=x,
% 	%date ZERO=2014-01-01,
%     %xticklabel=\year,
%     %xmin=2014-01-01,
%     %xmax=2018-01-01,
% 	xlabel = Number of NPC Requests,
% 	xlabel style = {font=\small\sffamily},
%     xticklabel style = {font=\small\sffamily},
% 	%xtick ={1,2,3,4,5,6,7,8,9},
% 	% Y Axis Options
% 	%ylabel = Proportion of Responses,
% 	ymajorgrids = false,
% 	ylabel shift = -.3cm,
% 	ymin = 0, ymax = 1,
% 	ytick = {0, 1.00},
% 	ylabel style = {font=\small\sffamily},
%     yticklabel style = {font=\small\sffamily},
% 	%ytick style = {font=\tiny},
% 	title style = {yshift=-.75cm,font=\sffamily\small}
% ]
% \pgfplotstableread[col sep = comma]{data/CBT_PropOverTime.csv}\cbtovertimedata;

% \nextgroupplot[xtick style = {white}, title style = {yshift=+.35cm,font=\sffamily\small}, title={Not Damaging}]
% \addplot+[mark=*, mark size=1.5pt, draw=gray, only marks,fill=gray,thin] table[x=Query,y=NotDamaging] {\cbtovertimedata};
% \addplot [solid, draw=black, fill=none,ultra thick] table[y={create col/linear regression={y=NotDamaging}}] {\cbtovertimedata};
% %\addlegendentry{Not Damaging}

% \nextgroupplot[xtick style = {white}, title={Empathize}]
% \addplot+[mark=*, mark size=1.5pt, draw=gray, only marks,fill=gray,thin] table[x=Query,y=Empathize] {\cbtovertimedata};
% \addplot [solid, draw=black, fill=none,ultra thick] table[y={create col/linear regression={y=Empathize}}] {\cbtovertimedata};
% %\addlegendentry{Empathize}

% \nextgroupplot[xtick style = {white}, title style = {yshift=+.25cm,font=\sffamily\small}, title={Reframe}]
% \addplot+[mark=*, mark size=1.5pt, draw=gray, only marks,fill=gray,thin] table[x=Query,y=Reframe] {\cbtovertimedata};
% \addplot [solid, draw=black, fill=none,ultra thick] table[y={create col/linear regression={y=Reframe}}] {\cbtovertimedata};
% %\addlegendentry{Reframe}

% \nextgroupplot[xtick style = {white}, title={Encourage}]
% \addplot+[mark=*, mark size=1.5pt, draw=gray, only marks,fill=gray,thin] table[x=Query,y=Encourage] {\cbtovertimedata};
% \addplot [solid, draw=black, fill=none,ultra thick] table[y={create col/linear regression={y=Encourage}}] {\cbtovertimedata};
% %\addlegendentry{Encourage}

% \nextgroupplot[title={Solution}]
% \addplot+[mark=*, mark size=1.5pt, draw=gray, only marks,fill=gray,thin] table[x=Query,y=Solution] {\cbtovertimedata};
% \addplot [solid, draw=black, fill=none,ultra thick] table[y={create col/linear regression={y=Solution}}] {\cbtovertimedata};
% %\addlegendentry{Solution}


% %\addplot+[mark=x, only marks,draw=Monster_0,fill=none,semithick] table[x=Query,y=Proportion, y error plus=CI-upper, y error minus=CI-lower] {\barbariansolutiondata};
% %\addlegendentry{Monster}
% %\addplot [draw=Monster_0, fill=none,ultra thick] table[y={create col/linear regression={y=Proportion}}] {\barbariansolutiondata};

% \end{groupplot}
% \end{tikzpicture}

% %\setlength{\belowcaptionskip}{-20pt}
% \caption{The proportion of responses that were classified as meeting five criteria: Empathize, Reframe, Encourage, Solution, and Not Damaging. The three categories related to CBT, Empathize, Reframe, and Encourage, were found to decrease over time ($p=0.05$). However, we did not find these changes for Solution and Not Damaging. 
% %decreased over time for all participants, however responses by participants in the \textsc{Cleric} group decreased more quickly than those in the \textsc{Monster} group.
% } 

% \label{fig:SolutionsOverQueryNumber}
% \end{figure}



\newpage
%%%%%%%%%%%%%%%%%%%%%%%%%%%%%%%%%%%%%%%%%%%%%%%%%%%%%%%%%%
\subsection{Illustrating Scenarios: Rotograph}
%%%%%%%%%%%%%%%%%%%%%%%%%%%%%%%%%%%%%%%%%%%%%%%%%%%%%%%%%%

- You don't need to be an artist
- take a picture, trace over it
- often by far the most effective technique for simply demonstrating key concepts in your paper 




\newpage
%%%%%%%%%%%%%%%%%%%%%%%%%%%%%%%%%%%%%%%%%%%%%%%%%%%%%%%%%%
\subsection{PRISMA Flow Diagram}
%%%%%%%%%%%%%%%%%%%%%%%%%%%%%%%%%%%%%%%%%%%%%%%%%%%%%%%%%%

- Example provided by Katja Rogers
- http://www.prisma-statement.org
- ``PRISMA is an evidence-based minimum set of items for reporting in systematic reviews and meta-analyses. PRISMA focuses on the reporting of reviews evaluating randomized trials, but can also be used as a basis for reporting systematic reviews of other types of research, particularly evaluations of interventions.''


\pgfdeclarelayer{background}
\pgfsetlayers{background,main}
\begin{figure}[b!]
{
\footnotesize
\begin{tikzpicture}
    \tikzstyle{block}=[
        draw,
        thick,
        text width=5cm,
        minimum height=1cm,
        align=center
    ]
    \tikzstyle{header}=[rotate=90, align=center]
    \tikzstyle{line}=[-latex, draw, very thick]
    \node[block] (included-0) {Studies included in qualitative and quantitative synthesis (N=60)};
    \node[block, above=2em of included-0] (eligibility-1) {Full-text articles assessed for eligibility \\ (N=791)};
    \node[block, right=2em of eligibility-1] (eligibility-2) {Full-text articles excluded, with reasons \\ (N=731)};
    \node[block, above=2em of eligibility-1] (screening-1) {Records screened \\ (N=1030)};
    \node[block, right=2em of screening-1] (screening-2) {Records excluded \\ (N=239)};
    \node[block, above=2em of screening-1] (screening-0) {Records after duplicates removed\\ (N=1030)};
    \node[block, above=2em of screening-0, minimum height=1.5cm] (identification-1) {Records identified through database searching \\ (N=1035)}; 
    \coordinate (screening-mid) at ($ (screening-0.west)!0.5!(screening-1.south west) $);
    \coordinate (identification-mid) at ($ (screening-0.west)!0.5!(identification-1.north west-|screening-0.west) $);
    \node[header, left=1cm of included-0.west, anchor=north] (included) {\textbf{Included}}; %left=\textwidth/6 of included-0.west, anchor=north
    \node[header, left=1cm of eligibility-1.west, anchor=north] (eligibility) {\textbf{Eligibility}};
    \node[header, left=1cm of screening-1.west, anchor=north] (screening) {\textbf{Screening}};
    \node[header, left=1cm of identification-mid, anchor=north] (identification) {\textbf{Identification}};
    % Draw edges
    \path [line] (identification-1) -- ([yshift=.5cm]screening-0.north) -- (screening-0);
    \path [line] (screening-0) -- (screening-1);
    \path [line] (screening-1) -- (screening-2);
    \path [line] (screening-1) -- (eligibility-1);
    \path [line] (eligibility-1) -- (eligibility-2);
    \path [line] (eligibility-1) -- (included-0);
\end{tikzpicture}
}
\caption{Number of results at each stage of the review, as represented in a PRISMA flow diagram.}
\end{figure}




\newpage
%%%%%%%%%%%%%%%%%%%%%%%%%%%%%%%%%%%%%%%%%%%%%%%%%%%%%%%%%%
\subsection{Scheduling: Gantt Charts}
%%%%%%%%%%%%%%%%%%%%%%%%%%%%%%%%%%%%%%%%%%%%%%%%%%%%%%%%%%

- supported by the \texttt{pgfgantt} package

\begin{figure}[b!]
\begin{ganttchart}[
    x unit=.208cm,
    y unit title=0.4cm,
    y unit chart=0.3cm,
    vgrid,
    time slot format=isodate-yearmonth,
    time slot unit=month,
    vgrid={*{3}{draw=none},dotted,*{3}{draw=none},dotted,*{3}{draw=none},*{1}{draw=gray}},
    %
    % TITLE OPTIONS
    %
    title/.style={draw=none, fill=none},
    title label node/.append style={below=-1.6ex},
    title top shift=-.5,
    title label font=\tiny,
    %
    % BAR OPTIONS
    %
    bar height=.5,
    bar/.append style={draw = none, fill=blue!50},
    bar label font=\tiny,
    %   
    % GROUP OPTIONS
    %
    group label font={\tiny},
    group right shift=0,
    group top shift=.6,
    group height=.3,
    group peaks height=.2,
   ]{2018-05}{2023-04}

    
    \gantttitle{\textbf{Year 1: 2019 -- 2020}}{12}
    \gantttitle{\textbf{Year 2: 2020 -- 2021}}{12}
    \gantttitle{\textbf{Year 3: 2021 -- 2022}}{12}
    \gantttitle{\textbf{Year 4: 2022 -- 2023}}{12}
    \gantttitle{\textbf{Year 5: 2023 -- 2024}}{12}
                
\ganttset{progress label text={}, link/.style={black, -to}}

\ganttgroup{\textbf{Objective 1: Design Framework}}{2018-05}{2021-04} \\
\ganttbar{Literature Review (1.1)}{2018-05}{2019-03} \\
\ganttbar{Artifact Review (1.2)}{2018-09}{2019-07} \\
\ganttbar{Framework Development (1.3)}{2019-05}{2020-04} \\
\ganttbar{Mentoring \& Publication (1.4)}{2020-05}{2021-04} \\

\ganttgroup{\textbf{Objective 2: Interaction Mechanics}}{2020-05}{2022-04} \\
\ganttbar{Design (2.1)}{2020-05}{2020-07} \\
\ganttbar{Development (2.2)}{2020-08}{2020-12} \\
\ganttbar{Lab Study (2.3)}{2021-01}{2021-07} \\
\ganttbar{Analysis \& Publication (2.4)}{2021-07}{2022-04} \\

\ganttgroup{\textbf{Objective 3: In-Situ Evaluation}}{2021-05}{2023-04} \\
\ganttbar{Requirements Gathering (3.1)}{2021-05}{2021-10} \\
\ganttbar{Development (3.2)}{2021-11}{2022-06} \\
\ganttbar{Ecological Validation (3.3)}{2022-06}{2022-12} \\
\ganttbar{Analysis \& Publication (3.4)}{2023-01}{2023-04} 


\end{ganttchart}
\caption{Research activities are designed to scaffold development of a design framework and evaluation methodology, culminating in an in-situ evaluation with Homewood Research Institute.}
\label{gantt}
\end{figure}
