% Katja's stats stuff
% Shorthand for common terms
\newcommand{\chisq}[1]{$\chi^2$(#1)=}
\newcommand{\mdn}{\emph{Mdn}=}
\newcommand{\iqr}{\emph{IQR}=}
\newcommand{\mean}{\emph{M}=}
%\newcommand{\sd}{\emph{SD}=}
%\newcommand{\p}{\emph{p}$<$}
\newcommand{\effect}{\emph{r}=}
\newcommand{\W}{\emph{W}=}
\newcommand{\V}{\emph{V}=}
\newcommand{\ttest}[1]{\emph{t}(#1)=}
\newcommand{\quoting}[1]{``\emph{#1}''}
\newcommand{\partic}[2]{``\emph{#1}''\textasciitilde\emph{#2}}%\texttildelow


% Math and Measurements
% --------------------------------

\newcommand{\vect}[1]{\boldsymbol{#1}}
\newcommand{\degree}{\ensuremath{{}^{\circ}}\xspace}
% a nicer cross, like 45 x 56 cm
\newcommand{\by}{\ensuremath{\times}\xspace}
\newcommand{\size}[3]{{#1}~{\by}~{#2}{#3}}


% Statistics Reporting 
% --------------------------------

% standard deviation
\newcommand{\sd}[1]{\mbox{{\textsc{sd}}~=~#1\xspace}}

% ANOVAS

% prefer this less than inequality (with thresholds of .01, .001, .0001, ...) when p < .01
% Typical Anova: DOF1, DOF2, F value, less than p
\newcommand{\anova}[4]{\mbox{$F_{#1,#2} =~#3$}, \mbox{$p~<~#4$}} 
% Typical Anova with effect size: DOF1, DOF2, F value, less than p, partial eta sq
\newcommand{\anovae}[5]{\mbox{$F_{#1,#2}~=~#3$}, \mbox{$p~<~#4$}, \mbox{$\eta^2_{p}~=~#5$}} 
% if close to 0.05, can use equality for p using the following
% Exact Anova: DOF1, DOF2, F value, exact p
\newcommand{\anovaex}[4]{\mbox{$F_{#1,#2}~=~#3$}, \mbox{$p~=~#4$}} 
% Exact Anova: DOF1, DOF2, F value, exact p, cohen's d, partial eta sq
\newcommand{\anovaexe}[5]{\mbox{$F_{#1,#2}~=~#3$}, \mbox{$p~=~#4$}, \mbox{$\eta^2_{p}~=~#5$}} 

% NOTE: usually don't give f-value, and dof for not signiciant tests, but this macro will do it if you need to
% Not significant Anova: DOF1, DOF2, F value
\newcommand{\anovans}[3]{\mbox{$F_{#1,#2}~=~#3$}, \mbox{$p~>~.05$}} 

% reporting p-value with posthoc testds
\newcommand{\p}[1]{{$p~<~#1$}}
\newcommand{\pex}[1]{{$p~=~#1$}}


% One-Way Anova : DF, DFDen, F et p
\newcommand{\owanova}[3]{{\small $F_{#1}~=~#2$, $p~=~#3$}} 

\newcommand{\chis}[1]{{\small $\chi^2~=~#1$}} % chi square seul
\newcommand{\chisquare}[3]{{\small $\chi^2(#1)~=~#2$, $p~=~#3$}} % chi square classique : DF, DFDen, F et p
\newcommand{\chisquares}[2]{{\small $\chi^2(#1)~=~#2$, $p~<~0.0001$}} % chi square "s-ignificatif" : DF, DFDen, F, avec p < 0,0001



\newcommand{\cor}{correlation ($\rho$, p)}


% Shortened post-hoc results if necessary
\newcommand{\infS}[1]{$\overset{_{#1}}{<}$}
\newcommand{\supS}[1]{$\overset{_{#1}}{>}$}

