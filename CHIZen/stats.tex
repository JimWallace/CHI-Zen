\usepackage{etoolbox}
\usepackage{siunitx}

\sisetup{
    % range-phrase = --, % make ranges use en-dash instead of "to" (e.g., ages)
    round-mode = places,
    round-precision = 1, % can change precision globally
    round-half = even
}

\newcommand{\fprecision}{1}
\newcommand{\chisqprecision}{1}
\newcommand{\formatprob}[2]{%
    \ifdimless{#2 pt}{.001 pt}{$#1 < .001$}{%
        \ifdimless{#2 pt}{.01 pt}{$#1 < .01$}{%
            $#1 = \num[add-integer-zero=false,%
                      round-mode=places,%
                      round-precision=2,%
                      round-half=even]{#2}$%
        }%
    }%
}
\newcommand{\p}[1]{\formatprob{p}{#1}}
\newcommand{\etasqp}[1]{\formatprob{\eta_{p}^2}{#1}}
\newcommand{\F}[3]{{$F_{#1,#2}=\num[round-mode=places,%
                                    round-precision=\fprecision]{#3}$}}
\newcommand{\owF}[2]{{$F_{#1}=\num[round-mode=places,%
                                   round-precision=\fprecision]{#2}$}}
\newcommand{\chisq}[2]{$\chi^2(#1)=\num[round-mode=places,%
                                        round-precision=\chisqprecision]{#2}$}
\newcommand{\sd}[2][]{$SD=\SI{#2}{#1}$}
\newcommand{\mean}[2][]{$M=\SI{#2}{#1}$}
\newcommand{\mdn}[2][]{$Mdn=\SI{#2}{#1}$}
\newcommand{\ages}[2]{Ages \numrange{#1}{#2}}
\newcommand{\gender}[2]{\num{#1} identified as #2}
\newcommand{\genderlist}[2]{\numlist{#1} identified as #2, respectively}
\newcommand{\by}{\,$\times$\,}
\newcommand{\size}[3][]{\SIrange[range-phrase=\by,%
                                 range-units=single]{#2}{#3}{#1}}
\newcommand{\sizen}[2][]{\SIlist[list-separator=\by,%
                                 list-final-separator=\by,%
                                 list-pair-separator=\by,%
                                %  list-units=single% % repeat (default),single,brackets
                                 ]{#2}{#1}}
\newcommand{\msd}[3][]{\mean[#1]{#2}, \sd[#1]{#3}}
\newcommand{\anova}[4]{\F{#1}{#2}{#3}, \p{#4}}
\newcommand{\anovae}[5]{\F{#1}{#2}{#3}, \p{#4}, \etasqp{#5}}
\newcommand{\owanova}[4]{\owF{#1}{#2}, \p{#3}, \etasqp{#4}}
\newcommand{\chisquare}[3]{\chisq{#1}{#2}, \p{#3}}