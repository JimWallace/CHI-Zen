

\newpage
\vspace{2em}
\begin{center}
    \emph{If you define the problem correctly, you almost have the solution. --- Steve Jobs}
\end{center}

%%%%%%%%%%%%%%%%%%%%%%%%%%%%%%%%%%%%%%%%%%%%%%%%%%%%%%
\section{Tables}
%%%%%%%%%%%%%%%%%%%%%%%%%%%%%%%%%%%%%%%%%%%%%%%%%%%%%%
As a rule of thumb, if you can show your data with a table, consider doing that first. Tables are quite effective for showing real data, and usually easy to interpret. The Vis literature even says that tables might be more convincing for a skeptical audience \citep{pandey2014persuasive}!

In general, I suggest the following guidelines for formatting: 
\begin{enumerate}
    \item Remove vertical lines
    \item Only use horizontal lines where necessary
    \item Use \texttt{ \textbackslash toprule}, \texttt{ \textbackslash midrule}, and \texttt{ \textbackslash bottomrule} to frame your table
\end{enumerate}

You can see an example in \autoref{tab:exampletable}, below.

\begin{center}
\begin{table*}[htbp!]
\small
\begin{center}
\begin{tabular}{l l l l l}
\toprule
Condition & \multicolumn{4}{l}{Teamwork Measures}\\
\cmidrule(r){2-5}
& \multicolumn{2}{l}{Measure 1} & Measure 2 & Measure 3 \\
\cmidrule(r){2-3} 
&  Sub 1 & Sub 2 &  \\ 


\midrule


Condition 1 & .478 & .468 & 1.68 & 12.3  \\
                  & (.0259) & (.031) & (.703) & (5.93) \\
\\

Condition 2 & .444 & .451 & 1.32 & 18.4 \\
                              & (.0478) & (.0585) & (.319) & (2.76) \\
\\

Condition 3 &  & .410 &  2.857 &  \\
                     &        & (.0384)  & (.350) & \\
\\


ANOVA Results & $F_{(1,12)} = 2.27,$ & $F_{(2,18)} = 2.76,$  & $F_{(2,18)} = 16.16,$ & $F_{(1,12)} = 6.16,$ \\
                          & $p = .158$ & $p = .090  $ &  $p < .0001$* &  $p = 0.029$ * \\
\bottomrule
\end{tabular}

\caption[a]{Mean values and standard deviations (in parentheses) for teamwork measures, and ANOVA results for comparisons between experimental conditions. Significant results denoted by *.}
\label{tab:exampletable}
\end{center}
\end{table*}
\end{center}