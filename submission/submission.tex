%%
%% This is file `sample-acmsmall.tex',
%% generated with the docstrip utility.
%%
%% The original source files were:
%%
%% samples.dtx  (with options: `acmsmall')
%% 
%% IMPORTANT NOTICE:
%% 
%% For the copyright see the source file.
%% 
%% Any modified versions of this file must be renamed
%% with new filenames distinct from sample-acmsmall.tex.
%% 
%% For distribution of the original source see the terms
%% for copying and modification in the file samples.dtx.
%% 
%% This generated file may be distributed as long as the
%% original source files, as listed above, are part of the
%% same distribution. (The sources need not necessarily be
%% in the same archive or directory.)
%%
%% The first command in your LaTeX source must be the \documentclass command.
\documentclass[acmsmall]{acmart}

%Robert's accented character fix
\usepackage[utf8]{inputenc}
%appendix tables
\usepackage{longtable}
\usepackage{float}

% Jim's graphing stuff
\input{IDEA_Plots.tex}

\usepackage{CJKutf8}

% Jim added this -- useful highlighting function \hl{}
%\usepackage{soul,color}

% Mark likes to add these options to his papers .... worth considering
\frenchspacing
\raggedbottom


% Jim's quotation stuff
%\renewenvironment{quote}{%
%  \list{}{%
%    \leftmargin0.4cm   % this is the adjusting screw
%    \rightmargin\leftmargin
%  }
%  \item\relax
%}
%{\endlist}

%%
%% \BibTeX command to typeset BibTeX logo in the docs
\AtBeginDocument{%
  \providecommand\BibTeX{{%
    \normalfont B\kern-0.5em{\scshape i\kern-0.25em b}\kern-0.8em\TeX}}}

%% Rights management information.  This information is sent to you
%% when you complete the rights form.  These commands have SAMPLE
%% values in them; it is your responsibility as an author to replace
%% the commands and values with those provided to you when you
%% complete the rights form.
% \setcopyright{acmcopyright}
% \copyrightyear{2020}
% \acmYear{2020}
% \acmDOI{XX.XXXX/XXXXXXX.XXXXXXX}


%%
%% These commands are for a JOURNAL article.
% \acmJournal{PACMHCI}
% \acmVolume{0}
% \acmNumber{0}
% \acmArticle{000}
% \acmMonth{0}

%%
%% Submission ID.
%% Use this when submitting an article to a sponsored event. You'll
%% receive a unique submission ID from the organizers
%% of the event, and this ID should be used as the parameter to this command.
%%\acmSubmissionID{123-A56-BU3}

%%
%% The majority of ACM publications use numbered citations and
%% references.  The command \citestyle{authoryear} switches to the
%% "author year" style.
%%
%% If you are preparing content for an event
%% sponsored by ACM SIGGRAPH, you must use the "author year" style of
%% citations and references.
%% Uncommenting
%% the next command will enable that style.
%%\citestyle{acmauthoryear}


%%
%% end of the preamble, start of the body of the document source.
\begin{document}

%%
%% The "title" command has an optional parameter,
%% allowing the author to define a "short title" to be used in page headers.
\title[CHI Zen]{CHI Zen: Examples, Tips, and Hacks for Continuous Improvement of Presenting HCI Research}

%%
%% The "author" command and its associated commands are used to define
%% the authors and their affiliations.
%% Of note is the shared affiliation of the first two authors, and the
%% "authornote" and "authornotemark" commands
%% used to denote shared contribution to the research.
\author{Jim Wallace}
\affiliation{%
  \institution{University of Waterloo}
}
\email{james.wallace@uwaterloo.ca}


%\renewcommand{\shortauthors}{F. Author et al.}

%%
%% By default, the full list of authors will be used in the page
%% headers. Often, this list is too long, and will overlap
%% other information printed in the page headers. This command allows
%% the author to define a more concise list
%% of authors' names for this purpose.
%\renewcommand{\shortauthors}{Gauthier, Costello, & Wallace}



%%
%% The abstract is a short summary of the work to be presented in the
%% article.



\begin{abstract}
% \begin{figure}[H]
%     \centering
%     \includegraphics[width=.32\textwidth]{submission/figures/title1.pdf} 
%     \includegraphics[width=.32\textwidth]{submission/figures/title2.pdf} 
%     \includegraphics[width=.32\textwidth]{submission/figures/minotaur.pdf}
%     \caption{In \textit{Merlynne}, players unlock gameplay with CBT-based replies to negative Reddit posts in a fantasy role playing setting.}
%     \label{fig:teaser}
% \end{figure}

\end{abstract}

%%
%% The code below is generated by the tool at http://dl.acm.org/ccs.cfm.
%% Please copy and paste the code instead of the example below.
%%
% \begin{CCSXML}
% <ccs2012>
% <concept>
% <concept_id>10003120.10003130.10011762</concept_id>
% <concept_desc>Human-centered computing~Empirical studies in collaborative and social computing</concept_desc>
% <concept_significance>500</concept_significance>
% </concept>
% </ccs2012>
% \end{CCSXML}

% \ccsdesc[500]{Human-centered computing~Empirical studies in collaborative and social computing}

%%
%% Keywords. The author(s) should pick words that accurately describe
%% the work being presented. Separate the keywords with commas.
%\keywords{mental health; peer-to-peer; serious game; cognitive behavioural therapy; proteus effect;}


%%
%% This command processes the author and affiliation and title
%% information and builds the first part of the formatted document.
\maketitle



\vspace{2em}
\begin{quote}
    \emph{Great things are done by a series of small things brought together. --- Vincent Van Gough}
\end{quote}


%%%%%%%%%%%%%%%%%%%%%%%%%%%%%%%%%%%%%%%%%%%%%%%%%%%%%%%%%%%%%%%%%%%%%%%%%%%%%
\section{Introduction}
%%%%%%%%%%%%%%%%%%%%%%%%%%%%%%%%%%%%%%%%%%%%%%%%%%%%%%%%%%%%%%%%%%%%%%%%%%%%%

Even the most experienced researchers are constantly looking for ways to improve their papers. And there are many different aspects of a paper to improve on: from high-level concerns like experimental design, positioning the paper within the literature, and consideration of ethical issues, to sweating the details in a statistical analyis, or getting that graph \emph{just right} to show off your earth-shattering results. But sometimes it's hard to put all the pieces together. That's where this project comes in. 

This project --- CHI Zen, a play on the words `KaiZen' (\begin{CJK}{UTF8}{min}改善\end{CJK}) or `continuous improvement' --- describes common pitfalls and best practices based on my experience writing papers for a number of years. That experience reflects some of my own preferences, but also a lot that I've learned from working with other HCI researchers. It also contains a \emph{lot} of examples that you are free to copy, modify, or re-use as often as you'd like. The goal is to have something of an \emph{information zoo} \cite{heer2010tour} that can help you think about how to best visualize results, and hopefully make the process of creating that visualization easy, too. 

This document, also provides a sample reference for ``IDEA Plots,'' a set of templates that I developed for \texttt{PGFPlots}. Please use IDEA Plots if you find it useful. If you do, I'd love to hear about it, and if you have any ways to imrpove the templates. You might also prefer to use a different graphing package, and that's cool too. The examples contained in this document hopefully can provide tips or shortcuts for use in your own papers. I suspect that you can generate more or less the same graphs with your tool of choice. 

Finally, my intention in making this project available via GitHub is to allow it to grow. If you find it useful, please drop me a line. If you find a way to improve it, please push the changes to GitHub. It might also just be an important prompt to have a conversation about best practices, or an opportunity to explore different ways of thinking about or presenting research. In any case, I hope that it's an opportunity to think about how we can improve our research practices, one small step at a time. 



%%%%%%%%%%%%%%%%%%%%%%%%%%%%%%%%%%%%%%%%%%%%%%%%%%%%%%%%%%%%%%%%%%%%%%%%%%%%%
\section{\LaTeX\ and IDEA Plots}
%%%%%%%%%%%%%%%%%%%%%%%%%%%%%%%%%%%%%%%%%%%%%%%%%%%%%%%%%%%%%%%%%%%%%%%%%%%%%

Ok, so why \LaTeX?

\begin{enumerate}
    \item Most formatting is done `for free' by the SIG CHI template
    \item It's free, and the University of Waterloo provides a Pro Overleaf account
    \item Nice integration with other tools like R, PGFPLots, etc. 
    \item Papers get rejected all the time, and reformatting is very fast and easy
\end{enumerate}





\subsection{IDEA Plots}

- These are just some templates I developed that encapsulate a lot of the common tasks for creating scientific figures in HCI papers
- e.g., creating axes, labels, legends, 
- I also lean towards a minimalist/ Tuftian feel -- this is a good starting point, you can always make things more complex where needed


Pros:
\begin{enumerate}
    \item Draws the graphs directly into your PDF, no worry about scalability (look at these graphs zoomed in on a 5K display)
    \item Uses same fonts etc. as rest of document, so graphs feel like they 'fit'
    \item Lots of control over width/height, colours, every aspect of
    \item Can draw directly from a data file, so nice integration with analysis tools like Excel and R
\end{enumerate}


Cons:
\begin{enumerate}
    \item Can be complicated -- try to provide these templates to help mitigate that issue
    \item Sometimes it just won't work the way it's supposed to
    \item PGFPlots is updated from time to time, some underlying features may change/break
\end{enumerate}


\subsection{IDEA Plots Features at a Glance}

%%%%%%%%%%%%%%%%%%%%%%%%%%%%%%%%%%%%%%%%%%%%%%%%%%%%%%%%%%%%%%%%%%%%%%%%%%%%%
\section{\LaTeX\ Packages and Tools}
%%%%%%%%%%%%%%%%%%%%%%%%%%%%%%%%%%%%%%%%%%%%%%%%%%%%%%%%%%%%%%%%%%%%%%%%%%%%%

- Lots of different options here
- Many come down to preference

- have included quite a few here, as needs evolved over years
- some basic documentation



%%%%%%%%%%%%%%%%%%%%%%%%%%%%%%%%%%%%%%%%%%%%%%%%%%%%%%
\section{Tables}
%%%%%%%%%%%%%%%%%%%%%%%%%%%%%%%%%%%%%%%%%%%%%%%%%%%%%%

- if you can show it with a table, consider that first
- shows your real data, usually easy to interpret
- literature says it might even be more convincing for a skeptical audience! \cite{pandey2014persuasive}


My rules:

- remove vertical lines
- minimize horizontal lines

\begin{center}
\begin{table*}[htbp!]
\small
\begin{center}
\begin{tabular}{l l l l l}
\toprule
Condition & \multicolumn{4}{l}{Teamwork Measures}\\
\cmidrule(r){2-5}
& \multicolumn{2}{l}{Measure 1} & Measure 2 & Measure 3 \\
\cmidrule(r){2-3} 
&  Sub 1 & Sub 2 &  \\ 


\midrule


Condition 1 & .478 & .468 & 1.68 & 12.3  \\
                  & (.0259) & (.031) & (.703) & (5.93) \\
\\

Condition 2 & .444 & .451 & 1.32 & 18.4 \\
                              & (.0478) & (.0585) & (.319) & (2.76) \\
\\

Condition 3 & --- & .410 &  2.857 & --- \\
                     &        & (.0384)  & (.350) & \\
\\


ANOVA Results & $F_{(1,12)} = 2.27,$ & $F_{(2,18)} = 2.76,$  & $F_{(2,18)} = 16.16,$ & $F_{(1,12)} = 6.16,$ \\
                          & $p = .158$ & $p = .090  $ &  $p < .0001$* &  $p = 0.029$ * \\
\bottomrule
\end{tabular}

\caption[a]{Mean values and standard deviations (in parentheses) for teamwork measures, and ANOVA results for comparisons between experimental conditions. Significant results denoted by *.}
\label{tab:bpf_teamwork_results}
\end{center}
\end{table*}
\end{center}

\newpage



%%%%%%%%%%%%%%%%%%%%%%%%%%%%%%%%%%%%%%%%%%%%%%%%%%%%%%%%%%%%%%%%%%%%%%%%%%%%%
\section{Showing Your Data}
%%%%%%%%%%%%%%%%%%%%%%%%%%%%%%%%%%%%%%%%%%%%%%%%%%%%%%%%%%%%%%%%%%%%%%%%%%%%%


- Figures and Tables in your results section often provide reviewers a first look at your results
- can also make the often dense results section more accessible to readers who are unfamiliar with your research area
- for readers with specific experience, particularly in replicated experimental settings such as a Fitts's Law study, can be the most important summary of research results in the paper 


Some rules of thumb from Munzner \citep{munznervisualization}:
\begin{enumerate}
	\item No unjustified 3D
	\item No unjustified 2D
	\item Eyes beat memory
	\item Get it right in Black and White
	\item Function first, form next
\end{enumerate}


Munzner \citep{munznervisualization}, and in particular Chapter 5, is a useful resource for determining which Marks and Channels to use in a graph, and lists them in order of effectiveness

- the style suggested in this document is informed and inspired by Tufte \citep{tufte1983visual}, who emphasizes minimalist presentation of quantitative data. Tufte's principles of information visualization 

 
 
%  For example, some good examples:
% Bartneck and Hu \cite{Bartneck:2009:SAC:1518701.1518810}


% Senellart \cite{Senellart:2013:DYR:2513166.2514938} discusses some useful cases of `bad' style, which superficially may appear to be picky but offer important guidance towards submitting polished papers. 
 
 
% Munzer \cite{Munzner:2008:PPW:1422919.1422927} 
% - some good advice in general, but more focused on info vis research

% - text is large and legible
% - all axes are labelled, including units of measure
% - axes start at 0, unless explicitly justified (for a good reason!)













%%%%%%%%%%%%%%%%%%%%%%%%%%%%%%%%%%%%%%%%%%%%%%%%%%%%%%%%%%
\subsection{Comparing Performance: Bar Charts and Box Plots}
%%%%%%%%%%%%%%%%%%%%%%%%%%%%%%%%%%%%%%%%%%%%%%%%%%%%%%%%%%

- common to compare performance, efficiency, error rate, or other dependent measures between two study groups. For example, our new technique for 3D pointing, compared to a computer mouse. 

- most simple way of doing so is to show the means and standard error in a bar graph, and this is commonly accepted throughout the ACM. 
- For example, Figure \ref{bargraph} is an example drawn from a UIST publication \citep{REF}.


\begin{figure}
\begin{tikzpicture}
\begin{axis}[IDEA bar,
        symbolic x coords={non-occluded,occluded},
        xtick = {non-occluded,occluded},
        enlarge x limits=0.5,
	    ymin = 0, ymax = 10,
        ylabel = {Selection Time (s)},
        width=\columnwidth,
        height = 5cm,
]

%Depth Ray
\addplot[style={draw=none,fill=AHSLight},error bars/.cd, y dir = both, y explicit, error bar style={black}]
coordinates {(occluded, 6.8)  +- (1.7,1.7)};
\addlegendentry{Depth Cursor}

%Tiltcasting
\addplot+[style={draw=none, fill=SchoolRedLight,line width = 0pt},error bars/.cd, y dir = both, y explicit, error bar style={black}]
coordinates {(non-occluded,3.24) +- (.712,.712)
		    (occluded,4)  +- (.922,.922)};
\addlegendentry{Tiltcasting}

%Smartcasting
\addplot+[style={draw=none, fill=EnvironmentLight},error bars/.cd, y dir = both, y explicit, error bar style={black}]
coordinates {(non-occluded, 2.10)  +- (.42,.42)};
\addlegendentry{Smartcasting}

\end{axis}
\end{tikzpicture}
\caption{An example bar graph with error bars. Data is specified in the latex file.}
\label{bargraph}
\end{figure}


- to be more accurate and descriptive, box plots are often used
- similar to a bar graph with error bars, but also indicate quartiles and outliers 
- typically, better practice to include a box plot when possible -- but can be substantially more difficult to create in Microsoft Word. 




%%%%%%%%%%%%%%%%%%%%%%%%%%%%%%%%%%%%%%%%%%%%%%%%%%%%%%%%%%
\subsection{Questionnaire Responses}
%%%%%%%%%%%%%%%%%%%%%%%%%%%%%%%%%%%%%%%%%%%%%%%%%%%%%%%%%%

- Common to want to show questionnaire respones, particularly to, e.g., Likert scale questions
- Jim's preference: stacked divergent bar graph

\begin{figure}
\begin{tikzpicture}
\pgfplotstableread[col sep = comma]{submission/data/Cleric_AvatarIdentificationStacked.csv}\ClericIdentificationData
\pgfplotstableread[col sep = comma]{submission/data/Monster_AvatarIdentificationStacked.csv}\MonsterIdentificationData

\begin{axis}[
    IDEA Likert,
    height = 6.5cm,
    width = .75\columnwidth,
    % Y AXIS
    symbolic y coords = {Ava_Conn,	Ava_WE,	AvavsPhys,	AvaVsPers,	AvaVsPhysIdeal,	AvaVsPersIdeal},
    ytick={Ava_Conn, Ava_WE, AvavsPhys, AvaVsPers, AvaVsPhysIdeal, AvaVsPersIdeal},
    yticklabels={Connectedness, Refer to \\ Avatar as ``We'', Physical \\ Similarity, Personality\\ Similarity, Physical Ideal, Ideal Personality},
    yticklabel style={align=right,font={\sffamily\small}},
    y axis line style={draw=none},
    y dir = reverse,
    ymajorgrids = false,
    % X AXIS
    xmin=-100, xmax=100,
    xlabel={\% of Responses},
    xlabel style = {font=\small\sffamily},
    xticklabel style = {font=\small\sffamily},
    xtick = {-100,-50,0,50,100},
    xticklabels = {100, 50, 0, 50, 100},
    extra x ticks = {-75, 75},
    extra x tick labels = {$\leftarrow$ Disagree, Agree $\rightarrow$},
    every extra x tick/.style={major tick length=0,yshift={-8pt}},
    ]
    \addplot[draw=none,fill=none, forget plot] coordinates {(0,Ava_Conn)(0,Ava_WE)(0,AvavsPhys)(0,AvaVsPers)(0,AvaVsPhysIdeal)(0,AvaVsPersIdeal)};
    \addplot[draw=none,fill=Cleric_3, forget plot, bar shift=.1cm] table [x expr={\thisrow{4}*-0.5},y=Measure] {\ClericIdentificationData};
    \addplot[draw=none,fill=Cleric_2, forget plot, bar shift=.1cm] table [x expr={\thisrow{3}*-1},y=Measure] {\ClericIdentificationData};
    \addplot[draw=none,fill=Cleric_1, forget plot, bar shift=.1cm] table [x expr={\thisrow{2}*-1}, y=Measure ] {\ClericIdentificationData};
    \addplot[draw=none,fill=Cleric_0, forget plot, bar shift=.1cm] table [x expr={\thisrow{1}*-1}, y=Measure ] {\ClericIdentificationData};
    \resetstackedplots
    \addplot[draw=none,fill=none, forget plot] coordinates {(0,Ava_Conn)(0,Ava_WE)(0,AvavsPhys)(0,AvaVsPers)(0,AvaVsPhysIdeal)(0,AvaVsPersIdeal)};
    \addplot[draw=none,fill=Monster_3, forget plot, bar shift=-.1cm] table [x expr={\thisrow{4}*-0.5},y=Measure] {\MonsterIdentificationData};
    \addplot[draw=none,fill=Monster_2, forget plot, bar shift=-.1cm] table [x expr={\thisrow{3}*-1},y=Measure] {\MonsterIdentificationData};
    \addplot[draw=none,fill=Monster_1, forget plot, bar shift=-.1cm] table [x expr={\thisrow{2}*-1}, y=Measure ] {\MonsterIdentificationData};
    \addplot[draw=none,fill=Monster_0, forget plot, bar shift=-.1cm] table [x expr={\thisrow{1}*-1}, y=Measure ] {\MonsterIdentificationData};
\end{axis}

\begin{axis}[
    IDEA Likert,
    height = 6.5cm,
    width = .75\columnwidth,
    % Y AXIS
    symbolic y coords = {Ava_Conn,	Ava_WE,	AvavsPhys,	AvaVsPers,	AvaVsPhysIdeal,	AvaVsPersIdeal},
    ytick={Ava_Conn, Ava_WE, AvavsPhys, AvaVsPers, AvaVsPhysIdeal, AvaVsPersIdeal},
    yticklabels={Connectedness, Refer to \\ Avatar as ``We'', Physical \\ Similarity, Personality\\ Similarity, Physical Ideal, Ideal Personality},
    yticklabel style={align=right,font={\sffamily\small}},
    y axis line style={draw=none},
    y dir = reverse,
    ymajorgrids = false,
    % X AXIS
    xmin=-100, xmax=100,
    xlabel={\% of Responses},
    xlabel style = {font=\small\sffamily},
    xticklabel style = {font=\small\sffamily},
    xtick = {-100,-50,0,50,100},
    xticklabels = {100, 50, 0, 50, 100},
    extra x ticks = {-75, 75},
    extra x tick labels = {$\leftarrow$ Disagree, Agree $\rightarrow$ },
    every extra x tick/.style={major tick length=0,yshift={-8pt}},
    ]
    \addplot[draw=none,fill=none, forget plot] coordinates {(0,Ava_Conn)(0,Ava_WE)(0,AvavsPhys)(0,AvaVsPers)(0,AvaVsPhysIdeal)(0,AvaVsPersIdeal)};
    
    \addplot[draw=none,fill=Cleric_3, forget plot, bar shift=.1cm] table [x expr={\thisrow{4}*0.5}, y=Measure ] {\ClericIdentificationData};
    \addplot[draw=none,fill=Cleric_4, forget plot, bar shift=.1cm] table [x expr={\thisrow{5}}, y=Measure ] {\ClericIdentificationData};
    \addplot[draw=none,fill=Cleric_5, forget plot, bar shift=.1cm] table [x expr={\thisrow{6}}, y=Measure ] {\ClericIdentificationData};
    \addplot[draw=none,fill=Cleric_6, forget plot, bar shift=.1cm] table [x expr={\thisrow{7}}, y=Measure ] {\ClericIdentificationData}; \label{plot:cleric6}
    \resetstackedplots
    \addplot[draw=none,fill=none, forget plot] coordinates {(0,Ava_Conn)(0,Ava_WE)(0,AvavsPhys)(0,AvaVsPers)(0,AvaVsPhysIdeal)(0,AvaVsPersIdeal)};
    \addplot[draw=none,fill=Monster_3, forget plot, bar shift=-.1cm] table [x expr={\thisrow{4}*0.5}, y=Measure ] {\MonsterIdentificationData};
    \addplot[draw=none,fill=Monster_4, forget plot, bar shift=-.1cm] table [x expr={\thisrow{5}}, y=Measure ] {\MonsterIdentificationData};
    \addplot[draw=none,fill=Monster_5, forget plot, bar shift=-.1cm] table [x expr={\thisrow{6}}, y=Measure ] {\MonsterIdentificationData};
    \addplot[draw=none,fill=Monster_6, forget plot, bar shift=-.1cm] table [x expr={\thisrow{7}}, y=Measure ] {\MonsterIdentificationData};
    %\addlegendimage{only marks, mark=o}
    %\addlegendimage{only marks, mark=o}
    %\legend{Neutral, Agree, Strongly Agree, Disagree,Strongly Disagree}
    after end axis/.code={
        \node at (axis cs:80,Ava_Conn) [anchor=east, ,font=\tiny] {\AsteriskBold};
        \node at (axis cs:80,Ava_WE) [anchor=east, font=\tiny] {\AsteriskBold};
        %\draw [white] (axis cs:0,AvaVsPersIdeal) -- (axis cs:0,Ava_Conn);
        %\draw [white] (axis cs:-50,AvaVsPersIdeal) -- (axis cs:-50,Ava_Conn);
        %\draw [white] (axis cs:50,AvaVsPersIdeal) -- (axis cs:50,Ava_Conn);
    }
    \coordinate (legend) at (axis description cs:1.15,.85);
\end{axis}

% this is a dummy `axis' environment only to create the legend
\matrix [
    matrix of nodes, 
    every node/.style={anchor=center}, 
    ] at (legend) {
        |[fill=Cleric_0]| & |[fill=Cleric_1]| & |[fill=Cleric_2]| & |[fill=Cleric_3]| & |[fill=Cleric_4]| & |[fill=Cleric_5]| & |[fill=Cleric_6]| & |[font=\small\sffamily]|Cleric \\
        |[fill=Monster_0]| & |[fill=Monster_1]| & |[fill=Monster_2]| & |[fill=Monster_3]| & |[fill=Monster_4]| & |[fill=Monster_5]| & |[fill=Monster_6]| & |[font=\small\sffamily]|Monster \\
    };
\end{tikzpicture}
    
\caption{Summary of avatar identification responses, based on 7-point Likert scales. Mann Whitney U tests revealed significant differences between \textsc{Avatar} groups for `Connectness' and `Refer to Avatar as We'. }
\label{fig:avatar_identification}
\end{figure}




%%%%%%%%%%%%%%%%%%%%%%%%%%%%%%%%%%%%%%%%%%%%%%%%%%%%%%%%%%
\subsubsection{Showing Trends over Time}
%%%%%%%%%%%%%%%%%%%%%%%%%%%%%%%%%%%%%%%%%%%%%%%%%%%%%%%%%%



\begin{figure}[t!]
\begin{tikzpicture}[baseline]
\begin{axis}[IDEA line,
	% Graph Options
	width=.475*\columnwidth,
	height=4cm,
	title=\texttt{r/stop\-drinking},
	% X Axis Options
	%date coordinates in=x,
	%date ZERO=2014-01-01,
    %xticklabel=\year,
    %xmin=2014-01-01,
    %xmax=2018-01-01,
    xlabel=Year,
	xtick={1,13,25,37,49},                              %% Jim: apologies... this is a hack
    xticklabels={2014,2015,2016,2017,2018},             %%      will figure out what's going on here if we have time later
	%xtick ={1,2,3,4,5,6,7,8,9},
	% Y Axis Options
	ylabel = Distinct Users / Active Threads,
	ylabel style = {font=\tiny},
	ymin = 0, ymax = 6000,
	ytick = {0,2000,4000,6000},
	ytick style = {font=\tiny},
    %legend image code/.code={\draw[#1] circle (0.1cm);}
]

\pgfplotstableread[col sep = comma]{submission/data/stopdrinking-usagebymonth.csv}\mydata;
\addplot+[draw=ScienceDark,fill=none,thick] table[x=index,y=users] {\mydata};
\addplot+[draw=ArtsLight,fill=none,thick] table[x=index,y=threads] {\mydata};

\end{axis}
\end{tikzpicture}%
%
\begin{tikzpicture}[baseline]
\begin{axis}[IDEA line,
	% Graph Options
	width=.475*\columnwidth,	
	height=4cm,
	title=\texttt{r/Opiates\-Recovery},
	% X Axis Options
	xtick={1,13,25,37,49},                              %% Jim: apologies... this is a hack
    xticklabels={2014,2015,2016,2017,2018},             %%      will figure out what's going on here if we have time later
	%date coordinates in=x,
	%date ZERO=2014-01-01,
    %xticklabel=\year,
    %xmin=2014-01-01,
    %xmax=2018-01-01,
	xlabel = Year,
	%xtick ={1,2,3,4,5,6,7,8,9},
	% Y Axis Options
	ylabel = Distinct Users / Active Threads,
	ylabel style = {font=\tiny},
	ymin = 0, ymax = 800,
	ytick = {0,200,400,600,800},
	ytick style = {font=\tiny},
	legend style={at={(0.5,-1)},anchor=south}
    %legend image code/.code={\draw[#1] circle (0.1cm);}
]

\pgfplotstableread[col sep = comma]{submission/data/opiaterecovery-usagebymonth.csv}\mydata;
\addplot+[draw=ScienceDark,fill=none,thick] table[x=index,y=users] {\mydata};
\addlegendentry{Distinct Users}
\addplot+[draw=ArtsLight,fill=none,thick] table[x=index,y=threads] {\mydata};
\addlegendentry{Active Threads}

\end{axis}
\end{tikzpicture}
%\setlength{\belowcaptionskip}{-20pt}
\caption{Distinct users and active threads for \texttt{r/stop\-drinking} (top) and \texttt{r/Opiates\-Recovery} (bottom) showing that both counts are experiencing an upwards trend over time for both subreddits.} 

\label{fig:activity}
\end{figure}



\begin{figure*}[tb!]
\begin{center}
\begin{tikzpicture}
\begin{groupplot}[
   group style={
       group size=1 by 5,
       x descriptions at=edge bottom,
       y descriptions at=edge left,
       vertical sep=1pt,
       horizontal sep=0pt},
  IDEA tufte panel, 
	% X Axis
	width=\columnwidth,
	xmin = 1990, xmax = 2015,
	xtick = {1990,1994,1998,2002,2006,2010,2015},
	x tick label style={font=\small,rotate=90, /pgf/number format/1000 sep=},
	axis line style={thick},
	% Y Axis
	ytick={10,20,30,40,50,60,70,80,90},
        ymin = 0,
        ymax = 50,
      ]


% Bibliographic
\nextgroupplot[bar width=7pt, height = 2.25cm, ymax = 25, title={Bibliographic}]
\addplot+[draw=none, fill=black!50] 
 coordinates {(1990,20) (1992,4) (1994,0) (1996,0) (1998,2) (2000,0) (2002,0) (2004,2) (2006,2) (2008,1) (2010,0) (2011,0) (2012,0) (2013,1) (2014,0) (2015,1) };       
  
% Not Empirical      
\nextgroupplot[bar width=7pt, height = 3.5cm, ymax=70, title={Not Empirical}]
\addplot+[draw=none, fill=black!50] 
coordinates {(1990,37) (1992,55) (1994,33) (1996,38) (1998,54) (2000,35) (2002,20) (2004,18) (2006,24) (2008,13) (2010,5) (2011,31) (2012,9) (2013,4) (2014,6) (2015,7) };

% Explanatory
\nextgroupplot[bar width=7pt, height = 3cm, title={Explanatory}]
\addplot+[draw=none, fill=black!50]  
coordinates {(1990,3) (1992,4) (1994,5) (1996,0) (1998,0) (2000,5) (2002,15) (2004,16) (2006,29) (2008,25) (2010,21) (2011,16) (2012,8) (2013,14) (2014,15) (2015,12) };

% D & E
\nextgroupplot[bar width=7pt, ymax=60, height = 3.5cm, title={Design and Evaluation}]
\addplot+[draw=none, fill=blue!50] 
coordinates {(1990,17) (1992,13) (1994,26) (1996,22) (1998,15) (2000,27) (2002,39) (2004,35) (2006,23) (2008,19) (2010,21) (2011,10) (2012,28) (2013,26) (2014,24) (2015,22) };

% Descriptive
\nextgroupplot[bar width=7pt, ymax = 70, height = 3.5cm, title={Descriptive}]
\addplot+[draw=none,fill=black!50]
coordinates {(1990,23) (1992,23) (1994,36) (1996, 40) (1998,29) (2000,32) (2002,27) (2004,29) (2006,23) (2008,42) (2010,53) (2011,42) (2012,55) (2013,55) (2014,55) (2015,58) };


\end{groupplot}
\end{tikzpicture}
\caption{An example of Tufte's stacked panel graph, which presents the same information as a stacked bar graph but better enables comparisons across each vertical dimension. Best practice is to use black and white when possible, with sparing colour for emphasis where appropriate. }
\label{stackedpanel}
\end{center}
\end{figure*}




\begin{figure}
\begin{tikzpicture}
\begin{groupplot}[
   group style={
       group size=1 by 5,
       x descriptions at=edge bottom,
       y descriptions at=edge left,
       vertical sep=10pt,
       horizontal sep=0pt},
  IDEA line,
	% Graph Options
	height=3cm,
	 separate axis lines,
	% X Axis Options
	xmin=0, xmax= 35,
	xtick={1,5,10,15,20,25,30, 35},
	%date coordinates in=x,
	%date ZERO=2014-01-01,
    %xticklabel=\year,
    %xmin=2014-01-01,
    %xmax=2018-01-01,
	xlabel = Number of NPC Requests,
	xlabel style = {font=\small\sffamily},
    xticklabel style = {font=\small\sffamily},
	%xtick ={1,2,3,4,5,6,7,8,9},
	% Y Axis Options
	%ylabel = Proportion of Responses,
	ymajorgrids = false,
	ylabel shift = -.3cm,
	ymin = 0, ymax = 1,
	ytick = {0, 1.00},
	ylabel style = {font=\small\sffamily},
    yticklabel style = {font=\small\sffamily},
	%ytick style = {font=\tiny},
	title style = {yshift=-.75cm,font=\sffamily\small}
]
\pgfplotstableread[col sep = comma]{submission/data/CBT_PropOverTime.csv}\cbtovertimedata;

\nextgroupplot[xtick style = {white}, title style = {yshift=+.35cm,font=\sffamily\small}, title={Not Damaging}]
\addplot+[mark=*, mark size=1.5pt, draw=gray, only marks,fill=gray,thin] table[x=Query,y=NotDamaging] {\cbtovertimedata};
\addplot [solid, draw=black, fill=none,ultra thick] table[y={create col/linear regression={y=NotDamaging}}] {\cbtovertimedata};
%\addlegendentry{Not Damaging}

\nextgroupplot[xtick style = {white}, title={Empathize}]
\addplot+[mark=*, mark size=1.5pt, draw=gray, only marks,fill=gray,thin] table[x=Query,y=Empathize] {\cbtovertimedata};
\addplot [solid, draw=black, fill=none,ultra thick] table[y={create col/linear regression={y=Empathize}}] {\cbtovertimedata};
%\addlegendentry{Empathize}

\nextgroupplot[xtick style = {white}, title style = {yshift=+.25cm,font=\sffamily\small}, title={Reframe}]
\addplot+[mark=*, mark size=1.5pt, draw=gray, only marks,fill=gray,thin] table[x=Query,y=Reframe] {\cbtovertimedata};
\addplot [solid, draw=black, fill=none,ultra thick] table[y={create col/linear regression={y=Reframe}}] {\cbtovertimedata};
%\addlegendentry{Reframe}

\nextgroupplot[xtick style = {white}, title={Encourage}]
\addplot+[mark=*, mark size=1.5pt, draw=gray, only marks,fill=gray,thin] table[x=Query,y=Encourage] {\cbtovertimedata};
\addplot [solid, draw=black, fill=none,ultra thick] table[y={create col/linear regression={y=Encourage}}] {\cbtovertimedata};
%\addlegendentry{Encourage}

\nextgroupplot[title={Solution}]
\addplot+[mark=*, mark size=1.5pt, draw=gray, only marks,fill=gray,thin] table[x=Query,y=Solution] {\cbtovertimedata};
\addplot [solid, draw=black, fill=none,ultra thick] table[y={create col/linear regression={y=Solution}}] {\cbtovertimedata};
%\addlegendentry{Solution}


%\addplot+[mark=x, only marks,draw=Monster_0,fill=none,semithick] table[x=Query,y=Proportion, y error plus=CI-upper, y error minus=CI-lower] {\barbariansolutiondata};
%\addlegendentry{Monster}
%\addplot [draw=Monster_0, fill=none,ultra thick] table[y={create col/linear regression={y=Proportion}}] {\barbariansolutiondata};

\end{groupplot}
\end{tikzpicture}

%\setlength{\belowcaptionskip}{-20pt}
\caption{The proportion of responses that were classified as meeting five criteria: Empathize, Reframe, Encourage, Solution, and Not Damaging. The three categories related to CBT, Empathize, Reframe, and Encourage, were found to decrease over time ($p=0.05$). However, we did not find these changes for Solution and Not Damaging. 
%decreased over time for all participants, however responses by participants in the \textsc{Cleric} group decreased more quickly than those in the \textsc{Monster} group.
} 

\label{fig:SolutionsOverQueryNumber}
\end{figure}
























\input{submission/06-discussion.tex}

\input{submission/07-limitations.tex}

\input{submission/08-conclusion.tex}


%%%%%%%%%%%%%%%%%%%%%%%%%%%%%%%%%%%%%%%%%%%%%%%%%%%%%%%%%%%%%%%%%%%%%%%%%%%%%
%\section{Acknowledgements}
%%%%%%%%%%%%%%%%%%%%%%%%%%%%%%%%%%%%%%%%%%%%%%%%%%%%%%%%%%%%%%%%%%%%%%%%%%%%%


\begin{markdown}

# Acknowledgements

This template is a compilation of common mistakes, reviewer feedback, and paper writing experience. It is intended to help identify the `little things' that often get in the way of a successful CHI submission, and may be helpful in reviewing your paper \textit{well before} the deadline. For example, to guide peer reviews, such as reviewing circles, or to help communicate necessary revisions. 

While I compiled the different examples, tips, and hacks in one place, they incorporate the experience, advice, and feedback from many others I've worked with, including (in alphabetical order): Robert Gauthier, Mark Hancock, Lennart Nacke, Adrian Reetz, Stacey Scott, and Dan Vogel. Suggestions are always welcome, please email to \href{mailto:james.wallace@uwaterloo.ca}{james.wallace@uwaterloo.ca}. 

\end{markdown}

%\bibliographystyle{SIGCHI-Reference-Format}
\bibliographystyle{ACM-Reference-Format}
\bibliography{submission/bibliography.bib}

%\input{submission/11-appendix.tex}

\end{document}
\endinput
%%
%% End of file.
