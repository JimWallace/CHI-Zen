%%%%%%%%%%%%%%%%%%%%%%%%%%%%%%%%%%%%%%%%%%%%%%%%%
%  HOW TO USE
%%%%%%%%%%%%%%%%%%%%%%%%%%%%%%%%%%%%%%%%%%%%%%%%%
%
% Include this line at the top of your TeX  File: 
%
% %%%%%%%%%%%%%%%%%%%%%%%%%%%%%%%%%%%%%%%%%%%%%%%%%
%  HOW TO USE
%%%%%%%%%%%%%%%%%%%%%%%%%%%%%%%%%%%%%%%%%%%%%%%%%
%
% Include this line at the top of your TeX  File: 
%
% %%%%%%%%%%%%%%%%%%%%%%%%%%%%%%%%%%%%%%%%%%%%%%%%%
%  HOW TO USE
%%%%%%%%%%%%%%%%%%%%%%%%%%%%%%%%%%%%%%%%%%%%%%%%%
%
% Include this line at the top of your TeX  File: 
%
% %%%%%%%%%%%%%%%%%%%%%%%%%%%%%%%%%%%%%%%%%%%%%%%%%
%  HOW TO USE
%%%%%%%%%%%%%%%%%%%%%%%%%%%%%%%%%%%%%%%%%%%%%%%%%
%
% Include this line at the top of your TeX  File: 
%
% \input{IDEA_Plots.tex}
%
%
% Then, add graphs to your paper using the following template: 
%
%\begin{tikzpicture}
%\begin{axis}[<GRAPH TYPE>,
%	% X Axis options (xmin, xmax, xtick={})
%	% Y Axis options (ymin, ymax, ytick={})
%       ]
%
%    % DATA GOES IN HERE
%
%\end{axis}
%\end{tikzpicture}
%
%
% OR, for multiple plots use this template:
%
%\begin{tikzpicture}
%\begin{groupplot}[
%   group style={
%       group size= <COL> by <ROW>,
%       x descriptions at=edge bottom,
%       y descriptions at=edge left,
%       vertical sep=0pt,
%       horizontal sep=0pt},
%  <GRAPH TYPE>, 
%	% X Axis options (xmin, xmax, xtick={})
%	% Y Axis options (ymin, ymax, ytick={})
%      ]
%
%    % DATA GOES IN HERE
%
%\end{groupplot}
%\end{tikzpicture}
%
%
%
% The following graph types are defined in this library:
%
% IDEA bar - A clustered bar graph, similar to Excel
% IDEA tufte panel - a bare-bones bar graph, suitable for stacked panel graphs in Tufte's style
% .. more that aren't listed here
%




%%%%%%%%%%%%%%%%%%%%%%%%%%%%%%%%%%%%%%%%%%%%%%%%%
%   Common Includes
%%%%%%%%%%%%%%%%%%%%%%%%%%%%%%%%%%%%%%%%%%%%%%%%%
\usepackage{pgfplots}
\usepackage{pgfplotstable}
\usepgfplotslibrary{dateplot}
\usepgfplotslibrary{groupplots}


%\usetikzlibrary{external}
%\tikzexternalize[prefix=tikz/] % Must be in main .tex file?

\usetikzlibrary{pgfplots.statistics}
\usetikzlibrary{matrix}

\usepackage{bbding}

\pgfplotsset{compat=newest}

%%%%%%%%%%%%%%%%%%%%%%%%%%%%%%%%%%%%%%%%%%%%%%%%%
%   Colour Definitions
%%%%%%%%%%%%%%%%%%%%%%%%%%%%%%%%%%%%%%%%%%%%%%%%%

\definecolor{RYB1}{RGB}{141, 211, 199}
\definecolor{RYB2}{RGB}{255, 255, 179}
\definecolor{RYB3}{RGB}{190, 186, 218}
\definecolor{RYB4}{RGB}{251, 128, 114}
\definecolor{RYB5}{RGB}{128, 177, 211}
\definecolor{RYB6}{RGB}{253, 180, 98}
\definecolor{RYB7}{RGB}{179, 222, 105}

\pgfplotscreateplotcyclelist{colorbrewer-RYB}{
{RYB1!50!black,fill=RYB1},
{RYB2!50!black,fill=RYB2},
{RYB3!50!black,fill=RYB3},
{RYB4!50!black,fill=RYB4},
{RYB5!50!black,fill=RYB5},
{RYB6!50!black,fill=RYB6},
{RYB7!50!black,fill=RYB7},
}

\pgfplotscreateplotcyclelist{colorbrewer-RYB-plain}{
{RYB1},
{RYB2},
{RYB3},
{RYB4},
{RYB5},
{RYB6},
{RYB7},
}

%%%%%%%%%%%%%%%%%%%%
% WATERLOO PALETTES - DIGITAL
%%%%%%%%%%%%%%%%%%%%

\definecolor{AHSLight}{RGB}{0, 154, 166}
\definecolor{AHSDark}{RGB}{0, 127, 138}

\definecolor{ArtsLight}{RGB}{233, 131, 0}
\definecolor{ArtsDark}{RGB}{172, 97, 0}

\definecolor{EngineeringLight}{RGB}{204, 170, 255}
\definecolor{EngineeringDark}{RGB}{87, 6, 140}

\definecolor{EnvironmentLight}{RGB}{182, 191, 0}
\definecolor{EnvironmentDark}{RGB}{116, 120, 0}

\definecolor{MathLight}{RGB}{255, 136, 221}
\definecolor{MathDark}{RGB}{224, 36, 154}
 
\definecolor{ScienceLight}{RGB}{119, 187, 225}
\definecolor{ScienceDark}{RGB}{0, 115, 207}


\definecolor{SchoolRedLight}{RGB}{247, 119, 119}
\definecolor{SchoolRedDark}{RGB}{150, 23, 46}



\pgfplotscreateplotcyclelist{waterloo-light}{
{AHSLight!50!black,fill=AHSLight},
{ArtsLight!50!black,fill=ArtsLight},
{EngineeringLight!50!black,fill=EngineeringLight},
{EnvironmentLight!50!black,fill=EnvironmentLight},
{MathLight!50!black,fill=MathLight},
{ScienceLight!50!black,fill=ScienceLight},
{SchoolRedLight!50!black,fill=SchoolRedLight},
}

\pgfplotscreateplotcyclelist{waterloo-dark}{
{AHSDark!50!black,fill=AHSDark},
{ArtsDark!50!black,fill=ArtsDark},
{EngineeringDark!50!black,fill=EngineeringDark},
{EnvironmentDark!50!black,fill=EnvironmentDark},
{MathDark!50!black,fill=MathDark},
{ScienceDark!50!black,fill=ScienceDark},
{SchoolRedDark!50!black,fill=SchoolRedDark},
}



%%%%%%%%%%%%%%%%%%%%
% WATERLOO  PALETTES - PRINT
%%%%%%%%%%%%%%%%%%%%

% FIX: CMYK is conflicting with other colour definitions in SIGCHI template

%\definecolor{AHSPrint}{CMYK}{100, 0, 30, 2}
%\definecolor{ArtsPrint}{CMYK}{0, 52, 100, 0}
%\definecolor{EngineeringPrint}{CMYK}{24, 0, 98, 8}
%\definecolor{EnvironmentPrint}{CMYK}{78, 94, 0, 0}
%\definecolor{MathPrint}{CMYK}{5, 90, 0, 0}
%\definecolor{SciencePrint}{CMYK}{90, 48, 0, 0}
%\definecolor{SchoolRedPrint}{CMYK}{3, 100, 66, 12}

%\pgfplotscreateplotcyclelist{waterloo-print}{
%{AHSPrint!50!black,fill=AHSPrint},
%{ArtsPrint!50!black,fill=ArtsPrint},
%{EngineeringPrint!50!black,fill=EngineeringPrint},
%{EnvironmentPrint!50!black,fill=EnvironmentPrint},
%{MathPrint!50!black,fill=MathPrint},
%{SciencePrint!50!black,fill=SciencePrint},
%{SchoolRedPrint!50!black,fill=SchoolRedPrint},
%}




%%%%%%%%%%%%%%%%%%%%
% 5-Point DIVERGENT PALETTES - From ColorBrewer2
%%%%%%%%%%%%%%%%%%%%

\definecolor{Likert5_SD}{RGB}{166,97,26}
\definecolor{Likert5_D}{RGB}{223,194,125}
\definecolor{Likert5_N}{RGB}{245,245,245}
\definecolor{Likert5_A}{RGB}{128,205,193}
\definecolor{Likert5_SA}{RGB}{1,133,113}

\definecolor{Likert5_1_SD}{RGB}{166,97,26}
\definecolor{Likert5_1_D}{RGB}{223,194,125}
\definecolor{Likert5_1_N}{RGB}{245,245,245}
\definecolor{Likert5_1_A}{RGB}{128,205,193}
\definecolor{Likert5_1_SA}{RGB}{1,133,113}

\definecolor{Likert5_2_SD}{RGB}{123,50,148}
\definecolor{Likert5_2_D}{RGB}{194,165,207}
\definecolor{Likert5_2_N}{RGB}{247,247,247}
\definecolor{Likert5_2_A}{RGB}{166,219,160}
\definecolor{Likert5_2_SA}{RGB}{0,136,55}



%%%%%%%%%%%%%%%%%%%%
% 7-Point DIVERGENT PALETTES - From ColorBrewer2
%%%%%%%%%%%%%%%%%%%%

% Purple -> Green
\definecolor{Likert7_0}{RGB}{118,42,131}
\definecolor{Likert7_1}{RGB}{175,141,195}
\definecolor{Likert7_2}{RGB}{231,212,232}
\definecolor{Likert7_3}{RGB}{229,229,229}
\definecolor{Likert7_4}{RGB}{217,240,211}
\definecolor{Likert7_5}{RGB}{127,191,123}
\definecolor{Likert7_6}{RGB}{27,120,55}

% Orange/Brown -> Blue
\definecolor{Likert7_2_0}{RGB}{140,81,10}
\definecolor{Likert7_2_1}{RGB}{216,179,101}
\definecolor{Likert7_2_2}{RGB}{246,232,195}
\definecolor{Likert7_2_3}{RGB}{229,229,229}
\definecolor{Likert7_2_4}{RGB}{199,234,229}
\definecolor{Likert7_2_5}{RGB}{90,180,172}
\definecolor{Likert7_2_6}{RGB}{1,102,94}

% Orange -> Purple
\definecolor{Likert7_3_0}{RGB}{179,88,6}
\definecolor{Likert7_3_1}{RGB}{241,163,64}
\definecolor{Likert7_3_2}{RGB}{254,224,182}
\definecolor{Likert7_3_3}{RGB}{229,229,229}
\definecolor{Likert7_3_4}{RGB}{216,218,235}
\definecolor{Likert7_3_5}{RGB}{153,142,195}
\definecolor{Likert7_3_6}{RGB}{84,39,136}

% Blue -> Red
\definecolor{Likert7_4_0}{RGB}{33,102,172}
\definecolor{Likert7_4_1}{RGB}{103,169,207}
\definecolor{Likert7_4_2}{RGB}{209,229,240}
\definecolor{Likert7_4_3}{RGB}{229,229,229}
\definecolor{Likert7_4_4}{RGB}{253,219,199}
\definecolor{Likert7_4_5}{RGB}{239,138,98}
\definecolor{Likert7_4_6}{RGB}{178,24,43}


%%%%%%%%%%%%%%%%%%%%
% 7-Point DIVERGENT PALETTES - Jim for Avatar Paper
%%%%%%%%%%%%%%%%%%%%

% Blue -> Green to match Cleric
\definecolor{Cleric_0}{RGB}{33,102,172}
\definecolor{Cleric_1}{RGB}{103,169,207}
\definecolor{Cleric_2}{RGB}{209,229,240}
\definecolor{Cleric_3}{RGB}{229,229,229}
\definecolor{Cleric_4}{RGB}{217,240,211}
\definecolor{Cleric_5}{RGB}{127,191,123}
\definecolor{Cleric_6}{RGB}{27,120,55}

% Orange -> Red
\definecolor{Monster_0}{RGB}{178,24,43}
\definecolor{Monster_1}{RGB}{239,138,98}
\definecolor{Monster_2}{RGB}{253,219,199}
\definecolor{Monster_3}{RGB}{229,229,229}
\definecolor{Monster_4}{RGB}{216,218,235}
\definecolor{Monster_5}{RGB}{153,142,195}
\definecolor{Monster_6}{RGB}{84,39,136}


\pgfplotscreateplotcyclelist{avatar}{Cleric_0, Monster_0}

%%%%%%%%%%%%%%%%%%%%%%%%%%%%%%%%%%%%%%%%%%%%%%%%%
%   Outer South Legend Hack
%%%%%%%%%%%%%%%%%%%%%%%%%%%%%%%%%%%%%%%%%%%%%%%%%
\makeatletter
\pgfplotsset{
    every axis x label/.append style={
        alias=current axis xlabel
    },
    legend pos/outer south/.style={
        /pgfplots/legend style={
            at={%
                (%
                \@ifundefined{pgf@sh@ns@current axis xlabel}%
                {xticklabel cs:0.5}%
                {current axis xlabel.south}%
                )%
            },
            anchor=north
        }
    }
}

%%%%%%%%%%%%%%%%%%%%%%%%%%%%%%%%%%%%%%%%%%%%%%%%%
%   Grid Figures Label Hack
%%%%%%%%%%%%%%%%%%%%%%%%%%%%%%%%%%%%%%%%%%%%%%%%%

\pgfplotsset{
    groupplot xlabel/.initial={},
    every groupplot x label/.style={
        at={($({\pgfplots@group@name\space c1r\pgfplots@group@rows.west}|-{\pgfplots@group@name\space c1r\pgfplots@group@rows.outer south})!0.5!({\pgfplots@group@name\space c\pgfplots@group@columns r\pgfplots@group@rows.east}|-{\pgfplots@group@name\space c\pgfplots@group@columns r\pgfplots@group@rows.outer south})$)},
        anchor=north,
    },
    groupplot ylabel/.initial={},
    every groupplot y label/.style={
            rotate=90,
        at={($({\pgfplots@group@name\space c1r1.north}-|{\pgfplots@group@name\space c1r1.outer
west})!0.5!({\pgfplots@group@name\space c1r\pgfplots@group@rows.south}-|{\pgfplots@group@name\space c1r\pgfplots@group@rows.outer west})$)},
        anchor=south
    },
    execute at end groupplot/.code={%
      \node [/pgfplots/every groupplot x label]
{\pgfkeysvalueof{/pgfplots/groupplot xlabel}};  
      \node [/pgfplots/every groupplot y label] 
{\pgfkeysvalueof{/pgfplots/groupplot ylabel}};  
    }
}

\def\endpgfplots@environment@groupplot{%
    \endpgfplots@environment@opt%
    \pgfkeys{/pgfplots/execute at end groupplot}%
    \endgroup%
}



%%%%%%%%%%%%%%%%%%%%%%%%%%%%%%%%%%%%%%%%%%%%%%%%%
%   IDEA Line
%%%%%%%%%%%%%%%%%%%%%%%%%%%%%%%%%%%%%%%%%%%%%%%%%
\pgfplotsset{
  IDEA line/.style={
 	cycle list name = avatar,
        width  = \columnwidth,
        height = 6cm,
        line width=10pt,
        	axis line style = very thin,
        font = \sffamily,
        % X Axis
        %major x tick style = transparent,
        xlabel style={name=xlabel},
	% Y Axis
        ymajorgrids = true,
        scaled y ticks = false,
        separate axis lines,
	axis lines* = left,
        ymajorgrids = true,
        major tick length = -5,
        %Legend
        legend style={at={(xlabel.south)},yshift=2ex, xshift=-2ex, anchor=north, legend columns = 3, draw=none, column sep = 0.25cm},
        	%legend image code/.code={%
            %        \draw[#1,fill] circle (0.075cm);
            %    },
	}
}

%%%%%%%%%%%%%%%%%%%%%%%%%%%%%%%%%%%%%%%%%%%%%%%%%
%   IDEA Bar / A Clustered Bar Graph
%%%%%%%%%%%%%%%%%%%%%%%%%%%%%%%%%%%%%%%%%%%%%%%%%
\pgfplotsset{
  IDEA bar/.style={
  	cycle list name = waterloo-light,
	ybar = 0pt,
        width  = \columnwidth,
        height = 6cm,
        	axis line style = very thin,
       	bar width=25pt,
        font = \sffamily,
        	% X Axis
        	major x tick style = transparent,
        	enlarge x limits = {abs=1.5cm},
	xlabel={a},
        	xlabel style={name=xlabel},
	% Y Axis
        	ymajorgrids = true,
        	ylabel = {Selection Time (s)},
        	scaled y ticks = false,
        	separate axis lines,
	axis lines* = left,
        	ymajorgrids = true,
        	major tick length = 0,
        %Legend
        	legend style={at={(xlabel.south)},yshift=2ex, xshift=-2ex, anchor=north, legend columns = 3, draw=none, column sep = 0.25cm},
        	legend image code/.code={\draw[#1] rectangle (0.25cm,0.25cm);}
	}
}

%%%%%%%%%%%%%%%%%%%%%%%%%%%%%%%%%%%%%%%%%%%%%%%%%
%   IDEA StackedBar / A Clustered Bar Graph
%%%%%%%%%%%%%%%%%%%%%%%%%%%%%%%%%%%%%%%%%%%%%%%%%
\pgfplotsset{
  IDEA stacked bar/.style={
  	ybar stacked,
  	cycle list name = waterloo-light,
        width  = \columnwidth,
        height = 6cm,
        	axis line style = very thin,
       	bar width=25pt,
       	font = \sffamily,
        	% X Axis
        	major x tick style = transparent,
        	enlarge x limits = {abs=1.5cm},
	xlabel={a},
        	xlabel style={name=xlabel},
	% Y Axis
        	ymajorgrids = true,
        	ylabel = {Selection Time (s)},
        	scaled y ticks = false,
        	separate axis lines,
	axis lines* = left,
        	ymajorgrids = true,
        	major tick length = 0,
        %Legend
        	legend style={at={(xlabel.south)},yshift=2ex, xshift=-2ex, anchor=north, legend columns = 3, draw=none, column sep = 0.25cm},
        	legend image code/.code={\draw[#1] rectangle (0.25cm,0.25cm);}
	}
}


%%%%%%%%%%%%%%%%%%%%%%%%%%%%%%%%%%%%%%%%%%%%%%%%%
%   IDEA Likhert / Divergent Bar
%%%%%%%%%%%%%%%%%%%%%%%%%%%%%%%%%%%%%%%%%%%%%%%%%
\pgfplotsset{
  IDEA Likert/.style={
 	xbar stacked,
        	axis line style = very thin,
       	bar width=5pt,
        	% X Axis
        	%major x tick style = transparent,
	xlabel={a},
        	xlabel style={name=xlabel},
	xtick = {},
	font = \sffamily,
	% Y Axis
	        %axis y line=none,
	        major y tick style = transparent,
        	%ymajorgrids = true,
        	scaled y ticks = false,
        	separate axis lines,
	axis lines* = left,
        	major tick length = -2,
        %Legend
        	legend style={at={(xlabel.south)},yshift=-2ex, xshift=-2ex, anchor=north, legend columns = 3, draw=none, column sep = 0.25cm},
        	legend image code/.code={\draw[#1] rectangle (0.25cm,0.25cm);}
	}
}

%%%%%%%%%%%%%%%%%%%%%%%%%%%%%%%%%%%%%%%%%%%%%%%%%
%   Tufte Stacked Panel 
%%%%%%%%%%%%%%%%%%%%%%%%%%%%%%%%%%%%%%%%%%%%%%%%%
\pgfplotsset{
  IDEA tufte panel/.style={
  cycle list name = waterloo-light,
 ybar, 
	% Generic Bar Graph Options
        width = 1.1\columnwidth,
        height = .325\columnwidth,
        font = \sffamily,
	% X Axis
	enlarge x limits = {abs=.35cm},
	axis line style = very thin,
	% Y Axis
	yticklabel=\empty,
        separate axis lines,
	y axis line style= { draw opacity=0 },
	axis lines* = left,
        axis on top,
        ymajorgrids = true,
        major tick length = 0,
        grid style = {white},
	% Legend Formatting
	title style = {yshift=-.75cm,font=\tiny}
	}
}

%%%%%%%%%%%%%%%%%%%%%%%%%%%%%%%%%%%%%%%%%%%%%%%%%
%   Tufte Range Frame
%%%%%%%%%%%%%%%%%%%%%%%%%%%%%%%%%%%%%%%%%%%%%%%%%
\pgfplotsset{
range frame/.style={
    tick align=outside,
    axis line style={opacity=0},
    after end axis/.code={
      \draw ({rel axis cs:0,0}
          -|{axis cs:\pgfplotsdataxmin,0})
        -- ({rel axis cs:0,0}
          -|{axis cs:\pgfplotsdataxmax,0});
      \draw ({rel axis cs:0,0}
          |-{axis cs:0,\pgfplotsdataymin})
        -- ({rel axis cs:0,0}
          |-{axis cs:0,\pgfplotsdataymax});
} }
}

%%%%%%%%%%%%%%%%%%%%%%%%%%%%%%%%%%%%%%%%%%%%%%%%%
%   Tufte Dot Dash 
%%%%%%%%%%%%%%%%%%%%%%%%%%%%%%%%%%%%%%%%%%%%%%%%%

\pgfplotsset{
  dot dash plot/.style={
    tufte scatter
    axis line style={opacity=0},
    tick style={thin, black},
    major tick length=0.15cm,
    xtick=data,
    xticklabels={},
    ytick=data,
    yticklabels={},
    extra x ticks={
      \pgfplotsdataxmin,
      \pgfplotsdataxmax
    },
    extra y ticks={
      \pgfplotsdataymin,
      \pgfplotsdataymax
    },
    extra tick style={
      xticklabel={\pgfmathprintnumber[
        fixed,
        fixed zerofill,
        precision=1
      ]{\tick}},
      yticklabel={\pgfmathprintnumber[
        fixed,
        fixed zerofill,
        precision=1
]{\tick}} }
} }

%%%%%%%%%%%%%%%%%%%%%%%%%%%%%%%%%%%%%%%%%%%%%%%%%
%   Tufte Box Plot 
%%%%%%%%%%%%%%%%%%%%%%%%%%%%%%%%%%%%%%%%%%%%%%%%%
\pgfplotsset{
  IDEA tufte box/.style={
	% Colour Options
	cycle list name = waterloo-light,
	ybar = 0pt,
        width  = \columnwidth,
        height = 6cm,
        	axis line style = very thin,
       	bar width=25pt,
       	font = \sffamily,
        	% X Axis
        	major x tick style = transparent,
        	enlarge x limits = {abs=1.5cm},
        	xlabel style={name=xlabel},
	% Y Axis
        	ymajorgrids = true,
        	scaled y ticks = false,
        	separate axis lines,
	axis lines* = left,
        	ymajorgrids = true,
        	major tick length = 0,
        %Legend
        	legend style={at={(xlabel.south)},yshift=2ex, xshift=-2ex, anchor=north, legend columns = 3, draw=none, column sep = 0.25cm},
        	legend image code/.code={\draw[#1] rectangle (0.25cm,0.25cm);}
 	% Boxplot specific options
	boxplot,
	boxplot/draw direction=y, 
    	%clip = false,
    	every axis plot/.style={
      	mark=o,
      	boxplot/draw direction=y,
      	boxplot/whisker extend=0,
      	boxplot/draw/box/.code={ },
	boxplot/draw/median/.code={%
	 \draw[mark size=2pt,/pgfplots/boxplot/every median/.try]
          \pgfextra
          \pgftransformshift{
            \pgfplotsboxplotpointabbox
              {\pgfplotsboxplotvalue{median}}
              {0.5}
          }
          \pgfsetfillcolor{black}
          \pgfuseplotmark{*}
          \endpgfextra
        	;
      	},
      }
  },
}


%%%%%%%%%%%%%%%%%%%%%%%%%%%%%%%%%%%%%%%%%%%%%%%%%
%   IDEA Box Plot 
%%%%%%%%%%%%%%%%%%%%%%%%%%%%%%%%%%%%%%%%%%%%%%%%%

\pgfplotsset{
    IDEA box/.style={
    	% Colour Options
	cycle list name = waterloo-light,
	ybar = 0pt,
        width  = \columnwidth,
        height = 6cm,
        axis line style = very thin,
        font = \sffamily,
        	% X Axis
        	major x tick style = transparent,
        	enlarge x limits = {abs=1cm},
        	xlabel style={name=xlabel},
	% Y Axis
        	ymajorgrids = true,
        	scaled y ticks = false,
        	separate axis lines,
	axis lines* = left,
        	ymajorgrids = true,
        	major tick length = 0,
        %Legend
        	legend style={at={(xlabel.south)},yshift=2ex, xshift=-2ex, anchor=north, legend columns = 3, draw=none, column sep = 0.25cm},
        	legend image code/.code={\draw[#1] rectangle (0.25cm,0.25cm);}
        % draw whiskers as a single line:
        boxplot/draw/whisker/.code 2 args={%
            \draw[/pgfplots/boxplot/every whisker/.try]
                (boxplot cs:##1) -- (boxplot cs:##2)
            ;
        },%
        %
        % fill the boxes:
        boxplot/every box/.style={
            fill,
        },
        % 
        % the median should be visualized as a thick white line:
        boxplot/every median/.style={
        		ultra thick,
		fill=white, draw=white,
        },
        %boxplot/draw/median/.code={%
        %    \draw[fill=white]
        %        (boxplot cs:\pgfplotsboxplotvalue{median}) circle (3pt)
        %    ;
        %},
        % draw the average as a circle
        boxplot/average=auto,
        boxplot/draw/average/.code={%
	   \draw[fill=white,draw=gray]
            	(boxplot cs:\pgfplotsboxplotvalue{average}) circle (2pt)
            ;
        },
        %
        % do not clip to avoid problems with the median:
        clip=false,
        %
        boxplot/draw direction=y,
        boxplot/whisker extend=0,
        boxplot/every whisker/.style = very thick,
        %
        %
        % width of boxes:
        boxplot/box extend=0.15,
    },
    %
    %
    rshift/.style={
        xshift=\pgfkeysvalueof{/pgfplots/rshift scale},
        legend image post style={xshift=-\pgfkeysvalueof{/pgfplots/rshift scale}},
    },
    lshift/.style={
        xshift=-\pgfkeysvalueof{/pgfplots/lshift scale},
        legend image post style={xshift=\pgfkeysvalueof{/pgfplots/lshift scale}},
    },
    rshift scale/.initial=1em,
    lshift scale/.initial=1em,
}

%%%%%%%%%%%%%%%%%%%%%%%%%%%%%%%%%%%%%%%%%%%%%%%%%
% allows multiple lines for the same Y variable in stacked plots
%%%%%%%%%%%%%%%%%%%%%%%%%%%%%%%%%%%%%%%%%%%%%%%%%
\newcommand\resetstackedplots{
\makeatletter
\pgfplots@stacked@isfirstplottrue
\makeatother
}

%%%%%%%%%%%%%%%%%%%%%%%%%%%%%%%%%%%%%%%%%%%%%%%%%
% EOF
%%%%%%%%%%%%%%%%%%%%%%%%%%%%%%%%%%%%%%%%%%%%%%%%%
\makeatother
%
%
% Then, add graphs to your paper using the following template: 
%
%\begin{tikzpicture}
%\begin{axis}[<GRAPH TYPE>,
%	% X Axis options (xmin, xmax, xtick={})
%	% Y Axis options (ymin, ymax, ytick={})
%       ]
%
%    % DATA GOES IN HERE
%
%\end{axis}
%\end{tikzpicture}
%
%
% OR, for multiple plots use this template:
%
%\begin{tikzpicture}
%\begin{groupplot}[
%   group style={
%       group size= <COL> by <ROW>,
%       x descriptions at=edge bottom,
%       y descriptions at=edge left,
%       vertical sep=0pt,
%       horizontal sep=0pt},
%  <GRAPH TYPE>, 
%	% X Axis options (xmin, xmax, xtick={})
%	% Y Axis options (ymin, ymax, ytick={})
%      ]
%
%    % DATA GOES IN HERE
%
%\end{groupplot}
%\end{tikzpicture}
%
%
%
% The following graph types are defined in this library:
%
% IDEA bar - A clustered bar graph, similar to Excel
% IDEA tufte panel - a bare-bones bar graph, suitable for stacked panel graphs in Tufte's style
% .. more that aren't listed here
%




%%%%%%%%%%%%%%%%%%%%%%%%%%%%%%%%%%%%%%%%%%%%%%%%%
%   Common Includes
%%%%%%%%%%%%%%%%%%%%%%%%%%%%%%%%%%%%%%%%%%%%%%%%%
\usepackage{pgfplots}
\usepackage{pgfplotstable}
\usepgfplotslibrary{dateplot}
\usepgfplotslibrary{groupplots}


%\usetikzlibrary{external}
%\tikzexternalize[prefix=tikz/] % Must be in main .tex file?

\usetikzlibrary{pgfplots.statistics}
\usetikzlibrary{matrix}

\usepackage{bbding}

\pgfplotsset{compat=newest}

%%%%%%%%%%%%%%%%%%%%%%%%%%%%%%%%%%%%%%%%%%%%%%%%%
%   Colour Definitions
%%%%%%%%%%%%%%%%%%%%%%%%%%%%%%%%%%%%%%%%%%%%%%%%%

\definecolor{RYB1}{RGB}{141, 211, 199}
\definecolor{RYB2}{RGB}{255, 255, 179}
\definecolor{RYB3}{RGB}{190, 186, 218}
\definecolor{RYB4}{RGB}{251, 128, 114}
\definecolor{RYB5}{RGB}{128, 177, 211}
\definecolor{RYB6}{RGB}{253, 180, 98}
\definecolor{RYB7}{RGB}{179, 222, 105}

\pgfplotscreateplotcyclelist{colorbrewer-RYB}{
{RYB1!50!black,fill=RYB1},
{RYB2!50!black,fill=RYB2},
{RYB3!50!black,fill=RYB3},
{RYB4!50!black,fill=RYB4},
{RYB5!50!black,fill=RYB5},
{RYB6!50!black,fill=RYB6},
{RYB7!50!black,fill=RYB7},
}

\pgfplotscreateplotcyclelist{colorbrewer-RYB-plain}{
{RYB1},
{RYB2},
{RYB3},
{RYB4},
{RYB5},
{RYB6},
{RYB7},
}

%%%%%%%%%%%%%%%%%%%%
% WATERLOO PALETTES - DIGITAL
%%%%%%%%%%%%%%%%%%%%

\definecolor{AHSLight}{RGB}{0, 154, 166}
\definecolor{AHSDark}{RGB}{0, 127, 138}

\definecolor{ArtsLight}{RGB}{233, 131, 0}
\definecolor{ArtsDark}{RGB}{172, 97, 0}

\definecolor{EngineeringLight}{RGB}{204, 170, 255}
\definecolor{EngineeringDark}{RGB}{87, 6, 140}

\definecolor{EnvironmentLight}{RGB}{182, 191, 0}
\definecolor{EnvironmentDark}{RGB}{116, 120, 0}

\definecolor{MathLight}{RGB}{255, 136, 221}
\definecolor{MathDark}{RGB}{224, 36, 154}
 
\definecolor{ScienceLight}{RGB}{119, 187, 225}
\definecolor{ScienceDark}{RGB}{0, 115, 207}


\definecolor{SchoolRedLight}{RGB}{247, 119, 119}
\definecolor{SchoolRedDark}{RGB}{150, 23, 46}



\pgfplotscreateplotcyclelist{waterloo-light}{
{AHSLight!50!black,fill=AHSLight},
{ArtsLight!50!black,fill=ArtsLight},
{EngineeringLight!50!black,fill=EngineeringLight},
{EnvironmentLight!50!black,fill=EnvironmentLight},
{MathLight!50!black,fill=MathLight},
{ScienceLight!50!black,fill=ScienceLight},
{SchoolRedLight!50!black,fill=SchoolRedLight},
}

\pgfplotscreateplotcyclelist{waterloo-dark}{
{AHSDark!50!black,fill=AHSDark},
{ArtsDark!50!black,fill=ArtsDark},
{EngineeringDark!50!black,fill=EngineeringDark},
{EnvironmentDark!50!black,fill=EnvironmentDark},
{MathDark!50!black,fill=MathDark},
{ScienceDark!50!black,fill=ScienceDark},
{SchoolRedDark!50!black,fill=SchoolRedDark},
}



%%%%%%%%%%%%%%%%%%%%
% WATERLOO  PALETTES - PRINT
%%%%%%%%%%%%%%%%%%%%

% FIX: CMYK is conflicting with other colour definitions in SIGCHI template

%\definecolor{AHSPrint}{CMYK}{100, 0, 30, 2}
%\definecolor{ArtsPrint}{CMYK}{0, 52, 100, 0}
%\definecolor{EngineeringPrint}{CMYK}{24, 0, 98, 8}
%\definecolor{EnvironmentPrint}{CMYK}{78, 94, 0, 0}
%\definecolor{MathPrint}{CMYK}{5, 90, 0, 0}
%\definecolor{SciencePrint}{CMYK}{90, 48, 0, 0}
%\definecolor{SchoolRedPrint}{CMYK}{3, 100, 66, 12}

%\pgfplotscreateplotcyclelist{waterloo-print}{
%{AHSPrint!50!black,fill=AHSPrint},
%{ArtsPrint!50!black,fill=ArtsPrint},
%{EngineeringPrint!50!black,fill=EngineeringPrint},
%{EnvironmentPrint!50!black,fill=EnvironmentPrint},
%{MathPrint!50!black,fill=MathPrint},
%{SciencePrint!50!black,fill=SciencePrint},
%{SchoolRedPrint!50!black,fill=SchoolRedPrint},
%}




%%%%%%%%%%%%%%%%%%%%
% 5-Point DIVERGENT PALETTES - From ColorBrewer2
%%%%%%%%%%%%%%%%%%%%

\definecolor{Likert5_SD}{RGB}{166,97,26}
\definecolor{Likert5_D}{RGB}{223,194,125}
\definecolor{Likert5_N}{RGB}{245,245,245}
\definecolor{Likert5_A}{RGB}{128,205,193}
\definecolor{Likert5_SA}{RGB}{1,133,113}

\definecolor{Likert5_1_SD}{RGB}{166,97,26}
\definecolor{Likert5_1_D}{RGB}{223,194,125}
\definecolor{Likert5_1_N}{RGB}{245,245,245}
\definecolor{Likert5_1_A}{RGB}{128,205,193}
\definecolor{Likert5_1_SA}{RGB}{1,133,113}

\definecolor{Likert5_2_SD}{RGB}{123,50,148}
\definecolor{Likert5_2_D}{RGB}{194,165,207}
\definecolor{Likert5_2_N}{RGB}{247,247,247}
\definecolor{Likert5_2_A}{RGB}{166,219,160}
\definecolor{Likert5_2_SA}{RGB}{0,136,55}



%%%%%%%%%%%%%%%%%%%%
% 7-Point DIVERGENT PALETTES - From ColorBrewer2
%%%%%%%%%%%%%%%%%%%%

% Purple -> Green
\definecolor{Likert7_0}{RGB}{118,42,131}
\definecolor{Likert7_1}{RGB}{175,141,195}
\definecolor{Likert7_2}{RGB}{231,212,232}
\definecolor{Likert7_3}{RGB}{229,229,229}
\definecolor{Likert7_4}{RGB}{217,240,211}
\definecolor{Likert7_5}{RGB}{127,191,123}
\definecolor{Likert7_6}{RGB}{27,120,55}

% Orange/Brown -> Blue
\definecolor{Likert7_2_0}{RGB}{140,81,10}
\definecolor{Likert7_2_1}{RGB}{216,179,101}
\definecolor{Likert7_2_2}{RGB}{246,232,195}
\definecolor{Likert7_2_3}{RGB}{229,229,229}
\definecolor{Likert7_2_4}{RGB}{199,234,229}
\definecolor{Likert7_2_5}{RGB}{90,180,172}
\definecolor{Likert7_2_6}{RGB}{1,102,94}

% Orange -> Purple
\definecolor{Likert7_3_0}{RGB}{179,88,6}
\definecolor{Likert7_3_1}{RGB}{241,163,64}
\definecolor{Likert7_3_2}{RGB}{254,224,182}
\definecolor{Likert7_3_3}{RGB}{229,229,229}
\definecolor{Likert7_3_4}{RGB}{216,218,235}
\definecolor{Likert7_3_5}{RGB}{153,142,195}
\definecolor{Likert7_3_6}{RGB}{84,39,136}

% Blue -> Red
\definecolor{Likert7_4_0}{RGB}{33,102,172}
\definecolor{Likert7_4_1}{RGB}{103,169,207}
\definecolor{Likert7_4_2}{RGB}{209,229,240}
\definecolor{Likert7_4_3}{RGB}{229,229,229}
\definecolor{Likert7_4_4}{RGB}{253,219,199}
\definecolor{Likert7_4_5}{RGB}{239,138,98}
\definecolor{Likert7_4_6}{RGB}{178,24,43}


%%%%%%%%%%%%%%%%%%%%
% 7-Point DIVERGENT PALETTES - Jim for Avatar Paper
%%%%%%%%%%%%%%%%%%%%

% Blue -> Green to match Cleric
\definecolor{Cleric_0}{RGB}{33,102,172}
\definecolor{Cleric_1}{RGB}{103,169,207}
\definecolor{Cleric_2}{RGB}{209,229,240}
\definecolor{Cleric_3}{RGB}{229,229,229}
\definecolor{Cleric_4}{RGB}{217,240,211}
\definecolor{Cleric_5}{RGB}{127,191,123}
\definecolor{Cleric_6}{RGB}{27,120,55}

% Orange -> Red
\definecolor{Monster_0}{RGB}{178,24,43}
\definecolor{Monster_1}{RGB}{239,138,98}
\definecolor{Monster_2}{RGB}{253,219,199}
\definecolor{Monster_3}{RGB}{229,229,229}
\definecolor{Monster_4}{RGB}{216,218,235}
\definecolor{Monster_5}{RGB}{153,142,195}
\definecolor{Monster_6}{RGB}{84,39,136}


\pgfplotscreateplotcyclelist{avatar}{Cleric_0, Monster_0}

%%%%%%%%%%%%%%%%%%%%%%%%%%%%%%%%%%%%%%%%%%%%%%%%%
%   Outer South Legend Hack
%%%%%%%%%%%%%%%%%%%%%%%%%%%%%%%%%%%%%%%%%%%%%%%%%
\makeatletter
\pgfplotsset{
    every axis x label/.append style={
        alias=current axis xlabel
    },
    legend pos/outer south/.style={
        /pgfplots/legend style={
            at={%
                (%
                \@ifundefined{pgf@sh@ns@current axis xlabel}%
                {xticklabel cs:0.5}%
                {current axis xlabel.south}%
                )%
            },
            anchor=north
        }
    }
}

%%%%%%%%%%%%%%%%%%%%%%%%%%%%%%%%%%%%%%%%%%%%%%%%%
%   Grid Figures Label Hack
%%%%%%%%%%%%%%%%%%%%%%%%%%%%%%%%%%%%%%%%%%%%%%%%%

\pgfplotsset{
    groupplot xlabel/.initial={},
    every groupplot x label/.style={
        at={($({\pgfplots@group@name\space c1r\pgfplots@group@rows.west}|-{\pgfplots@group@name\space c1r\pgfplots@group@rows.outer south})!0.5!({\pgfplots@group@name\space c\pgfplots@group@columns r\pgfplots@group@rows.east}|-{\pgfplots@group@name\space c\pgfplots@group@columns r\pgfplots@group@rows.outer south})$)},
        anchor=north,
    },
    groupplot ylabel/.initial={},
    every groupplot y label/.style={
            rotate=90,
        at={($({\pgfplots@group@name\space c1r1.north}-|{\pgfplots@group@name\space c1r1.outer
west})!0.5!({\pgfplots@group@name\space c1r\pgfplots@group@rows.south}-|{\pgfplots@group@name\space c1r\pgfplots@group@rows.outer west})$)},
        anchor=south
    },
    execute at end groupplot/.code={%
      \node [/pgfplots/every groupplot x label]
{\pgfkeysvalueof{/pgfplots/groupplot xlabel}};  
      \node [/pgfplots/every groupplot y label] 
{\pgfkeysvalueof{/pgfplots/groupplot ylabel}};  
    }
}

\def\endpgfplots@environment@groupplot{%
    \endpgfplots@environment@opt%
    \pgfkeys{/pgfplots/execute at end groupplot}%
    \endgroup%
}



%%%%%%%%%%%%%%%%%%%%%%%%%%%%%%%%%%%%%%%%%%%%%%%%%
%   IDEA Line
%%%%%%%%%%%%%%%%%%%%%%%%%%%%%%%%%%%%%%%%%%%%%%%%%
\pgfplotsset{
  IDEA line/.style={
 	cycle list name = avatar,
        width  = \columnwidth,
        height = 6cm,
        line width=10pt,
        	axis line style = very thin,
        font = \sffamily,
        % X Axis
        %major x tick style = transparent,
        xlabel style={name=xlabel},
	% Y Axis
        ymajorgrids = true,
        scaled y ticks = false,
        separate axis lines,
	axis lines* = left,
        ymajorgrids = true,
        major tick length = -5,
        %Legend
        legend style={at={(xlabel.south)},yshift=2ex, xshift=-2ex, anchor=north, legend columns = 3, draw=none, column sep = 0.25cm},
        	%legend image code/.code={%
            %        \draw[#1,fill] circle (0.075cm);
            %    },
	}
}

%%%%%%%%%%%%%%%%%%%%%%%%%%%%%%%%%%%%%%%%%%%%%%%%%
%   IDEA Bar / A Clustered Bar Graph
%%%%%%%%%%%%%%%%%%%%%%%%%%%%%%%%%%%%%%%%%%%%%%%%%
\pgfplotsset{
  IDEA bar/.style={
  	cycle list name = waterloo-light,
	ybar = 0pt,
        width  = \columnwidth,
        height = 6cm,
        	axis line style = very thin,
       	bar width=25pt,
        font = \sffamily,
        	% X Axis
        	major x tick style = transparent,
        	enlarge x limits = {abs=1.5cm},
	xlabel={a},
        	xlabel style={name=xlabel},
	% Y Axis
        	ymajorgrids = true,
        	ylabel = {Selection Time (s)},
        	scaled y ticks = false,
        	separate axis lines,
	axis lines* = left,
        	ymajorgrids = true,
        	major tick length = 0,
        %Legend
        	legend style={at={(xlabel.south)},yshift=2ex, xshift=-2ex, anchor=north, legend columns = 3, draw=none, column sep = 0.25cm},
        	legend image code/.code={\draw[#1] rectangle (0.25cm,0.25cm);}
	}
}

%%%%%%%%%%%%%%%%%%%%%%%%%%%%%%%%%%%%%%%%%%%%%%%%%
%   IDEA StackedBar / A Clustered Bar Graph
%%%%%%%%%%%%%%%%%%%%%%%%%%%%%%%%%%%%%%%%%%%%%%%%%
\pgfplotsset{
  IDEA stacked bar/.style={
  	ybar stacked,
  	cycle list name = waterloo-light,
        width  = \columnwidth,
        height = 6cm,
        	axis line style = very thin,
       	bar width=25pt,
       	font = \sffamily,
        	% X Axis
        	major x tick style = transparent,
        	enlarge x limits = {abs=1.5cm},
	xlabel={a},
        	xlabel style={name=xlabel},
	% Y Axis
        	ymajorgrids = true,
        	ylabel = {Selection Time (s)},
        	scaled y ticks = false,
        	separate axis lines,
	axis lines* = left,
        	ymajorgrids = true,
        	major tick length = 0,
        %Legend
        	legend style={at={(xlabel.south)},yshift=2ex, xshift=-2ex, anchor=north, legend columns = 3, draw=none, column sep = 0.25cm},
        	legend image code/.code={\draw[#1] rectangle (0.25cm,0.25cm);}
	}
}


%%%%%%%%%%%%%%%%%%%%%%%%%%%%%%%%%%%%%%%%%%%%%%%%%
%   IDEA Likhert / Divergent Bar
%%%%%%%%%%%%%%%%%%%%%%%%%%%%%%%%%%%%%%%%%%%%%%%%%
\pgfplotsset{
  IDEA Likert/.style={
 	xbar stacked,
        	axis line style = very thin,
       	bar width=5pt,
        	% X Axis
        	%major x tick style = transparent,
	xlabel={a},
        	xlabel style={name=xlabel},
	xtick = {},
	font = \sffamily,
	% Y Axis
	        %axis y line=none,
	        major y tick style = transparent,
        	%ymajorgrids = true,
        	scaled y ticks = false,
        	separate axis lines,
	axis lines* = left,
        	major tick length = -2,
        %Legend
        	legend style={at={(xlabel.south)},yshift=-2ex, xshift=-2ex, anchor=north, legend columns = 3, draw=none, column sep = 0.25cm},
        	legend image code/.code={\draw[#1] rectangle (0.25cm,0.25cm);}
	}
}

%%%%%%%%%%%%%%%%%%%%%%%%%%%%%%%%%%%%%%%%%%%%%%%%%
%   Tufte Stacked Panel 
%%%%%%%%%%%%%%%%%%%%%%%%%%%%%%%%%%%%%%%%%%%%%%%%%
\pgfplotsset{
  IDEA tufte panel/.style={
  cycle list name = waterloo-light,
 ybar, 
	% Generic Bar Graph Options
        width = 1.1\columnwidth,
        height = .325\columnwidth,
        font = \sffamily,
	% X Axis
	enlarge x limits = {abs=.35cm},
	axis line style = very thin,
	% Y Axis
	yticklabel=\empty,
        separate axis lines,
	y axis line style= { draw opacity=0 },
	axis lines* = left,
        axis on top,
        ymajorgrids = true,
        major tick length = 0,
        grid style = {white},
	% Legend Formatting
	title style = {yshift=-.75cm,font=\tiny}
	}
}

%%%%%%%%%%%%%%%%%%%%%%%%%%%%%%%%%%%%%%%%%%%%%%%%%
%   Tufte Range Frame
%%%%%%%%%%%%%%%%%%%%%%%%%%%%%%%%%%%%%%%%%%%%%%%%%
\pgfplotsset{
range frame/.style={
    tick align=outside,
    axis line style={opacity=0},
    after end axis/.code={
      \draw ({rel axis cs:0,0}
          -|{axis cs:\pgfplotsdataxmin,0})
        -- ({rel axis cs:0,0}
          -|{axis cs:\pgfplotsdataxmax,0});
      \draw ({rel axis cs:0,0}
          |-{axis cs:0,\pgfplotsdataymin})
        -- ({rel axis cs:0,0}
          |-{axis cs:0,\pgfplotsdataymax});
} }
}

%%%%%%%%%%%%%%%%%%%%%%%%%%%%%%%%%%%%%%%%%%%%%%%%%
%   Tufte Dot Dash 
%%%%%%%%%%%%%%%%%%%%%%%%%%%%%%%%%%%%%%%%%%%%%%%%%

\pgfplotsset{
  dot dash plot/.style={
    tufte scatter
    axis line style={opacity=0},
    tick style={thin, black},
    major tick length=0.15cm,
    xtick=data,
    xticklabels={},
    ytick=data,
    yticklabels={},
    extra x ticks={
      \pgfplotsdataxmin,
      \pgfplotsdataxmax
    },
    extra y ticks={
      \pgfplotsdataymin,
      \pgfplotsdataymax
    },
    extra tick style={
      xticklabel={\pgfmathprintnumber[
        fixed,
        fixed zerofill,
        precision=1
      ]{\tick}},
      yticklabel={\pgfmathprintnumber[
        fixed,
        fixed zerofill,
        precision=1
]{\tick}} }
} }

%%%%%%%%%%%%%%%%%%%%%%%%%%%%%%%%%%%%%%%%%%%%%%%%%
%   Tufte Box Plot 
%%%%%%%%%%%%%%%%%%%%%%%%%%%%%%%%%%%%%%%%%%%%%%%%%
\pgfplotsset{
  IDEA tufte box/.style={
	% Colour Options
	cycle list name = waterloo-light,
	ybar = 0pt,
        width  = \columnwidth,
        height = 6cm,
        	axis line style = very thin,
       	bar width=25pt,
       	font = \sffamily,
        	% X Axis
        	major x tick style = transparent,
        	enlarge x limits = {abs=1.5cm},
        	xlabel style={name=xlabel},
	% Y Axis
        	ymajorgrids = true,
        	scaled y ticks = false,
        	separate axis lines,
	axis lines* = left,
        	ymajorgrids = true,
        	major tick length = 0,
        %Legend
        	legend style={at={(xlabel.south)},yshift=2ex, xshift=-2ex, anchor=north, legend columns = 3, draw=none, column sep = 0.25cm},
        	legend image code/.code={\draw[#1] rectangle (0.25cm,0.25cm);}
 	% Boxplot specific options
	boxplot,
	boxplot/draw direction=y, 
    	%clip = false,
    	every axis plot/.style={
      	mark=o,
      	boxplot/draw direction=y,
      	boxplot/whisker extend=0,
      	boxplot/draw/box/.code={ },
	boxplot/draw/median/.code={%
	 \draw[mark size=2pt,/pgfplots/boxplot/every median/.try]
          \pgfextra
          \pgftransformshift{
            \pgfplotsboxplotpointabbox
              {\pgfplotsboxplotvalue{median}}
              {0.5}
          }
          \pgfsetfillcolor{black}
          \pgfuseplotmark{*}
          \endpgfextra
        	;
      	},
      }
  },
}


%%%%%%%%%%%%%%%%%%%%%%%%%%%%%%%%%%%%%%%%%%%%%%%%%
%   IDEA Box Plot 
%%%%%%%%%%%%%%%%%%%%%%%%%%%%%%%%%%%%%%%%%%%%%%%%%

\pgfplotsset{
    IDEA box/.style={
    	% Colour Options
	cycle list name = waterloo-light,
	ybar = 0pt,
        width  = \columnwidth,
        height = 6cm,
        axis line style = very thin,
        font = \sffamily,
        	% X Axis
        	major x tick style = transparent,
        	enlarge x limits = {abs=1cm},
        	xlabel style={name=xlabel},
	% Y Axis
        	ymajorgrids = true,
        	scaled y ticks = false,
        	separate axis lines,
	axis lines* = left,
        	ymajorgrids = true,
        	major tick length = 0,
        %Legend
        	legend style={at={(xlabel.south)},yshift=2ex, xshift=-2ex, anchor=north, legend columns = 3, draw=none, column sep = 0.25cm},
        	legend image code/.code={\draw[#1] rectangle (0.25cm,0.25cm);}
        % draw whiskers as a single line:
        boxplot/draw/whisker/.code 2 args={%
            \draw[/pgfplots/boxplot/every whisker/.try]
                (boxplot cs:##1) -- (boxplot cs:##2)
            ;
        },%
        %
        % fill the boxes:
        boxplot/every box/.style={
            fill,
        },
        % 
        % the median should be visualized as a thick white line:
        boxplot/every median/.style={
        		ultra thick,
		fill=white, draw=white,
        },
        %boxplot/draw/median/.code={%
        %    \draw[fill=white]
        %        (boxplot cs:\pgfplotsboxplotvalue{median}) circle (3pt)
        %    ;
        %},
        % draw the average as a circle
        boxplot/average=auto,
        boxplot/draw/average/.code={%
	   \draw[fill=white,draw=gray]
            	(boxplot cs:\pgfplotsboxplotvalue{average}) circle (2pt)
            ;
        },
        %
        % do not clip to avoid problems with the median:
        clip=false,
        %
        boxplot/draw direction=y,
        boxplot/whisker extend=0,
        boxplot/every whisker/.style = very thick,
        %
        %
        % width of boxes:
        boxplot/box extend=0.15,
    },
    %
    %
    rshift/.style={
        xshift=\pgfkeysvalueof{/pgfplots/rshift scale},
        legend image post style={xshift=-\pgfkeysvalueof{/pgfplots/rshift scale}},
    },
    lshift/.style={
        xshift=-\pgfkeysvalueof{/pgfplots/lshift scale},
        legend image post style={xshift=\pgfkeysvalueof{/pgfplots/lshift scale}},
    },
    rshift scale/.initial=1em,
    lshift scale/.initial=1em,
}

%%%%%%%%%%%%%%%%%%%%%%%%%%%%%%%%%%%%%%%%%%%%%%%%%
% allows multiple lines for the same Y variable in stacked plots
%%%%%%%%%%%%%%%%%%%%%%%%%%%%%%%%%%%%%%%%%%%%%%%%%
\newcommand\resetstackedplots{
\makeatletter
\pgfplots@stacked@isfirstplottrue
\makeatother
}

%%%%%%%%%%%%%%%%%%%%%%%%%%%%%%%%%%%%%%%%%%%%%%%%%
% EOF
%%%%%%%%%%%%%%%%%%%%%%%%%%%%%%%%%%%%%%%%%%%%%%%%%
\makeatother
%
%
% Then, add graphs to your paper using the following template: 
%
%\begin{tikzpicture}
%\begin{axis}[<GRAPH TYPE>,
%	% X Axis options (xmin, xmax, xtick={})
%	% Y Axis options (ymin, ymax, ytick={})
%       ]
%
%    % DATA GOES IN HERE
%
%\end{axis}
%\end{tikzpicture}
%
%
% OR, for multiple plots use this template:
%
%\begin{tikzpicture}
%\begin{groupplot}[
%   group style={
%       group size= <COL> by <ROW>,
%       x descriptions at=edge bottom,
%       y descriptions at=edge left,
%       vertical sep=0pt,
%       horizontal sep=0pt},
%  <GRAPH TYPE>, 
%	% X Axis options (xmin, xmax, xtick={})
%	% Y Axis options (ymin, ymax, ytick={})
%      ]
%
%    % DATA GOES IN HERE
%
%\end{groupplot}
%\end{tikzpicture}
%
%
%
% The following graph types are defined in this library:
%
% IDEA bar - A clustered bar graph, similar to Excel
% IDEA tufte panel - a bare-bones bar graph, suitable for stacked panel graphs in Tufte's style
% .. more that aren't listed here
%




%%%%%%%%%%%%%%%%%%%%%%%%%%%%%%%%%%%%%%%%%%%%%%%%%
%   Common Includes
%%%%%%%%%%%%%%%%%%%%%%%%%%%%%%%%%%%%%%%%%%%%%%%%%
\usepackage{pgfplots}
\usepackage{pgfplotstable}
\usepgfplotslibrary{dateplot}
\usepgfplotslibrary{groupplots}


%\usetikzlibrary{external}
%\tikzexternalize[prefix=tikz/] % Must be in main .tex file?

\usetikzlibrary{pgfplots.statistics}
\usetikzlibrary{matrix}

\usepackage{bbding}

\pgfplotsset{compat=newest}

%%%%%%%%%%%%%%%%%%%%%%%%%%%%%%%%%%%%%%%%%%%%%%%%%
%   Colour Definitions
%%%%%%%%%%%%%%%%%%%%%%%%%%%%%%%%%%%%%%%%%%%%%%%%%

\definecolor{RYB1}{RGB}{141, 211, 199}
\definecolor{RYB2}{RGB}{255, 255, 179}
\definecolor{RYB3}{RGB}{190, 186, 218}
\definecolor{RYB4}{RGB}{251, 128, 114}
\definecolor{RYB5}{RGB}{128, 177, 211}
\definecolor{RYB6}{RGB}{253, 180, 98}
\definecolor{RYB7}{RGB}{179, 222, 105}

\pgfplotscreateplotcyclelist{colorbrewer-RYB}{
{RYB1!50!black,fill=RYB1},
{RYB2!50!black,fill=RYB2},
{RYB3!50!black,fill=RYB3},
{RYB4!50!black,fill=RYB4},
{RYB5!50!black,fill=RYB5},
{RYB6!50!black,fill=RYB6},
{RYB7!50!black,fill=RYB7},
}

\pgfplotscreateplotcyclelist{colorbrewer-RYB-plain}{
{RYB1},
{RYB2},
{RYB3},
{RYB4},
{RYB5},
{RYB6},
{RYB7},
}

%%%%%%%%%%%%%%%%%%%%
% WATERLOO PALETTES - DIGITAL
%%%%%%%%%%%%%%%%%%%%

\definecolor{AHSLight}{RGB}{0, 154, 166}
\definecolor{AHSDark}{RGB}{0, 127, 138}

\definecolor{ArtsLight}{RGB}{233, 131, 0}
\definecolor{ArtsDark}{RGB}{172, 97, 0}

\definecolor{EngineeringLight}{RGB}{204, 170, 255}
\definecolor{EngineeringDark}{RGB}{87, 6, 140}

\definecolor{EnvironmentLight}{RGB}{182, 191, 0}
\definecolor{EnvironmentDark}{RGB}{116, 120, 0}

\definecolor{MathLight}{RGB}{255, 136, 221}
\definecolor{MathDark}{RGB}{224, 36, 154}
 
\definecolor{ScienceLight}{RGB}{119, 187, 225}
\definecolor{ScienceDark}{RGB}{0, 115, 207}


\definecolor{SchoolRedLight}{RGB}{247, 119, 119}
\definecolor{SchoolRedDark}{RGB}{150, 23, 46}



\pgfplotscreateplotcyclelist{waterloo-light}{
{AHSLight!50!black,fill=AHSLight},
{ArtsLight!50!black,fill=ArtsLight},
{EngineeringLight!50!black,fill=EngineeringLight},
{EnvironmentLight!50!black,fill=EnvironmentLight},
{MathLight!50!black,fill=MathLight},
{ScienceLight!50!black,fill=ScienceLight},
{SchoolRedLight!50!black,fill=SchoolRedLight},
}

\pgfplotscreateplotcyclelist{waterloo-dark}{
{AHSDark!50!black,fill=AHSDark},
{ArtsDark!50!black,fill=ArtsDark},
{EngineeringDark!50!black,fill=EngineeringDark},
{EnvironmentDark!50!black,fill=EnvironmentDark},
{MathDark!50!black,fill=MathDark},
{ScienceDark!50!black,fill=ScienceDark},
{SchoolRedDark!50!black,fill=SchoolRedDark},
}



%%%%%%%%%%%%%%%%%%%%
% WATERLOO  PALETTES - PRINT
%%%%%%%%%%%%%%%%%%%%

% FIX: CMYK is conflicting with other colour definitions in SIGCHI template

%\definecolor{AHSPrint}{CMYK}{100, 0, 30, 2}
%\definecolor{ArtsPrint}{CMYK}{0, 52, 100, 0}
%\definecolor{EngineeringPrint}{CMYK}{24, 0, 98, 8}
%\definecolor{EnvironmentPrint}{CMYK}{78, 94, 0, 0}
%\definecolor{MathPrint}{CMYK}{5, 90, 0, 0}
%\definecolor{SciencePrint}{CMYK}{90, 48, 0, 0}
%\definecolor{SchoolRedPrint}{CMYK}{3, 100, 66, 12}

%\pgfplotscreateplotcyclelist{waterloo-print}{
%{AHSPrint!50!black,fill=AHSPrint},
%{ArtsPrint!50!black,fill=ArtsPrint},
%{EngineeringPrint!50!black,fill=EngineeringPrint},
%{EnvironmentPrint!50!black,fill=EnvironmentPrint},
%{MathPrint!50!black,fill=MathPrint},
%{SciencePrint!50!black,fill=SciencePrint},
%{SchoolRedPrint!50!black,fill=SchoolRedPrint},
%}




%%%%%%%%%%%%%%%%%%%%
% 5-Point DIVERGENT PALETTES - From ColorBrewer2
%%%%%%%%%%%%%%%%%%%%

\definecolor{Likert5_SD}{RGB}{166,97,26}
\definecolor{Likert5_D}{RGB}{223,194,125}
\definecolor{Likert5_N}{RGB}{245,245,245}
\definecolor{Likert5_A}{RGB}{128,205,193}
\definecolor{Likert5_SA}{RGB}{1,133,113}

\definecolor{Likert5_1_SD}{RGB}{166,97,26}
\definecolor{Likert5_1_D}{RGB}{223,194,125}
\definecolor{Likert5_1_N}{RGB}{245,245,245}
\definecolor{Likert5_1_A}{RGB}{128,205,193}
\definecolor{Likert5_1_SA}{RGB}{1,133,113}

\definecolor{Likert5_2_SD}{RGB}{123,50,148}
\definecolor{Likert5_2_D}{RGB}{194,165,207}
\definecolor{Likert5_2_N}{RGB}{247,247,247}
\definecolor{Likert5_2_A}{RGB}{166,219,160}
\definecolor{Likert5_2_SA}{RGB}{0,136,55}



%%%%%%%%%%%%%%%%%%%%
% 7-Point DIVERGENT PALETTES - From ColorBrewer2
%%%%%%%%%%%%%%%%%%%%

% Purple -> Green
\definecolor{Likert7_0}{RGB}{118,42,131}
\definecolor{Likert7_1}{RGB}{175,141,195}
\definecolor{Likert7_2}{RGB}{231,212,232}
\definecolor{Likert7_3}{RGB}{229,229,229}
\definecolor{Likert7_4}{RGB}{217,240,211}
\definecolor{Likert7_5}{RGB}{127,191,123}
\definecolor{Likert7_6}{RGB}{27,120,55}

% Orange/Brown -> Blue
\definecolor{Likert7_2_0}{RGB}{140,81,10}
\definecolor{Likert7_2_1}{RGB}{216,179,101}
\definecolor{Likert7_2_2}{RGB}{246,232,195}
\definecolor{Likert7_2_3}{RGB}{229,229,229}
\definecolor{Likert7_2_4}{RGB}{199,234,229}
\definecolor{Likert7_2_5}{RGB}{90,180,172}
\definecolor{Likert7_2_6}{RGB}{1,102,94}

% Orange -> Purple
\definecolor{Likert7_3_0}{RGB}{179,88,6}
\definecolor{Likert7_3_1}{RGB}{241,163,64}
\definecolor{Likert7_3_2}{RGB}{254,224,182}
\definecolor{Likert7_3_3}{RGB}{229,229,229}
\definecolor{Likert7_3_4}{RGB}{216,218,235}
\definecolor{Likert7_3_5}{RGB}{153,142,195}
\definecolor{Likert7_3_6}{RGB}{84,39,136}

% Blue -> Red
\definecolor{Likert7_4_0}{RGB}{33,102,172}
\definecolor{Likert7_4_1}{RGB}{103,169,207}
\definecolor{Likert7_4_2}{RGB}{209,229,240}
\definecolor{Likert7_4_3}{RGB}{229,229,229}
\definecolor{Likert7_4_4}{RGB}{253,219,199}
\definecolor{Likert7_4_5}{RGB}{239,138,98}
\definecolor{Likert7_4_6}{RGB}{178,24,43}


%%%%%%%%%%%%%%%%%%%%
% 7-Point DIVERGENT PALETTES - Jim for Avatar Paper
%%%%%%%%%%%%%%%%%%%%

% Blue -> Green to match Cleric
\definecolor{Cleric_0}{RGB}{33,102,172}
\definecolor{Cleric_1}{RGB}{103,169,207}
\definecolor{Cleric_2}{RGB}{209,229,240}
\definecolor{Cleric_3}{RGB}{229,229,229}
\definecolor{Cleric_4}{RGB}{217,240,211}
\definecolor{Cleric_5}{RGB}{127,191,123}
\definecolor{Cleric_6}{RGB}{27,120,55}

% Orange -> Red
\definecolor{Monster_0}{RGB}{178,24,43}
\definecolor{Monster_1}{RGB}{239,138,98}
\definecolor{Monster_2}{RGB}{253,219,199}
\definecolor{Monster_3}{RGB}{229,229,229}
\definecolor{Monster_4}{RGB}{216,218,235}
\definecolor{Monster_5}{RGB}{153,142,195}
\definecolor{Monster_6}{RGB}{84,39,136}


\pgfplotscreateplotcyclelist{avatar}{Cleric_0, Monster_0}

%%%%%%%%%%%%%%%%%%%%%%%%%%%%%%%%%%%%%%%%%%%%%%%%%
%   Outer South Legend Hack
%%%%%%%%%%%%%%%%%%%%%%%%%%%%%%%%%%%%%%%%%%%%%%%%%
\makeatletter
\pgfplotsset{
    every axis x label/.append style={
        alias=current axis xlabel
    },
    legend pos/outer south/.style={
        /pgfplots/legend style={
            at={%
                (%
                \@ifundefined{pgf@sh@ns@current axis xlabel}%
                {xticklabel cs:0.5}%
                {current axis xlabel.south}%
                )%
            },
            anchor=north
        }
    }
}

%%%%%%%%%%%%%%%%%%%%%%%%%%%%%%%%%%%%%%%%%%%%%%%%%
%   Grid Figures Label Hack
%%%%%%%%%%%%%%%%%%%%%%%%%%%%%%%%%%%%%%%%%%%%%%%%%

\pgfplotsset{
    groupplot xlabel/.initial={},
    every groupplot x label/.style={
        at={($({\pgfplots@group@name\space c1r\pgfplots@group@rows.west}|-{\pgfplots@group@name\space c1r\pgfplots@group@rows.outer south})!0.5!({\pgfplots@group@name\space c\pgfplots@group@columns r\pgfplots@group@rows.east}|-{\pgfplots@group@name\space c\pgfplots@group@columns r\pgfplots@group@rows.outer south})$)},
        anchor=north,
    },
    groupplot ylabel/.initial={},
    every groupplot y label/.style={
            rotate=90,
        at={($({\pgfplots@group@name\space c1r1.north}-|{\pgfplots@group@name\space c1r1.outer
west})!0.5!({\pgfplots@group@name\space c1r\pgfplots@group@rows.south}-|{\pgfplots@group@name\space c1r\pgfplots@group@rows.outer west})$)},
        anchor=south
    },
    execute at end groupplot/.code={%
      \node [/pgfplots/every groupplot x label]
{\pgfkeysvalueof{/pgfplots/groupplot xlabel}};  
      \node [/pgfplots/every groupplot y label] 
{\pgfkeysvalueof{/pgfplots/groupplot ylabel}};  
    }
}

\def\endpgfplots@environment@groupplot{%
    \endpgfplots@environment@opt%
    \pgfkeys{/pgfplots/execute at end groupplot}%
    \endgroup%
}



%%%%%%%%%%%%%%%%%%%%%%%%%%%%%%%%%%%%%%%%%%%%%%%%%
%   IDEA Line
%%%%%%%%%%%%%%%%%%%%%%%%%%%%%%%%%%%%%%%%%%%%%%%%%
\pgfplotsset{
  IDEA line/.style={
 	cycle list name = avatar,
        width  = \columnwidth,
        height = 6cm,
        line width=10pt,
        	axis line style = very thin,
        font = \sffamily,
        % X Axis
        %major x tick style = transparent,
        xlabel style={name=xlabel},
	% Y Axis
        ymajorgrids = true,
        scaled y ticks = false,
        separate axis lines,
	axis lines* = left,
        ymajorgrids = true,
        major tick length = -5,
        %Legend
        legend style={at={(xlabel.south)},yshift=2ex, xshift=-2ex, anchor=north, legend columns = 3, draw=none, column sep = 0.25cm},
        	%legend image code/.code={%
            %        \draw[#1,fill] circle (0.075cm);
            %    },
	}
}

%%%%%%%%%%%%%%%%%%%%%%%%%%%%%%%%%%%%%%%%%%%%%%%%%
%   IDEA Bar / A Clustered Bar Graph
%%%%%%%%%%%%%%%%%%%%%%%%%%%%%%%%%%%%%%%%%%%%%%%%%
\pgfplotsset{
  IDEA bar/.style={
  	cycle list name = waterloo-light,
	ybar = 0pt,
        width  = \columnwidth,
        height = 6cm,
        	axis line style = very thin,
       	bar width=25pt,
        font = \sffamily,
        	% X Axis
        	major x tick style = transparent,
        	enlarge x limits = {abs=1.5cm},
	xlabel={a},
        	xlabel style={name=xlabel},
	% Y Axis
        	ymajorgrids = true,
        	ylabel = {Selection Time (s)},
        	scaled y ticks = false,
        	separate axis lines,
	axis lines* = left,
        	ymajorgrids = true,
        	major tick length = 0,
        %Legend
        	legend style={at={(xlabel.south)},yshift=2ex, xshift=-2ex, anchor=north, legend columns = 3, draw=none, column sep = 0.25cm},
        	legend image code/.code={\draw[#1] rectangle (0.25cm,0.25cm);}
	}
}

%%%%%%%%%%%%%%%%%%%%%%%%%%%%%%%%%%%%%%%%%%%%%%%%%
%   IDEA StackedBar / A Clustered Bar Graph
%%%%%%%%%%%%%%%%%%%%%%%%%%%%%%%%%%%%%%%%%%%%%%%%%
\pgfplotsset{
  IDEA stacked bar/.style={
  	ybar stacked,
  	cycle list name = waterloo-light,
        width  = \columnwidth,
        height = 6cm,
        	axis line style = very thin,
       	bar width=25pt,
       	font = \sffamily,
        	% X Axis
        	major x tick style = transparent,
        	enlarge x limits = {abs=1.5cm},
	xlabel={a},
        	xlabel style={name=xlabel},
	% Y Axis
        	ymajorgrids = true,
        	ylabel = {Selection Time (s)},
        	scaled y ticks = false,
        	separate axis lines,
	axis lines* = left,
        	ymajorgrids = true,
        	major tick length = 0,
        %Legend
        	legend style={at={(xlabel.south)},yshift=2ex, xshift=-2ex, anchor=north, legend columns = 3, draw=none, column sep = 0.25cm},
        	legend image code/.code={\draw[#1] rectangle (0.25cm,0.25cm);}
	}
}


%%%%%%%%%%%%%%%%%%%%%%%%%%%%%%%%%%%%%%%%%%%%%%%%%
%   IDEA Likhert / Divergent Bar
%%%%%%%%%%%%%%%%%%%%%%%%%%%%%%%%%%%%%%%%%%%%%%%%%
\pgfplotsset{
  IDEA Likert/.style={
 	xbar stacked,
        	axis line style = very thin,
       	bar width=5pt,
        	% X Axis
        	%major x tick style = transparent,
	xlabel={a},
        	xlabel style={name=xlabel},
	xtick = {},
	font = \sffamily,
	% Y Axis
	        %axis y line=none,
	        major y tick style = transparent,
        	%ymajorgrids = true,
        	scaled y ticks = false,
        	separate axis lines,
	axis lines* = left,
        	major tick length = -2,
        %Legend
        	legend style={at={(xlabel.south)},yshift=-2ex, xshift=-2ex, anchor=north, legend columns = 3, draw=none, column sep = 0.25cm},
        	legend image code/.code={\draw[#1] rectangle (0.25cm,0.25cm);}
	}
}

%%%%%%%%%%%%%%%%%%%%%%%%%%%%%%%%%%%%%%%%%%%%%%%%%
%   Tufte Stacked Panel 
%%%%%%%%%%%%%%%%%%%%%%%%%%%%%%%%%%%%%%%%%%%%%%%%%
\pgfplotsset{
  IDEA tufte panel/.style={
  cycle list name = waterloo-light,
 ybar, 
	% Generic Bar Graph Options
        width = 1.1\columnwidth,
        height = .325\columnwidth,
        font = \sffamily,
	% X Axis
	enlarge x limits = {abs=.35cm},
	axis line style = very thin,
	% Y Axis
	yticklabel=\empty,
        separate axis lines,
	y axis line style= { draw opacity=0 },
	axis lines* = left,
        axis on top,
        ymajorgrids = true,
        major tick length = 0,
        grid style = {white},
	% Legend Formatting
	title style = {yshift=-.75cm,font=\tiny}
	}
}

%%%%%%%%%%%%%%%%%%%%%%%%%%%%%%%%%%%%%%%%%%%%%%%%%
%   Tufte Range Frame
%%%%%%%%%%%%%%%%%%%%%%%%%%%%%%%%%%%%%%%%%%%%%%%%%
\pgfplotsset{
range frame/.style={
    tick align=outside,
    axis line style={opacity=0},
    after end axis/.code={
      \draw ({rel axis cs:0,0}
          -|{axis cs:\pgfplotsdataxmin,0})
        -- ({rel axis cs:0,0}
          -|{axis cs:\pgfplotsdataxmax,0});
      \draw ({rel axis cs:0,0}
          |-{axis cs:0,\pgfplotsdataymin})
        -- ({rel axis cs:0,0}
          |-{axis cs:0,\pgfplotsdataymax});
} }
}

%%%%%%%%%%%%%%%%%%%%%%%%%%%%%%%%%%%%%%%%%%%%%%%%%
%   Tufte Dot Dash 
%%%%%%%%%%%%%%%%%%%%%%%%%%%%%%%%%%%%%%%%%%%%%%%%%

\pgfplotsset{
  dot dash plot/.style={
    tufte scatter
    axis line style={opacity=0},
    tick style={thin, black},
    major tick length=0.15cm,
    xtick=data,
    xticklabels={},
    ytick=data,
    yticklabels={},
    extra x ticks={
      \pgfplotsdataxmin,
      \pgfplotsdataxmax
    },
    extra y ticks={
      \pgfplotsdataymin,
      \pgfplotsdataymax
    },
    extra tick style={
      xticklabel={\pgfmathprintnumber[
        fixed,
        fixed zerofill,
        precision=1
      ]{\tick}},
      yticklabel={\pgfmathprintnumber[
        fixed,
        fixed zerofill,
        precision=1
]{\tick}} }
} }

%%%%%%%%%%%%%%%%%%%%%%%%%%%%%%%%%%%%%%%%%%%%%%%%%
%   Tufte Box Plot 
%%%%%%%%%%%%%%%%%%%%%%%%%%%%%%%%%%%%%%%%%%%%%%%%%
\pgfplotsset{
  IDEA tufte box/.style={
	% Colour Options
	cycle list name = waterloo-light,
	ybar = 0pt,
        width  = \columnwidth,
        height = 6cm,
        	axis line style = very thin,
       	bar width=25pt,
       	font = \sffamily,
        	% X Axis
        	major x tick style = transparent,
        	enlarge x limits = {abs=1.5cm},
        	xlabel style={name=xlabel},
	% Y Axis
        	ymajorgrids = true,
        	scaled y ticks = false,
        	separate axis lines,
	axis lines* = left,
        	ymajorgrids = true,
        	major tick length = 0,
        %Legend
        	legend style={at={(xlabel.south)},yshift=2ex, xshift=-2ex, anchor=north, legend columns = 3, draw=none, column sep = 0.25cm},
        	legend image code/.code={\draw[#1] rectangle (0.25cm,0.25cm);}
 	% Boxplot specific options
	boxplot,
	boxplot/draw direction=y, 
    	%clip = false,
    	every axis plot/.style={
      	mark=o,
      	boxplot/draw direction=y,
      	boxplot/whisker extend=0,
      	boxplot/draw/box/.code={ },
	boxplot/draw/median/.code={%
	 \draw[mark size=2pt,/pgfplots/boxplot/every median/.try]
          \pgfextra
          \pgftransformshift{
            \pgfplotsboxplotpointabbox
              {\pgfplotsboxplotvalue{median}}
              {0.5}
          }
          \pgfsetfillcolor{black}
          \pgfuseplotmark{*}
          \endpgfextra
        	;
      	},
      }
  },
}


%%%%%%%%%%%%%%%%%%%%%%%%%%%%%%%%%%%%%%%%%%%%%%%%%
%   IDEA Box Plot 
%%%%%%%%%%%%%%%%%%%%%%%%%%%%%%%%%%%%%%%%%%%%%%%%%

\pgfplotsset{
    IDEA box/.style={
    	% Colour Options
	cycle list name = waterloo-light,
	ybar = 0pt,
        width  = \columnwidth,
        height = 6cm,
        axis line style = very thin,
        font = \sffamily,
        	% X Axis
        	major x tick style = transparent,
        	enlarge x limits = {abs=1cm},
        	xlabel style={name=xlabel},
	% Y Axis
        	ymajorgrids = true,
        	scaled y ticks = false,
        	separate axis lines,
	axis lines* = left,
        	ymajorgrids = true,
        	major tick length = 0,
        %Legend
        	legend style={at={(xlabel.south)},yshift=2ex, xshift=-2ex, anchor=north, legend columns = 3, draw=none, column sep = 0.25cm},
        	legend image code/.code={\draw[#1] rectangle (0.25cm,0.25cm);}
        % draw whiskers as a single line:
        boxplot/draw/whisker/.code 2 args={%
            \draw[/pgfplots/boxplot/every whisker/.try]
                (boxplot cs:##1) -- (boxplot cs:##2)
            ;
        },%
        %
        % fill the boxes:
        boxplot/every box/.style={
            fill,
        },
        % 
        % the median should be visualized as a thick white line:
        boxplot/every median/.style={
        		ultra thick,
		fill=white, draw=white,
        },
        %boxplot/draw/median/.code={%
        %    \draw[fill=white]
        %        (boxplot cs:\pgfplotsboxplotvalue{median}) circle (3pt)
        %    ;
        %},
        % draw the average as a circle
        boxplot/average=auto,
        boxplot/draw/average/.code={%
	   \draw[fill=white,draw=gray]
            	(boxplot cs:\pgfplotsboxplotvalue{average}) circle (2pt)
            ;
        },
        %
        % do not clip to avoid problems with the median:
        clip=false,
        %
        boxplot/draw direction=y,
        boxplot/whisker extend=0,
        boxplot/every whisker/.style = very thick,
        %
        %
        % width of boxes:
        boxplot/box extend=0.15,
    },
    %
    %
    rshift/.style={
        xshift=\pgfkeysvalueof{/pgfplots/rshift scale},
        legend image post style={xshift=-\pgfkeysvalueof{/pgfplots/rshift scale}},
    },
    lshift/.style={
        xshift=-\pgfkeysvalueof{/pgfplots/lshift scale},
        legend image post style={xshift=\pgfkeysvalueof{/pgfplots/lshift scale}},
    },
    rshift scale/.initial=1em,
    lshift scale/.initial=1em,
}

%%%%%%%%%%%%%%%%%%%%%%%%%%%%%%%%%%%%%%%%%%%%%%%%%
% allows multiple lines for the same Y variable in stacked plots
%%%%%%%%%%%%%%%%%%%%%%%%%%%%%%%%%%%%%%%%%%%%%%%%%
\newcommand\resetstackedplots{
\makeatletter
\pgfplots@stacked@isfirstplottrue
\makeatother
}

%%%%%%%%%%%%%%%%%%%%%%%%%%%%%%%%%%%%%%%%%%%%%%%%%
% EOF
%%%%%%%%%%%%%%%%%%%%%%%%%%%%%%%%%%%%%%%%%%%%%%%%%
\makeatother
%
%
% Then, add graphs to your paper using the following template: 
%
%\begin{tikzpicture}
%\begin{axis}[<GRAPH TYPE>,
%	% X Axis options (xmin, xmax, xtick={})
%	% Y Axis options (ymin, ymax, ytick={})
%       ]
%
%    % DATA GOES IN HERE
%
%\end{axis}
%\end{tikzpicture}
%
%
% OR, for multiple plots use this template:
%
%\begin{tikzpicture}
%\begin{groupplot}[
%   group style={
%       group size= <COL> by <ROW>,
%       x descriptions at=edge bottom,
%       y descriptions at=edge left,
%       vertical sep=0pt,
%       horizontal sep=0pt},
%  <GRAPH TYPE>, 
%	% X Axis options (xmin, xmax, xtick={})
%	% Y Axis options (ymin, ymax, ytick={})
%      ]
%
%    % DATA GOES IN HERE
%
%\end{groupplot}
%\end{tikzpicture}
%
%
%
% The following graph types are defined in this library:
%
% IDEA bar - A clustered bar graph, similar to Excel
% IDEA tufte panel - a bare-bones bar graph, suitable for stacked panel graphs in Tufte's style
% .. more that aren't listed here
%




%%%%%%%%%%%%%%%%%%%%%%%%%%%%%%%%%%%%%%%%%%%%%%%%%
%   Common Includes
%%%%%%%%%%%%%%%%%%%%%%%%%%%%%%%%%%%%%%%%%%%%%%%%%
\usepackage{pgfplots}
\usepackage{pgfplotstable}
\usepgfplotslibrary{dateplot}
\usepgfplotslibrary{groupplots}


%\usetikzlibrary{external}
%\tikzexternalize[prefix=tikz/] % Must be in main .tex file?

\usetikzlibrary{pgfplots.statistics}
\usetikzlibrary{matrix}

\usepackage{bbding}

\pgfplotsset{compat=newest}

%%%%%%%%%%%%%%%%%%%%%%%%%%%%%%%%%%%%%%%%%%%%%%%%%
%   Colour Definitions
%%%%%%%%%%%%%%%%%%%%%%%%%%%%%%%%%%%%%%%%%%%%%%%%%

\definecolor{RYB1}{RGB}{141, 211, 199}
\definecolor{RYB2}{RGB}{255, 255, 179}
\definecolor{RYB3}{RGB}{190, 186, 218}
\definecolor{RYB4}{RGB}{251, 128, 114}
\definecolor{RYB5}{RGB}{128, 177, 211}
\definecolor{RYB6}{RGB}{253, 180, 98}
\definecolor{RYB7}{RGB}{179, 222, 105}

\pgfplotscreateplotcyclelist{colorbrewer-RYB}{
{RYB1!50!black,fill=RYB1},
{RYB2!50!black,fill=RYB2},
{RYB3!50!black,fill=RYB3},
{RYB4!50!black,fill=RYB4},
{RYB5!50!black,fill=RYB5},
{RYB6!50!black,fill=RYB6},
{RYB7!50!black,fill=RYB7},
}

\pgfplotscreateplotcyclelist{colorbrewer-RYB-plain}{
{RYB1},
{RYB2},
{RYB3},
{RYB4},
{RYB5},
{RYB6},
{RYB7},
}

%%%%%%%%%%%%%%%%%%%%
% WATERLOO PALETTES - DIGITAL
%%%%%%%%%%%%%%%%%%%%

\definecolor{AHSLight}{RGB}{0, 154, 166}
\definecolor{AHSDark}{RGB}{0, 127, 138}

\definecolor{ArtsLight}{RGB}{233, 131, 0}
\definecolor{ArtsDark}{RGB}{172, 97, 0}

\definecolor{EngineeringLight}{RGB}{204, 170, 255}
\definecolor{EngineeringDark}{RGB}{87, 6, 140}

\definecolor{EnvironmentLight}{RGB}{182, 191, 0}
\definecolor{EnvironmentDark}{RGB}{116, 120, 0}

\definecolor{MathLight}{RGB}{255, 136, 221}
\definecolor{MathDark}{RGB}{224, 36, 154}
 
\definecolor{ScienceLight}{RGB}{119, 187, 225}
\definecolor{ScienceDark}{RGB}{0, 115, 207}


\definecolor{SchoolRedLight}{RGB}{247, 119, 119}
\definecolor{SchoolRedDark}{RGB}{150, 23, 46}



\pgfplotscreateplotcyclelist{waterloo-light}{
{AHSLight!50!black,fill=AHSLight},
{ArtsLight!50!black,fill=ArtsLight},
{EngineeringLight!50!black,fill=EngineeringLight},
{EnvironmentLight!50!black,fill=EnvironmentLight},
{MathLight!50!black,fill=MathLight},
{ScienceLight!50!black,fill=ScienceLight},
{SchoolRedLight!50!black,fill=SchoolRedLight},
}

\pgfplotscreateplotcyclelist{waterloo-dark}{
{AHSDark!50!black,fill=AHSDark},
{ArtsDark!50!black,fill=ArtsDark},
{EngineeringDark!50!black,fill=EngineeringDark},
{EnvironmentDark!50!black,fill=EnvironmentDark},
{MathDark!50!black,fill=MathDark},
{ScienceDark!50!black,fill=ScienceDark},
{SchoolRedDark!50!black,fill=SchoolRedDark},
}



%%%%%%%%%%%%%%%%%%%%
% WATERLOO  PALETTES - PRINT
%%%%%%%%%%%%%%%%%%%%

% FIX: CMYK is conflicting with other colour definitions in SIGCHI template

%\definecolor{AHSPrint}{CMYK}{100, 0, 30, 2}
%\definecolor{ArtsPrint}{CMYK}{0, 52, 100, 0}
%\definecolor{EngineeringPrint}{CMYK}{24, 0, 98, 8}
%\definecolor{EnvironmentPrint}{CMYK}{78, 94, 0, 0}
%\definecolor{MathPrint}{CMYK}{5, 90, 0, 0}
%\definecolor{SciencePrint}{CMYK}{90, 48, 0, 0}
%\definecolor{SchoolRedPrint}{CMYK}{3, 100, 66, 12}

%\pgfplotscreateplotcyclelist{waterloo-print}{
%{AHSPrint!50!black,fill=AHSPrint},
%{ArtsPrint!50!black,fill=ArtsPrint},
%{EngineeringPrint!50!black,fill=EngineeringPrint},
%{EnvironmentPrint!50!black,fill=EnvironmentPrint},
%{MathPrint!50!black,fill=MathPrint},
%{SciencePrint!50!black,fill=SciencePrint},
%{SchoolRedPrint!50!black,fill=SchoolRedPrint},
%}




%%%%%%%%%%%%%%%%%%%%
% 5-Point DIVERGENT PALETTES - From ColorBrewer2
%%%%%%%%%%%%%%%%%%%%

\definecolor{Likert5_SD}{RGB}{166,97,26}
\definecolor{Likert5_D}{RGB}{223,194,125}
\definecolor{Likert5_N}{RGB}{245,245,245}
\definecolor{Likert5_A}{RGB}{128,205,193}
\definecolor{Likert5_SA}{RGB}{1,133,113}

\definecolor{Likert5_1_SD}{RGB}{166,97,26}
\definecolor{Likert5_1_D}{RGB}{223,194,125}
\definecolor{Likert5_1_N}{RGB}{245,245,245}
\definecolor{Likert5_1_A}{RGB}{128,205,193}
\definecolor{Likert5_1_SA}{RGB}{1,133,113}

\definecolor{Likert5_2_SD}{RGB}{123,50,148}
\definecolor{Likert5_2_D}{RGB}{194,165,207}
\definecolor{Likert5_2_N}{RGB}{247,247,247}
\definecolor{Likert5_2_A}{RGB}{166,219,160}
\definecolor{Likert5_2_SA}{RGB}{0,136,55}



%%%%%%%%%%%%%%%%%%%%
% 7-Point DIVERGENT PALETTES - From ColorBrewer2
%%%%%%%%%%%%%%%%%%%%

% Purple -> Green
\definecolor{Likert7_0}{RGB}{118,42,131}
\definecolor{Likert7_1}{RGB}{175,141,195}
\definecolor{Likert7_2}{RGB}{231,212,232}
\definecolor{Likert7_3}{RGB}{229,229,229}
\definecolor{Likert7_4}{RGB}{217,240,211}
\definecolor{Likert7_5}{RGB}{127,191,123}
\definecolor{Likert7_6}{RGB}{27,120,55}

% Orange/Brown -> Blue
\definecolor{Likert7_2_0}{RGB}{140,81,10}
\definecolor{Likert7_2_1}{RGB}{216,179,101}
\definecolor{Likert7_2_2}{RGB}{246,232,195}
\definecolor{Likert7_2_3}{RGB}{229,229,229}
\definecolor{Likert7_2_4}{RGB}{199,234,229}
\definecolor{Likert7_2_5}{RGB}{90,180,172}
\definecolor{Likert7_2_6}{RGB}{1,102,94}

% Orange -> Purple
\definecolor{Likert7_3_0}{RGB}{179,88,6}
\definecolor{Likert7_3_1}{RGB}{241,163,64}
\definecolor{Likert7_3_2}{RGB}{254,224,182}
\definecolor{Likert7_3_3}{RGB}{229,229,229}
\definecolor{Likert7_3_4}{RGB}{216,218,235}
\definecolor{Likert7_3_5}{RGB}{153,142,195}
\definecolor{Likert7_3_6}{RGB}{84,39,136}

% Blue -> Red
\definecolor{Likert7_4_0}{RGB}{33,102,172}
\definecolor{Likert7_4_1}{RGB}{103,169,207}
\definecolor{Likert7_4_2}{RGB}{209,229,240}
\definecolor{Likert7_4_3}{RGB}{229,229,229}
\definecolor{Likert7_4_4}{RGB}{253,219,199}
\definecolor{Likert7_4_5}{RGB}{239,138,98}
\definecolor{Likert7_4_6}{RGB}{178,24,43}


%%%%%%%%%%%%%%%%%%%%
% 7-Point DIVERGENT PALETTES - Jim for Avatar Paper
%%%%%%%%%%%%%%%%%%%%

% Blue -> Green to match Cleric
\definecolor{Cleric_0}{RGB}{33,102,172}
\definecolor{Cleric_1}{RGB}{103,169,207}
\definecolor{Cleric_2}{RGB}{209,229,240}
\definecolor{Cleric_3}{RGB}{229,229,229}
\definecolor{Cleric_4}{RGB}{217,240,211}
\definecolor{Cleric_5}{RGB}{127,191,123}
\definecolor{Cleric_6}{RGB}{27,120,55}

% Orange -> Red
\definecolor{Monster_0}{RGB}{178,24,43}
\definecolor{Monster_1}{RGB}{239,138,98}
\definecolor{Monster_2}{RGB}{253,219,199}
\definecolor{Monster_3}{RGB}{229,229,229}
\definecolor{Monster_4}{RGB}{216,218,235}
\definecolor{Monster_5}{RGB}{153,142,195}
\definecolor{Monster_6}{RGB}{84,39,136}


\pgfplotscreateplotcyclelist{avatar}{Cleric_0, Monster_0}

%%%%%%%%%%%%%%%%%%%%%%%%%%%%%%%%%%%%%%%%%%%%%%%%%
%   Outer South Legend Hack
%%%%%%%%%%%%%%%%%%%%%%%%%%%%%%%%%%%%%%%%%%%%%%%%%
\makeatletter
\pgfplotsset{
    every axis x label/.append style={
        alias=current axis xlabel
    },
    legend pos/outer south/.style={
        /pgfplots/legend style={
            at={%
                (%
                \@ifundefined{pgf@sh@ns@current axis xlabel}%
                {xticklabel cs:0.5}%
                {current axis xlabel.south}%
                )%
            },
            anchor=north
        }
    }
}

%%%%%%%%%%%%%%%%%%%%%%%%%%%%%%%%%%%%%%%%%%%%%%%%%
%   Grid Figures Label Hack
%%%%%%%%%%%%%%%%%%%%%%%%%%%%%%%%%%%%%%%%%%%%%%%%%

\pgfplotsset{
    groupplot xlabel/.initial={},
    every groupplot x label/.style={
        at={($({\pgfplots@group@name\space c1r\pgfplots@group@rows.west}|-{\pgfplots@group@name\space c1r\pgfplots@group@rows.outer south})!0.5!({\pgfplots@group@name\space c\pgfplots@group@columns r\pgfplots@group@rows.east}|-{\pgfplots@group@name\space c\pgfplots@group@columns r\pgfplots@group@rows.outer south})$)},
        anchor=north,
    },
    groupplot ylabel/.initial={},
    every groupplot y label/.style={
            rotate=90,
        at={($({\pgfplots@group@name\space c1r1.north}-|{\pgfplots@group@name\space c1r1.outer
west})!0.5!({\pgfplots@group@name\space c1r\pgfplots@group@rows.south}-|{\pgfplots@group@name\space c1r\pgfplots@group@rows.outer west})$)},
        anchor=south
    },
    execute at end groupplot/.code={%
      \node [/pgfplots/every groupplot x label]
{\pgfkeysvalueof{/pgfplots/groupplot xlabel}};  
      \node [/pgfplots/every groupplot y label] 
{\pgfkeysvalueof{/pgfplots/groupplot ylabel}};  
    }
}

\def\endpgfplots@environment@groupplot{%
    \endpgfplots@environment@opt%
    \pgfkeys{/pgfplots/execute at end groupplot}%
    \endgroup%
}



%%%%%%%%%%%%%%%%%%%%%%%%%%%%%%%%%%%%%%%%%%%%%%%%%
%   IDEA Line
%%%%%%%%%%%%%%%%%%%%%%%%%%%%%%%%%%%%%%%%%%%%%%%%%
\pgfplotsset{
  IDEA line/.style={
 	cycle list name = avatar,
        width  = \columnwidth,
        height = 6cm,
        line width=10pt,
        	axis line style = very thin,
        font = \sffamily,
        % X Axis
        %major x tick style = transparent,
        xlabel style={name=xlabel},
	% Y Axis
        ymajorgrids = true,
        scaled y ticks = false,
        separate axis lines,
	axis lines* = left,
        ymajorgrids = true,
        major tick length = -5,
        %Legend
        legend style={at={(xlabel.south)},yshift=2ex, xshift=-2ex, anchor=north, legend columns = 3, draw=none, column sep = 0.25cm},
        	%legend image code/.code={%
            %        \draw[#1,fill] circle (0.075cm);
            %    },
	}
}

%%%%%%%%%%%%%%%%%%%%%%%%%%%%%%%%%%%%%%%%%%%%%%%%%
%   IDEA Bar / A Clustered Bar Graph
%%%%%%%%%%%%%%%%%%%%%%%%%%%%%%%%%%%%%%%%%%%%%%%%%
\pgfplotsset{
  IDEA bar/.style={
  	cycle list name = waterloo-light,
	ybar = 0pt,
        width  = \columnwidth,
        height = 6cm,
        	axis line style = very thin,
       	bar width=25pt,
        font = \sffamily,
        	% X Axis
        	major x tick style = transparent,
        	enlarge x limits = {abs=1.5cm},
	xlabel={a},
        	xlabel style={name=xlabel},
	% Y Axis
        	ymajorgrids = true,
        	ylabel = {Selection Time (s)},
        	scaled y ticks = false,
        	separate axis lines,
	axis lines* = left,
        	ymajorgrids = true,
        	major tick length = 0,
        %Legend
        	legend style={at={(xlabel.south)},yshift=2ex, xshift=-2ex, anchor=north, legend columns = 3, draw=none, column sep = 0.25cm},
        	legend image code/.code={\draw[#1] rectangle (0.25cm,0.25cm);}
	}
}

%%%%%%%%%%%%%%%%%%%%%%%%%%%%%%%%%%%%%%%%%%%%%%%%%
%   IDEA StackedBar / A Clustered Bar Graph
%%%%%%%%%%%%%%%%%%%%%%%%%%%%%%%%%%%%%%%%%%%%%%%%%
\pgfplotsset{
  IDEA stacked bar/.style={
  	ybar stacked,
  	cycle list name = waterloo-light,
        width  = \columnwidth,
        height = 6cm,
        	axis line style = very thin,
       	bar width=25pt,
       	font = \sffamily,
        	% X Axis
        	major x tick style = transparent,
        	enlarge x limits = {abs=1.5cm},
	xlabel={a},
        	xlabel style={name=xlabel},
	% Y Axis
        	ymajorgrids = true,
        	ylabel = {Selection Time (s)},
        	scaled y ticks = false,
        	separate axis lines,
	axis lines* = left,
        	ymajorgrids = true,
        	major tick length = 0,
        %Legend
        	legend style={at={(xlabel.south)},yshift=2ex, xshift=-2ex, anchor=north, legend columns = 3, draw=none, column sep = 0.25cm},
        	legend image code/.code={\draw[#1] rectangle (0.25cm,0.25cm);}
	}
}


%%%%%%%%%%%%%%%%%%%%%%%%%%%%%%%%%%%%%%%%%%%%%%%%%
%   IDEA Likhert / Divergent Bar
%%%%%%%%%%%%%%%%%%%%%%%%%%%%%%%%%%%%%%%%%%%%%%%%%
\pgfplotsset{
  IDEA Likert/.style={
 	xbar stacked,
        	axis line style = very thin,
       	bar width=5pt,
        	% X Axis
        	%major x tick style = transparent,
	xlabel={a},
        	xlabel style={name=xlabel},
	xtick = {},
	font = \sffamily,
	% Y Axis
	        %axis y line=none,
	        major y tick style = transparent,
        	%ymajorgrids = true,
        	scaled y ticks = false,
        	separate axis lines,
	axis lines* = left,
        	major tick length = -2,
        %Legend
        	legend style={at={(xlabel.south)},yshift=-2ex, xshift=-2ex, anchor=north, legend columns = 3, draw=none, column sep = 0.25cm},
        	legend image code/.code={\draw[#1] rectangle (0.25cm,0.25cm);}
	}
}

%%%%%%%%%%%%%%%%%%%%%%%%%%%%%%%%%%%%%%%%%%%%%%%%%
%   Tufte Stacked Panel 
%%%%%%%%%%%%%%%%%%%%%%%%%%%%%%%%%%%%%%%%%%%%%%%%%
\pgfplotsset{
  IDEA tufte panel/.style={
  cycle list name = waterloo-light,
 ybar, 
	% Generic Bar Graph Options
        width = 1.1\columnwidth,
        height = .325\columnwidth,
        font = \sffamily,
	% X Axis
	enlarge x limits = {abs=.35cm},
	axis line style = very thin,
	% Y Axis
	yticklabel=\empty,
        separate axis lines,
	y axis line style= { draw opacity=0 },
	axis lines* = left,
        axis on top,
        ymajorgrids = true,
        major tick length = 0,
        grid style = {white},
	% Legend Formatting
	title style = {yshift=-.75cm,font=\tiny}
	}
}

%%%%%%%%%%%%%%%%%%%%%%%%%%%%%%%%%%%%%%%%%%%%%%%%%
%   Tufte Range Frame
%%%%%%%%%%%%%%%%%%%%%%%%%%%%%%%%%%%%%%%%%%%%%%%%%
\pgfplotsset{
range frame/.style={
    tick align=outside,
    axis line style={opacity=0},
    after end axis/.code={
      \draw ({rel axis cs:0,0}
          -|{axis cs:\pgfplotsdataxmin,0})
        -- ({rel axis cs:0,0}
          -|{axis cs:\pgfplotsdataxmax,0});
      \draw ({rel axis cs:0,0}
          |-{axis cs:0,\pgfplotsdataymin})
        -- ({rel axis cs:0,0}
          |-{axis cs:0,\pgfplotsdataymax});
} }
}

%%%%%%%%%%%%%%%%%%%%%%%%%%%%%%%%%%%%%%%%%%%%%%%%%
%   Tufte Dot Dash 
%%%%%%%%%%%%%%%%%%%%%%%%%%%%%%%%%%%%%%%%%%%%%%%%%

\pgfplotsset{
  dot dash plot/.style={
    tufte scatter
    axis line style={opacity=0},
    tick style={thin, black},
    major tick length=0.15cm,
    xtick=data,
    xticklabels={},
    ytick=data,
    yticklabels={},
    extra x ticks={
      \pgfplotsdataxmin,
      \pgfplotsdataxmax
    },
    extra y ticks={
      \pgfplotsdataymin,
      \pgfplotsdataymax
    },
    extra tick style={
      xticklabel={\pgfmathprintnumber[
        fixed,
        fixed zerofill,
        precision=1
      ]{\tick}},
      yticklabel={\pgfmathprintnumber[
        fixed,
        fixed zerofill,
        precision=1
]{\tick}} }
} }

%%%%%%%%%%%%%%%%%%%%%%%%%%%%%%%%%%%%%%%%%%%%%%%%%
%   Tufte Box Plot 
%%%%%%%%%%%%%%%%%%%%%%%%%%%%%%%%%%%%%%%%%%%%%%%%%
\pgfplotsset{
  IDEA tufte box/.style={
	% Colour Options
	cycle list name = waterloo-light,
	ybar = 0pt,
        width  = \columnwidth,
        height = 6cm,
        	axis line style = very thin,
       	bar width=25pt,
       	font = \sffamily,
        	% X Axis
        	major x tick style = transparent,
        	enlarge x limits = {abs=1.5cm},
        	xlabel style={name=xlabel},
	% Y Axis
        	ymajorgrids = true,
        	scaled y ticks = false,
        	separate axis lines,
	axis lines* = left,
        	ymajorgrids = true,
        	major tick length = 0,
        %Legend
        	legend style={at={(xlabel.south)},yshift=2ex, xshift=-2ex, anchor=north, legend columns = 3, draw=none, column sep = 0.25cm},
        	legend image code/.code={\draw[#1] rectangle (0.25cm,0.25cm);}
 	% Boxplot specific options
	boxplot,
	boxplot/draw direction=y, 
    	%clip = false,
    	every axis plot/.style={
      	mark=o,
      	boxplot/draw direction=y,
      	boxplot/whisker extend=0,
      	boxplot/draw/box/.code={ },
	boxplot/draw/median/.code={%
	 \draw[mark size=2pt,/pgfplots/boxplot/every median/.try]
          \pgfextra
          \pgftransformshift{
            \pgfplotsboxplotpointabbox
              {\pgfplotsboxplotvalue{median}}
              {0.5}
          }
          \pgfsetfillcolor{black}
          \pgfuseplotmark{*}
          \endpgfextra
        	;
      	},
      }
  },
}


%%%%%%%%%%%%%%%%%%%%%%%%%%%%%%%%%%%%%%%%%%%%%%%%%
%   IDEA Box Plot 
%%%%%%%%%%%%%%%%%%%%%%%%%%%%%%%%%%%%%%%%%%%%%%%%%

\pgfplotsset{
    IDEA box/.style={
    	% Colour Options
	cycle list name = waterloo-light,
	ybar = 0pt,
        width  = \columnwidth,
        height = 6cm,
        axis line style = very thin,
        font = \sffamily,
        	% X Axis
        	major x tick style = transparent,
        	enlarge x limits = {abs=1cm},
        	xlabel style={name=xlabel},
	% Y Axis
        	ymajorgrids = true,
        	scaled y ticks = false,
        	separate axis lines,
	axis lines* = left,
        	ymajorgrids = true,
        	major tick length = 0,
        %Legend
        	legend style={at={(xlabel.south)},yshift=2ex, xshift=-2ex, anchor=north, legend columns = 3, draw=none, column sep = 0.25cm},
        	legend image code/.code={\draw[#1] rectangle (0.25cm,0.25cm);}
        % draw whiskers as a single line:
        boxplot/draw/whisker/.code 2 args={%
            \draw[/pgfplots/boxplot/every whisker/.try]
                (boxplot cs:##1) -- (boxplot cs:##2)
            ;
        },%
        %
        % fill the boxes:
        boxplot/every box/.style={
            fill,
        },
        % 
        % the median should be visualized as a thick white line:
        boxplot/every median/.style={
        		ultra thick,
		fill=white, draw=white,
        },
        %boxplot/draw/median/.code={%
        %    \draw[fill=white]
        %        (boxplot cs:\pgfplotsboxplotvalue{median}) circle (3pt)
        %    ;
        %},
        % draw the average as a circle
        boxplot/average=auto,
        boxplot/draw/average/.code={%
	   \draw[fill=white,draw=gray]
            	(boxplot cs:\pgfplotsboxplotvalue{average}) circle (2pt)
            ;
        },
        %
        % do not clip to avoid problems with the median:
        clip=false,
        %
        boxplot/draw direction=y,
        boxplot/whisker extend=0,
        boxplot/every whisker/.style = very thick,
        %
        %
        % width of boxes:
        boxplot/box extend=0.15,
    },
    %
    %
    rshift/.style={
        xshift=\pgfkeysvalueof{/pgfplots/rshift scale},
        legend image post style={xshift=-\pgfkeysvalueof{/pgfplots/rshift scale}},
    },
    lshift/.style={
        xshift=-\pgfkeysvalueof{/pgfplots/lshift scale},
        legend image post style={xshift=\pgfkeysvalueof{/pgfplots/lshift scale}},
    },
    rshift scale/.initial=1em,
    lshift scale/.initial=1em,
}

%%%%%%%%%%%%%%%%%%%%%%%%%%%%%%%%%%%%%%%%%%%%%%%%%
% allows multiple lines for the same Y variable in stacked plots
%%%%%%%%%%%%%%%%%%%%%%%%%%%%%%%%%%%%%%%%%%%%%%%%%
\newcommand\resetstackedplots{
\makeatletter
\pgfplots@stacked@isfirstplottrue
\makeatother
}

%%%%%%%%%%%%%%%%%%%%%%%%%%%%%%%%%%%%%%%%%%%%%%%%%
% EOF
%%%%%%%%%%%%%%%%%%%%%%%%%%%%%%%%%%%%%%%%%%%%%%%%%
\makeatother