%%%%%%%%%%%%%%%%%%%%%%%%%%%%%%%%%%%%%%%%%%%%%%%%%%%%%%%%%%%%%%%%%%%%%%%%%%%%%
\section{\LaTeX\ and IDEA Plots}
%%%%%%%%%%%%%%%%%%%%%%%%%%%%%%%%%%%%%%%%%%%%%%%%%%%%%%%%%%%%%%%%%%%%%%%%%%%%%

Ok, so why \LaTeX?

\begin{enumerate}
    \item Most formatting is done `for free' by the SIG CHI template
    \item It's free, and the University of Waterloo provides a Pro Overleaf account
    \item Nice integration with other tools like R, PGFPLots, etc. 
    \item Papers get rejected all the time, and reformatting is very fast and easy
\end{enumerate}





\subsection{IDEA Plots}

- These are just some templates I developed that encapsulate a lot of the common tasks for creating scientific figures in HCI papers
- e.g., creating axes, labels, legends, 
- I also lean towards a minimalist/ Tuftian feel -- this is a good starting point, you can always make things more complex where needed


Pros:
\begin{enumerate}
    \item Draws the graphs directly into your PDF, no worry about scalability (look at these graphs zoomed in on a 5K display)
    \item Uses same fonts etc. as rest of document, so graphs feel like they 'fit'
    \item Lots of control over width/height, colours, every aspect of
    \item Can draw directly from a data file, so nice integration with analysis tools like Excel and R
\end{enumerate}


Cons:
\begin{enumerate}
    \item Can be complicated -- try to provide these templates to help mitigate that issue
    \item Sometimes it just won't work the way it's supposed to
    \item PGFPlots is updated from time to time, some underlying features may change/break
\end{enumerate}


\subsection{IDEA Plots Features at a Glance}