
\vspace{2em}
\begin{center}
    \emph{Excellence is not a destination, it is a continuous journey that never ends.  --- Brian Tracy}
\end{center}

%%%%%%%%%%%%%%%%%%%%%%%%%%%%%%%%%%%%%%%%%%%%%%%%%%%%%%%%%%%%%%%%%%%%%%%%%%%%%
\section{\LaTeX\ and IDEA Plots}
%%%%%%%%%%%%%%%%%%%%%%%%%%%%%%%%%%%%%%%%%%%%%%%%%%%%%%%%%%%%%%%%%%%%%%%%%%%%%

\LaTeX is far from a perfect writing tool. For one thing, it's a `typesetting environment' and not a word processor, so little things like the difference between a ` and a ' can end up being important. When you're writing a paper, you generally want to focus on the words you're writing, and not getting the syntax right. So why use \LaTeX?

\begin{enumerate}
    \item Most formatting is done `for free' by the SIG CHI template, and it does a great job of this for you
    \item The University of Waterloo provides a Pro Overleaf account (think Google Docs for academic writing) if you login with your UW userid
    \item \LaTeX\ integrates well with other (free) tools like R, PGFPLots, etc. 
    \item Papers get rejected all the time, and reformatting is very fast and easy
\end{enumerate}

Since \LaTeX\ can be complex, this project also embodies some changes to the template that make our typical workflow a little easier:

\begin{enumerate}
    \item Sections are separated out into files, so that authors can edit each section without interrupting each other
    \item Additional packages:
    \item IDEA Plots --- see below. 
    \item ... 
\end{enumerate}



\subsection{PGFPlots and IDEA Plots}
As as I was completeing my PhD, I decided to spend some time trying to learn to make better figures since this was something I expected to spend a lot of time doing over the rest of my career, and I might as well think about how to do it well from the start. Some of the things I'd hoped to improve on were using scalable graphics, having better control over the width/height of figures, and being able to use experimental data itself to create the images (better fit with a typical academic workflow). I also prefer a minimalist feel, and wanted my papers to reflect that (And even if you prefer something different, this is a good starting point, you can always make things more complex where needed). 

So after doing some research, I found that \texttt{PGFPlots} was a good fit for my needs. But it wasn't perfect, and in particular can be complicated to get right. So in the interest of saving the time of re-inventing the wheel, and (hopefully) enabling others to use my work for their own papers, I decided to create a set of templates that would encapsulate the common types of graphs I use in HCI research. I called these templates \texttt{IDEA Plots}. As I developed the templates, they grew to include style choices, colour palettes, and some common shortcuts for layout. I expect they'll continue to evolve for a while to come.  

All of the examples in this document use the templates from \texttt{IDEA Plots}, with some in-place customization based on the needs of the graph. If you'd like to use those templates in your own work, be sure to include \texttt{IDEAPlots.tex} in your own project, and to \texttt{\textbackslash include\{IDEAPlots.tex\}} early on in your paper.  